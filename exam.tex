\documentclass[11pt,a4paper,oneside]{scrartcl}
\usepackage[utf8]{inputenc}
\usepackage[english,russian]{babel}
\usepackage[top=1cm,bottom=1cm,left=1cm,right=1cm]{geometry}
\usepackage{amsmath}
\usepackage{amssymb}
\usepackage{stmaryrd}
\usepackage{cmll}
\usepackage{xcolor}
\usepackage{proof}
\usepackage{comment}
\usepackage{titletoc}
\usepackage{amsthm}
\newtheorem{thm}{Теорема}
\newtheorem{dfn}{Определение}
\theoremstyle{definition}
\newtheorem{exm}{Пример}
\newtheorem{axm}{Аксиома}
\newtheorem{lmm}{Лемма}
\newtheorem{snote}{Замечание}
\setcounter{tocdepth}{1}
\usepackage{hyperref}
\usepackage{tikz}
\usetikzlibrary{hobby,fit,backgrounds,calc,shapes.geometric,patterns}
\hypersetup{
    colorlinks=true,
    linkcolor=blue,
    filecolor=magenta,      
    urlcolor=cyan,
}
\usepackage{graphicx} % Required for inserting images

\title{Matlog Exam}
\author{Artemiy Maslov}
\date{January 2026}

\begin{document}

\maketitle

\newpage

\tableofcontents

\newpage

\section{Исчисление высказываний. Предметный язык и язык исследователя (метаязык). 
Язык исчисления высказываний. Оценка высказываний, общезначимость, следование.
Доказуемость, гипотезы (контекст), выводимость. Корректность, полнота, противоречивость и 
непротиворечивость (эквивалентные формулировки). Теорема о дедукции для исчисления высказываний.}

Для задания\textbf{ исчисления высказываний} необходим предметный язык, теория моделей и теория доказательств.

\textbf{Предметный язык}(формальный язык, тексты на котором мы будем анализировать) КИВ состоит из высказываний.

\textbf{Высказывание} - это строка, которая является либо атомарным высказыванием(пропозициональной переменной), либо составным высказыванием, построенным с помощью отрицания, конъюнкции, дизъюнкции или импликации.

\textbf{Метаязык} - язык, с помощью которого мы анализируем предметный язык.

\text {Для построения }\textbf{оценки высказываний} необходимо задать множество истинностных значений $V = \{\textit{И},\textit{Л}\,\}$ \text{ и определить функцию оценки} $f: \mathcal{P} \rightarrow V$. \text{Оценки высказываний задаются рекурсивно, интуитивным способом.}

Если $\alpha$ истинна при любой оценке переменных, то она \textbf{общезначима}: $\models \alpha$.

Если $\alpha$ истинна при любой оценке переменных, при которой истинны 
высказывания $\gamma_1, \dots, \gamma_n$, то $\alpha$ --- \textbf{следствие} этих высказываний: $\gamma_1, \dots, \gamma_n \models \alpha$.

Схема высказываний - высказывания с метапеременными.

Будем говорить, что высказывание $\sigma$ строится по схеме $\textit{Ш}$, 
если существует такая замена метапеременных $\textit{ч}_1$, $\textit{ч}_2$, ..., $\textit{ч}_n$ 
в схеме $\textit{Ш}$ на какие-либо выражения $\varphi_1$, $\varphi_2$, ..., $\varphi_n$, 
что после её проведения получается высказывание $\sigma$: $\sigma = \textit{Ш}[\textit{ч}_1 := \varphi_1][\textit{ч}_2 := \varphi_2]...[\textit{ч}_n := \varphi_n]$.

Аксиомы ИВ - следующие схемы высказываний:

\begin{tabular}{ll}
(1) & $\alpha \rightarrow \beta \rightarrow \alpha$ \\
(2) & $(\alpha \rightarrow \beta) \rightarrow (\alpha \rightarrow \beta \rightarrow \gamma) \rightarrow (\alpha \rightarrow \gamma)$ \\
(3) & $\alpha \rightarrow \beta \rightarrow \alpha \& \beta$\\
(4) & $\alpha \& \beta \rightarrow \alpha$\\
(5) & $\alpha \& \beta \rightarrow \beta$\\
(6) & $\alpha \rightarrow \alpha \vee \beta$\\
(7) & $\beta \rightarrow \alpha \vee \beta$\\
(8) & $(\alpha \rightarrow \gamma) \rightarrow (\beta \rightarrow \gamma) \rightarrow (\alpha \vee \beta \rightarrow \gamma)$\\
(9) & $(\alpha \rightarrow \beta) \rightarrow (\alpha \rightarrow \neg \beta) \rightarrow \neg \alpha$\\
(10) & $\neg \neg \alpha \rightarrow \alpha$
\end{tabular}

Правило вывода MP - Если имеет место $\alpha$ и $\alpha\rightarrow\beta$, то имеет место $\beta$.\\

\textbf{Доказательством в ИВ} назовём конечную последовательность высказываний $\delta_1, \delta_2, \dots, \delta_n$, причём каждое $\delta_i$ либо:
\begin{itemize}
    \item является аксиомой(существует замена переменных для какой-то схемы)
    \item получается из $\delta_1,\dots,\delta_{i-1}$ по правилу Modus Ponens.
\end{itemize}

\textbf{Доказательство формулы} $\alpha$ - такое доказательство $\delta_1, \delta_2, \dots, \delta_n$, что $\alpha\equiv\delta_n$.\\

Формула \textbf{доказуема}, если существует доказательство.\\

\textbf{Вывод} формулы $\alpha$ из гипотез $\gamma_1,\dots,\gamma_k$ - последовательность
$\delta_1,\dots,\delta_n$, причём каждое $\delta_i$ либо:
\begin{itemize}
\item является аксиомой;
\item либо получается по правилу Modus Ponens из предыдущих;
\item либо является одной из гипотез.
\end{itemize}

Формула $\alpha$ \textbf{выводима} из гипотез $\gamma_1,\dots,\gamma_k$, если существует её вывод.\\

\textbf{Контекст} - список формул.\\

Теория \textbf{корректна}, если любое доказуемое в ней утверждение общезначимо.\\

Теория \textbf{полна}, если любое общезначимое в ней утверждение доказуемо.\\

Теория \textbf{противоречива}, если найдётся такая формула $\alpha$, что $\vdash\alpha$ и $\vdash\neg\alpha$. 

Эквивалентно в КИВ и ИИВ:
\begin{itemize}
\item теория непротиворечива;
\item $\not\vdash A \with \neg A$;
\item Найдётся $\alpha$, что $\not\vdash\alpha$.
\end{itemize}

(\textbf{Теорема о дедукции}) $\Gamma,\alpha\vdash\beta$ выполнено тогда и только тогда, когда выполнено $\Gamma\vdash\alpha\rightarrow\beta$.

$\Leftarrow$

Пусть $\Gamma \vdash \alpha \rightarrow \beta$. Тогда существует вывод
\[
\delta_1, \delta_2, \dots, \delta_{n-1}, \alpha \rightarrow \beta,
\]
где каждая $\delta_i$ либо принадлежит $\Gamma$, либо получена по правилам вывода.  
Добавим к этому выводу формулу $\alpha$ (как гипотезу) и применим \textit{modus ponens}:
\[
\delta_1, \dots, \delta_{n-1},\; \alpha \rightarrow \beta,\; \alpha,\; \beta.
\]
Теперь $\beta$ выводима из $\Gamma \cup \{ \alpha \}$, т.е. $\Gamma, \alpha \vdash \beta$. 

$\Rightarrow$

(индукция по длине вывода). Если $\delta_1, \dots, \delta_n$ --- вывод
$\Gamma,\alpha\vdash\delta_n$, то найдётся вывод $\zeta_k$ для $\Gamma\vdash\alpha\rightarrow\delta_n$,
причём $\zeta_1 \equiv \alpha\rightarrow\delta_1, \dots, \zeta_n \equiv \alpha\rightarrow\delta_n$.

\begin{itemize}
\item База $(n=1)$: частный случай перехода (без M.P.).

\item Переход. Пусть $\delta_1, \dots, \delta_{n+1}$ --- исходный вывод. И пусть (по индукционному предположению)
уже по начальному фрагменту $\delta_1, \dots, \delta_n$ построен вывод $\zeta_k$ утверждения 
$\Gamma\vdash\alpha\rightarrow\delta_n$. 

Но $\delta_{n+1}$ как-то был обоснован --- разберём случаи:
\begin{enumerate}
\item $\delta_{n+1}$ --- аксиома или $\delta_{n+1} \in \Gamma$ %(выполнено без доказательства в новом выводе)\pause
\item $\delta_{n+1}\equiv\alpha$\pause
\item $\delta_{n+1}$ --- Modus Ponens из $\delta_j$ и 
$\delta_k \equiv \delta_j\rightarrow\delta_{n+1}$.
\end{enumerate}

В каждом из случаев можно дополнить черновик до полноценного вывода.

\section{Теорема о полноте исчисления высказываний. Условное отрицание. 14 лемм о связках. 
Лемма об устранении посылок. Доказательство теоремы.}

(\textbf{Теорема о полноте исчисления высказываний}) Если $\models\alpha$, то $\vdash\alpha$.\\

\begin{proof}
Пусть $\llbracket\alpha\rrbracket = x$. Тогда \textbf{условное отрицание} задается как $\llparenthesis\alpha\rrparenthesis$:
$$\llparenthesis\alpha\rrparenthesis = \left\{\begin{array}{ll}\alpha, & x = \textnormal{И}\\
       \neg\alpha, & x = \textnormal{Л}\end{array}\right.$$

(\textbf{Лемма о связках}) $\llparenthesis\varphi\rrparenthesis, \llparenthesis\psi\rrparenthesis \vdash \llparenthesis\varphi\star\psi\rrparenthesis$

\begin{center}\begin{tabular}{rclp{1cm}rcl}
$\neg\varphi, \neg\psi$&$ \vdash $&$\neg (\varphi \with \psi)$& & $\neg\varphi, \neg\psi$&$ \vdash $&$     (\varphi \rightarrow  \psi)$ \\
$\neg\varphi,     \psi$&$ \vdash $&$\neg (\varphi \with \psi)$& &$\neg\varphi,     \psi$&$ \vdash $&$     (\varphi \rightarrow  \psi)$ \\
$    \varphi, \neg\psi$&$ \vdash $&$\neg (\varphi \with \psi)$& &$ \varphi, \neg\psi$&$ \vdash $&$\neg (\varphi \rightarrow  \psi)$ \\
$    \varphi,     \psi$&$ \vdash $&$     (\varphi \with \psi)$& &$    \varphi,     \psi$&$ \vdash $&$     (\varphi \rightarrow  \psi)$ \\
$\neg\varphi, \neg\psi$&$ \vdash $&$\neg (\varphi \vee  \psi)$& &$    \varphi          $&$ \vdash $&$     \neg\neg\varphi$ \\
$\neg\varphi,     \psi$&$ \vdash $&$     (\varphi \vee  \psi)$& &$\neg\varphi          $&$ \vdash $&$         \neg\varphi$\\
$    \varphi, \neg\psi$&$ \vdash $&$     (\varphi \vee  \psi)$ \\
$    \varphi,     \psi$&$ \vdash $&$     (\varphi \vee  \psi)$
\end{tabular}\end{center}

(\textbf{Лемма об условном отрицании формул}) Пусть пропозициональные переменные $\Xi := \{X_1, \dots, X_n\}$ ---
все переменные, которые используются в формуле $\alpha$. И пусть
задана некоторая оценка переменных.

Тогда, $\llparenthesis \Xi \rrparenthesis \vdash\llparenthesis\alpha\rrparenthesis$

Индукция по длине формулы $\alpha$.
\begin{itemize}
\item База: формула $\alpha$ --- атомарная, т.е. $\alpha \equiv X_i$. Тогда при любом $\Xi$ выполнено 
$\llparenthesis\Xi\rrparenthesis^{X_i := \text{И}} \vdash X_i$ и $\llparenthesis\Xi\rrparenthesis^{X_i := \text{Л}} \vdash \neg X_i$.
\item Переход: $\alpha \equiv \varphi\star\psi$, причём $\llparenthesis\Xi\rrparenthesis\vdash\llparenthesis\varphi\rrparenthesis$
и $\llparenthesis\Xi\rrparenthesis\vdash\llparenthesis\psi\rrparenthesis$
Тогда построим вывод: 

\begin{tabular}{lll}
$(1)\dots(n)$ & $\llparenthesis\varphi\rrparenthesis$ & индукционное предположение\\
$(n+1)\dots(k)$ & $\llparenthesis\psi\rrparenthesis$ & индукционное предположение\\
$(k+1)\dots(l)$ & $\llparenthesis\varphi\star\psi\rrparenthesis$ & 
  лемма о связках: $\llparenthesis\varphi\rrparenthesis$ и $\llparenthesis\psi\rrparenthesis$ доказаны выше,\\
  & & значит, их можно использовать как гипотезы
\end{tabular}
\end{itemize}


(\textbf{Лемма об устранении посылок}) Пусть при всех оценках переменных
$\llparenthesis\Xi\rrparenthesis \vdash \alpha$, тогда
$\vdash\alpha$.

Индукция по количеству переменных $n$.

\begin{itemize}
\item База: $n=0$. Тогда $\vdash\alpha$ есть из условия.\item Переход: пусть $\llparenthesis X_1, X_2,  \dots X_{n+1} \rrparenthesis \vdash \alpha$.
Рассмотрим $2^n$ пар выводов: $$\llparenthesis X_1, X_2, \dots X_n\rrparenthesis,X_{n+1} \vdash \alpha\quad\quad\llparenthesis X_1, X_2, \dots X_n\rrparenthesis,\neg X_{n+1} \vdash \alpha$$
По лемме об исключении допущения тогда
$$\llparenthesis X_1, X_2, \dots X_n \rrparenthesis \vdash \alpha$$
\end{itemize}

В виду общезначимости, $\llparenthesis\Xi\rrparenthesis\vdash\alpha$. 
При этом, $\llparenthesis X_1, X_2, \dots X_n \rrparenthesis  \vdash \alpha$ при всех оценках
переменных $X_1, \dots X_n$. Значит, $\vdash\alpha$ по индукционному предположению.

\end{proof}

\section{Топологические пространства. Определение. Примеры (топология стрелки, Зарисского, 
топология на деревьях). Открытые и замкнутые множества. Связность. Компактность. Непрерывные функции.}

\textbf{Топологическим пространством} называется упорядоченная пара $\langle X, \Omega \rangle$,
где $X$ --- некоторое множество, а $\Omega \subseteq \mathcal{P}(X)$, причём:
\begin{enumerate}
\item $\varnothing, X \in \Omega$
\item если $A_1, \dots, A_n \in \Omega$, то $A_1 \cap A_2 \cap \dots \cap A_n \in \Omega$;
\item если $\{A_\alpha\}$ --- семейство множеств из $\Omega$, то и $\bigcup_\alpha A_\alpha \in \Omega$.
\end{enumerate}


Множество $\Omega$ называется топологией. Элементы $\Omega$ называются \textbf{открытыми множествами}.\\

Внутренность множества $A^\circ$ --- наибольшее $T$, что $T \in \Omega$ и $T \subseteq A$.\\

Функция $f: X \rightarrow Y$ \textbf{непрерывна}, если прообраз любого открытого множества открыт.\\

Множество \textbf{компактно}, если из любого его открытого покрытия можно выбрать конечное
подпокрытие.\\

Пространство $\langle X, \Omega\rangle$ \textbf{связно}, если нет $A,B \in \Omega$, что $A\cup B = X$,
$A \cap B = \varnothing$ и $A,B \ne \varnothing$.\\
\newpage

\textbf{Примеры}
\begin{itemize}
\item Топология стрелки

X = $\mathbb{R}$,

$\Omega = \{\, U \subset \mathbb{R} \mid U = \bigcup_{\alpha} [a_\alpha,b_\alpha),\ a_\alpha<b_\alpha \,\}.$
\item Топология Зарисского

X = k, 

$\Omega = \{\, U \subset X \mid X \setminus U$ \text{ — конечное множество} $\,\}.$
\item Топология на деревьях

Пусть некоторый лес задан конечным множеством вершин $V$ и
отношением $(\preceq)$, связывающим предков и потомков ($a \preceq b$, если $b$ --- потомок $a$). Тогда подмножество его вершин $X\subseteq V$ назовём открытым, 
если из $a \in X$ и $a \preceq b$ следует, что $b \in X$.

\end{itemize}

\section{ Гильбертов вывод и натуральный вывод. Интуиционистское исчисление высказываний.
Доказательства чистого существования. BHK-интерпретация. 
Закон исключённого третьего, принцип взрыва, связь с КИВ и ИИВ.
Решётки. Дистрибутивная решётка. Псевдодополнение. Булевы и псевдобулевы алгебры.}

\textbf{Натуральный вывод, связь с КИВ}
\begin{itemize}
\item Формулы языка (секвенции) имеют вид: $\Gamma\vdash\alpha$.
\item Аксиома:\\$\infer[\text{(акс.)}]{\Gamma,\alpha\vdash\alpha}{\vphantom{\Gamma}}$ 

\item Правила введения связок:\\$\infer{\Gamma\vdash\alpha\rightarrow\beta}{\Gamma,\alpha\vdash\beta}\quad\quad\infer{\Gamma\vdash\alpha\vee\beta}{\Gamma\vdash\alpha}$, $\infer{\Gamma\vdash\alpha\vee\beta}{\Gamma\vdash\beta}\quad\quad\infer{\Gamma\vdash\alpha\with\beta}{\Gamma\vdash\alpha\quad\quad\Gamma\vdash\beta}$

\item Правила удаления связок:\\$\infer{\Gamma\vdash\beta}{\Gamma\vdash\alpha\quad\Gamma\vdash\alpha\rightarrow\beta}\quad\quad\infer{\Gamma\vdash\gamma}{\Gamma\vdash\alpha\rightarrow\gamma\quad\Gamma\vdash\beta\rightarrow\gamma\quad\Gamma\vdash\alpha\vee\beta}$
 $\infer{\Gamma\vdash\alpha}{\Gamma\vdash\alpha\with\beta}\quad\quad\infer{\Gamma\vdash\beta}{\Gamma\vdash\alpha\with\beta}\quad\quad\infer{\Gamma\vdash\alpha}{\Gamma\vdash\bot}$
\item Пример доказательства:\vspace{-0.3cm}
$$\infer[(\text{введ}\with)]{A\with B\vdash B \with A}{\infer[(\text{удал}\with)]{A \with B \vdash B}{\infer[(\text{акс.})]{A \with B\vdash A \with B}{}}
                                           \quad\quad\infer[(\text{удал}\with)]{A \with B \vdash A}{\infer[(\text{акс.})]{A \with B\vdash A \with B}{}}}$$
\end{itemize}

Из важного - гипотезы включены в формулу, вместо $\neg$ нульместный $\bot$.\\
Классический нормальный вывод получится при замене принципа взрыва на снятие двойного отрицания: $$\infer{\Gamma\vdash A}{\Gamma\vdash (A\rightarrow\bot)\rightarrow\bot}.$$
Нормальный и гильбертов вывод эквивалентны в любой логике. КИВ строго сильнее ИИВ(например, в ИИВ нельзя доказать исключенное третье).

Можно построить аксиоматику ИИВ в гильбертовском стиле, заменив аксиому  $\neg \neg \alpha \rightarrow \alpha$ на $\alpha \rightarrow \neg\alpha \rightarrow \beta$.\\

\textbf{Интуиционистское исчисление высказываний}

Основные положения:
\begin{enumerate}
\item Математика не формальна.
\item Математика независима от окружающего мира.
\item Математика не зависит от логики — это логика зависит от математики.
\end{enumerate}

В интуиционизме \textbf{доказательства чистого существования} не конструктивны, поскольку не позволяют построить сам объект. Применяется \textbf{ВНК-интерпретация}:
\begin{itemize}
\item $\alpha\ \&\ \beta$ построено, если построены $\alpha$ и $\beta$ \item $\alpha \vee \beta$ построено, если построено $\alpha$ или $\beta$,
и мы знаем, что именно \item $\alpha\rightarrow\beta$ построено, если есть способ перестроения
$\alpha$ в $\beta$\item $\bot$ — конструкция, не имеющая построения\item $\neg\alpha$ построено, если построено $\alpha\rightarrow\bot$
\end{itemize}

Возьмём за $\alpha$ нерешённую проблему, тогда конструкция $\alpha\vee\neg\alpha$ не имеет построения.\\

Множество нижних граней $X\subseteq\mathcal{U}$: $\mbox{\upshape lwb}_\mathcal{U} X = \{ y\in \mathcal{U}\ |\ y \preceq x$\text{ при всех }$ x \in X\}$.\\
Минимальный ($m \in X$): нет меньшего & при всех $y \in X$, $y \preceq m$ влечёт $y = m$ \\
Наименьший ($m \in X$): меньше всех & при всех $y \in X$ выполнено $m \preceq y$\\
Инфимум: наибольшая нижняя грань & $\inf_\mathcal{U} X = \mbox{\upshape наиб}(\mbox{\upshape lwb}_\mathcal{U} X)$\\

\textbf{Решёткой} называется упорядоченная пара: $\langle X, (\preceq)\rangle$, 
где $X$ --- некоторое множество, а $(\preceq)$ --- частичный порядок на $X$, такой, 
что для любых $a,b \in X$ определены $a + b = \sup\{a,b\}$ и $a \cdot b = \inf\{a,b\}$.

\textbf{Псевдодополнением} $a \rightarrow b$ называется наибольший из $\{ x \ |\ a \cdot x \preceq b\}$.

\begin{center}\tikz{
  \node[circle,inner sep=0.3] (A) at (1,-1.3) {$a$};
  \node[fill=cyan!0.2,circle,inner sep=0.3] (B) at (0.5,-0.5) {$b$};
  \node[circle,inner sep=0.3] (C) at (1.5,-0.5) {$c$};
  \node[circle,inner sep=0.3] (D) at (1,0.3) {$d$};
  \foreach \b/\e in {A/B, A/C, B/D, C/D} {
      \draw[stealth-, line width=1, color=black!50!green] (\b) to (\e);
  }
  \draw[fill=cyan,fill=cyan,opacity=0.2](C.north west) 
     to[closed,curve through={
       (A.south west) .. (A.south east)
     }] (C.north east);
}\end{center}
Здесь $b \rightarrow c = \text{наиб}\{x\ |\ b \cdot x \preceq c\} = \text{наиб}\{ a, c \} = c$.\\

\textbf{Дистрибутивной решёткой} называется такая, что для любых $a,b,c$ выполнено
$a \cdot (b + c) = a \cdot b + a \cdot c$.

Импликативная решётка --- такая, в которой для любых элементов есть псевдодополнение.(Любая импликативная - дистрибутивная.

0 --- наименьший элемент решётки, а 1 --- наибольший элемент решётки.\\

Импликативная решётка с 0 --- \textbf{псевдобулева алгебра} (алгебра Гейтинга).
В такой решётке определено $\sim a := a \rightarrow 0$.

\textbf{Булева алгебра} --- псевдобулева алгебра, в которой $a\ + \sim a = 1$ для всех $a$.\\


Символы булевой алгебры: $(\with),(\vee),(\neg),$\text{F},\text{T}.\\
Символы решёток: $(+),(\cdot),(\rightarrow),(\sim),0,1$\\
Упорядочивание: $\text{F} \le \text{T}$.

\begin{enumerate}
\item $a \with b = \min(a,b)$, $a \vee b = \max(a,b)$ 
(анализ таблицы истинности), отсюда $a \cdot b = a \with b$ и $a + b = a \vee b$.

\item $a \rightarrow b = \neg a \vee b$, так как:
$$a \rightarrow b = \text{max}\{ c | c \with a \le b\} = \left\{\begin{array}{ll}\neg a,& b = \text{F}\\
                                                 \text{T},& b = \text{T}\end{array}\right.$$

\item $0 = \min\{\text{T},\text{F}\} = \text{F}$, $1 = \max\{\text{T},\text{F}\} = \text{T}$, $\sim a = a \rightarrow 0 = \neg a \vee \text{F} = \neg a$.
Заметим, что $a\ + \sim a = a \vee \neg a = \text{T}$.
\end{enumerate}
Итого: булева алгебра --- импликативная решётка с 0 и с $a\ + \sim a = 1$.\\

Пусть некоторое исчисление высказываний оценивается значениями из некоторой решётки.
Назовём оценку согласованной с исчислением, если 
$\llbracket\alpha\with\beta\rrbracket = \llbracket\alpha\rrbracket\cdot\llbracket\beta\rrbracket$,
$\llbracket\alpha\vee\beta\rrbracket = \llbracket\alpha\rrbracket+\llbracket\beta\rrbracket$,
$\llbracket\alpha\rightarrow\beta\rrbracket = \llbracket\alpha\rrbracket\rightarrow\llbracket\beta\rrbracket$,
$\llbracket\neg\alpha\rrbracket =\ \sim\llbracket\alpha\rrbracket$,
$\llbracket A \with\neg A\rrbracket = 0$, $\llbracket A\rightarrow A \rrbracket = 1$. \\

Любая псевдобулева алгебра, являющаяся согласованной оценкой интуиционистского исчисления высказываний,
является его корректной моделью: если $\vdash\alpha$, то $\llbracket\alpha\rrbracket = 1$.\\

Любая булева алгебра, являющаяся согласованной оценкой классического исчисления высказываний, 
является его корректной моделью: если $\vdash\alpha$, то $\llbracket\alpha\rrbracket = 1$.

\section{Алгебра Линденбаума. Полнота интуиционистского исчисления высказываний в псевдобулевых 
алгебрах. Модели Крипке. Вынужденность. Сведение моделей Крипке к псевдобулевым алгебрам. 
Нетабличность ИИВ (формулировка теоремы).}

Определим предпорядок на высказываниях: $\alpha \preceq \beta := \alpha \vdash \beta$ в интуиционистском исчислении высказываний.
Также $\alpha\approx\beta$, если $\alpha\preceq\beta$ и $\beta\preceq\alpha$.

Пусть $L$ --- множество всех высказываний. Тогда \textbf{алгебра Линденбаума} $\mathcal{L} = L/_\approx$.\\

(Теорема) Интуиционистское исчисление высказываний \textbf{полно в псевдобулевых алгебрах}:
если $\models\alpha$ во всех псевдобулевых алгебрах, то $\vdash\alpha$. 

\begin{proof}Возьмём в качестве модели исчисления алгебру Линденбаума: 
$\llbracket \alpha \rrbracket = [\alpha]_\mathcal{L}$. 

Пусть $\models\alpha$. Тогда $\llbracket\alpha\rrbracket = 1$ во всех псевдобулевых алгебрах, в том числе
и $\llbracket\alpha\rrbracket = 1_\mathcal{L}$. То есть $[\alpha]_\mathcal{L} = [A\rightarrow A]_\mathcal{L}$.
То есть $A \rightarrow A \approx \alpha$. Значит, в частности, $A \rightarrow A \vdash \alpha$. 
Значит, $\vdash\alpha$.\end{proof}

\textbf{Модель Крипке} $\langle \mathcal{W}, (\preceq), (\Vdash)\rangle$:
\begin{itemize}
\item $\mathcal{W}$ --- множество миров, $(\preceq)$ --- нестрогий частичный порядок на $\mathcal{W}$;
\item $(\Vdash)\subseteq \mathcal{W}\times P$ --- отношение вынуждения
между мирами и переменными(завуалированная "истинность"), причём, если $W_i \preceq W_j$ и $W_i \Vdash X$, то $W_j \Vdash X$.
\end{itemize}

Доопределим вынужденность:
\begin{itemize}
\item $W \Vdash \alpha\with\beta$, если $W \Vdash \alpha$ и $W \Vdash \beta$;
\item $W \Vdash \alpha\vee\beta$, если $W \Vdash \alpha$ или $W \Vdash \beta$;
\item $W \Vdash \alpha\rightarrow\beta$, если всегда при $W \preceq W_1$ и $W_1 \Vdash \alpha$ выполнено $W_1 \Vdash \beta$
\item $W \Vdash \neg\alpha$, если всегда при $W \preceq W_1$ выполнено $W_1 \not\Vdash \alpha$.
\end{itemize}

Будем говорить, что $\Vdash\alpha$, если $W\Vdash\alpha$ при всех $W \in \mathcal{W}$.
Будем говорить, что $\models_\kappa\alpha$, если $\Vdash\alpha$ во всех моделях Крипке.

(\textbf{Сведение к псевдобулевым алгебрам})Пусть $\langle \mathcal{W}, (\preceq), (\Vdash)\rangle$ ---
некоторая модель Крипке.
Тогда она есть корректная модель интуиционистского исчисления высказываний.

\begin{proof}
Доказательство для древовидного $(\preceq)$, обобщение на произвольный порядок легко построить.

Заметим, что $V(\alpha) := \{ w \in \mathcal{W}\ |\ w\Vdash\alpha\}$ открыто в топологии для деревьев(истинно в узле, значит истинно в потомке).
Значит, положив $V = \{\ S\ |\ S \subseteq \mathcal{W}\ \with\ S $\text{ --- открыто }$\}$ и
$\llbracket \alpha \rrbracket = V(\alpha)$, получим алгебру Гейтинга(все открыто, операции переходят).
\end{proof}

\textbf{Табличная модель}

Пусть задано $V$, значение $T \in V$ (<<истина>>), функция $f_P: P \rightarrow V$, 
функции $f_\with, f_\vee, f_\rightarrow : V \times V \rightarrow V$,
функция $f_\neg: V \rightarrow V$.

Тогда оценка $\llbracket X \rrbracket = f_P(X)$, 
$\llbracket \alpha\star\beta \rrbracket = f_\star(\llbracket \alpha \rrbracket, \llbracket \beta \rrbracket)$,
$\llbracket \neg\alpha \rrbracket = f_\neg(\llbracket\alpha\rrbracket)$ --- табличная.

Если $\vdash \alpha$ влечёт $\llbracket\alpha\rrbracket = T$ при всех оценках пропозициональных переменных $f_P$, 
то $\mathcal{M} := \langle V, T, f_\with, f_\vee, f_\rightarrow, f_\neg\rangle$ --- табличная модель.

\textbf{(Лемма)}Табличная модель конечна, если $V$ конечно.

\textbf{(Нетабличность ИИВ)}Не существует полной конечной табличной модели для интуиционистского исчисления высказываний.

\begin{proof}
Пусть существует полная конечная табличная модель $\mathcal{M}$, $V = \{v_1, v_2, \dots, v_n\}$(по лемме).
То есть, если $\models_\mathcal{M}\alpha$, то $\vdash\alpha$.

Рассмотрим $$\alpha_n = 
            \bigvee_{1 \le p < q \le n+1} A_p \rightarrow A_q
           $$
Рассмотрим оценку $f_P: \{A_1 \dots A_{n+1}\} \rightarrow \{v_1 \dots v_n\}$.
По принципу Дирихле существуют $p \ne q$, что $\llbracket A_p \rrbracket = \llbracket A_q \rrbracket$.
С другой стороны, $\vdash X \rightarrow X$ --- поэтому $f_\rightarrow(\llbracket X \rrbracket, \llbracket X \rrbracket) = T$,
значит, $$\llbracket A_p \rightarrow A_q \rrbracket = f_\rightarrow(v,v) = f_\rightarrow (\llbracket A_p \rrbracket, \llbracket A_q \rrbracket) = f_\rightarrow(\llbracket X \rrbracket, \llbracket X \rrbracket) = T$$

Аналогично, $\vdash \sigma \vee (X \rightarrow X) \vee \tau$, отсюда $\llbracket \alpha_n \rrbracket = \llbracket \sigma \vee (X \rightarrow X) \vee \tau \rrbracket = T$.

Однако, в такой модели $\not\Vdash \alpha_n$:

\begin{center}\tikz{
\node at (0,0)   (R) {$W_R$};
\node at (3,1.5) (A1) {$W_1$}; \node[right] at (3.5,1.5) (A11) {\color{black!50!red} $\Vdash A_1$};
  \draw[red,fill=red,opacity=0.2](A1.south west) 
     to[closed,curve through={($(A1.south west)!0.5!(A1.south east)$) .. (A1.north east)}] (A1.north west);

\node at (3,0.5) (A2) {$W_2$}; \node[right] at (3.5,0.5) (A21) {\color{black!50!magenta} $\Vdash A_2$};
\draw[red,fill=magenta,opacity=0.2](A2.north west) 
     to[closed,curve through={(A2.south west) .. (A2.south east)}] (A2.north east);
\node at (3,-0.2) (A3) {$\dots$};
\node at (3,-0.9) (A4) {$W_n$}; \node[right] at (3.5,-0.9) (A41) {\color{teal} $\Vdash A_n$};
\draw[red,fill=teal,opacity=0.2](A4.north west) 
     to[closed,curve through={($(A4.north west)!0.5!(A4.north east)$) .. (A4.south east)}] (A4.south west);

\draw[->] (R) to (A1); 
\draw[->] (R) to (A2); 
\draw[->] (R) to (A4); 

\node[right] at (6,1.5) {Если $q > 1$, то}; \node[right] at (8.6, 1.5) {$W_1 \not\Vdash A_q$ и $W_1 \not\Vdash A_1 \rightarrow A_q$};
\node[right] at (6,0.5) {Если $q > 2$, то}; \node[right] at (8.6, 0.5) { $W_2 \not\Vdash A_q$ и $W_2 \not\Vdash A_2 \rightarrow A_q$};
\node[right] at (6,-0.5) {Если $q > n$, то}; \node[right] at (8.6,-0.5) {$W_n \not\Vdash A_{n+1}$; $W_n \not\Vdash A_n \rightarrow A_{n+1}$};
\node[right] at (6,-1.5) {Если $p < q$, то}; \node[right] at (8.6, -1.5) { $W_p \not\Vdash A_q$ и $W_p \not\Vdash A_p \rightarrow A_q$};
}
\end{center}

Если $p < q$, то $W_p \not\Vdash A_p \rightarrow A_q$, то есть $W_R \not\Vdash A_p \rightarrow A_q$(недоказуема в детях, значит недоказуема в потомках).

Отсюда: $W_R \not\Vdash \bigvee_{p < q} A_p \rightarrow A_q$, $W_R \not\Vdash \alpha_n$,
 потому $\not\models \alpha_n$ и $\not\vdash \alpha_n$.
 
\end{proof}

\section{Гёделева алгебра. Операция $\Gamma(A)$. Дизъюнктивность ИИВ.
Подрешётка. Разрешимость интуиционистского исчисления высказываний.}

Исчисление \textbf{дизъюнктивно}, если при любых $\alpha$ и $\beta$ из $\vdash\alpha\vee\beta$ следует $\vdash\alpha$ или $\vdash\beta$.

Решётка \textbf{гёделева}, если $a + b = 1$ влечёт $a = 1$ или $b = 1$.

Для алгебры Гейтинга $\mathcal{A} = \langle A, (\preceq) \rangle$ определим операцию <<гёделевизации>>: 
$\Gamma(\mathcal{A}) = \langle A\cup\{\omega\}, (\preceq_{\Gamma(\mathcal{A})}) \rangle$, где
отношение $(\preceq_{\Gamma(\mathcal{A})})$ --- минимальное отношение порядка,
удовлетворяющее условиям:

\vspace{-0.5cm}
\begin{center}\begin{tabular}{cc}
\begin{minipage}{9cm}
\begin{itemize}
\item $a \preceq_{\Gamma(\mathcal{A})} b$, если $a \preceq_\mathcal{A} b$ и $a,b \notin \{\omega,1\}$;
\item $a \preceq_{\Gamma(\mathcal{A})} \omega$, если $a \ne 1$;
\item $\omega \preceq_{\Gamma(\mathcal{A})} 1$
\end{itemize}
\end{minipage}
&
\begin{minipage}{4cm}\begin{center}
\tikz{
    \filldraw[pattern=north west lines,pattern color=gray] (1,-1) circle (1cm);
    \node[right] at (2.2,-1) (A) {$A \setminus \{1\}$};
    \node[circle,fill,inner sep=2pt, outer sep=0pt,label=right:$1$] at (1,1) (Max) {};
    \node[circle,fill,inner sep=2pt, outer sep=0pt,label=above right:$\omega$] at (1,0) (Omega) {}; 
    \draw[-stealth,line width=1] (Max) to (Omega);
}\end{center}
\end{minipage}
\end{tabular}\end{center}

(Оценка)Определим $\llbracket\cdot\rrbracket_{\Gamma(\mathcal{L})} : \mathcal{F} \rightarrow \Gamma(\mathcal{L})$.
Положим $\llbracket X \rrbracket_{\Gamma(\mathcal{L})} := \llbracket X \rrbracket_\mathcal{L}$.
Связки определим естественным образом:
$\llbracket \alpha\with\beta \rrbracket_{\Gamma(\mathcal{L})} := \llbracket \alpha\rrbracket_{\Gamma(\mathcal{L})}\cdot\llbracket\beta \rrbracket_{\Gamma(\mathcal{L})}$
и т.п.


\textbf{(Теорема) }Оценка является алгеброй Гейтинга, согласованной с ИИВ.

Пусть $\mathcal{A}, \mathcal{B}$ --- алгебры Гейтинга. Тогда $g: \mathcal{A} \rightarrow \mathcal{B}$ --- гомоморфизм,
если $g(a \star b) = g(a) \star g(b)$, $g(0_\mathcal{A}) = 0_\mathcal{B}$ и $g(1_\mathcal{A}) = 1_\mathcal{B}$.

Будем говорить, что оценка $\llbracket\cdot\rrbracket_\mathcal{A}$ согласована
с $\llbracket\cdot\rrbracket_\mathcal{B}$ и гомоморфизмом $g$, если $g(\mathcal{A}) = \mathcal{B}$ и
$g(\llbracket\alpha\rrbracket_\mathcal{A}) = \llbracket\alpha\rrbracket_\mathcal{B}$.

$$\mathcal{G}(a) = \left\{\begin{array}{ll} a, & a \ne \omega\\
                                  1, & a = \omega\end{array}\right.$$

(Лемма)$\mathcal{G}$ --- гомоморфизм $\Gamma(\mathcal{L})$ и $\mathcal{L}$, причём 
оценка $\llbracket\cdot\rrbracket_{\Gamma(\mathcal{L})}$ согласована с $\mathcal{G}$
и $\llbracket\cdot\rrbracket_\mathcal{L}$.

\textbf{Дизъюнктивность ИИВ} Если $\vdash \alpha\vee\beta$, то либо $\vdash\alpha$, либо $\vdash\beta$.

\begin{proof}Пусть $\vdash\alpha\vee\beta$. Тогда $\llbracket\alpha\vee\beta\rrbracket_{\Gamma(\mathcal{L})} = 1$
  (так как данная оценка согласована с ИИВ). Тогда $\llbracket\alpha\rrbracket_{\Gamma(\mathcal{L})} = 1$ или
  $\llbracket\beta\rrbracket_{\Gamma(\mathcal{L})} = 1$ (так как $\Gamma(\mathcal{L})$ гёделева). 
  
  Пусть $\llbracket\alpha\rrbracket_{\Gamma(\mathcal{L})} = 1$, 
  тогда $\mathcal{G}(\llbracket\alpha\rrbracket_{\Gamma(\mathcal{L})}) = \llbracket\alpha\rrbracket_\mathcal{L} = 1$, 
  тогда $\vdash\alpha$ (по полноте $\mathcal{L}$).
  \end{proof}

(в геделевой алгебре легко показывается дизъюнктивность, а из-за согласованности все переносится на псевдобулеву алгебру).

Решётка $\mathcal{L'} = \langle L', \preceq \rangle$ --- \textbf{подрешётка решётки} $\mathcal{L} = \langle L, \preceq \rangle$, 
если $L' \subseteq L$, $(\preceq') \subseteq (\preceq)$ и 
при $a,b \in L'$ выполнено $a +_{\mathcal{L'}} b = a +_{\mathcal{L}} b$ и $a \cdot_{\mathcal{L'}} b = a \cdot_{\mathcal{L}} b$.

(Лемма) Существует дистрибутивная подрешётка $\mathcal{L'}$, содержащая
$a_1, \dots, a_n$, что $|L'| \le 2^{2^n}$.

\begin{proof}
  Пусть $\mathcal{L'} = \langle\{ \varphi(a_1,\dots,a_n)\ |\ \varphi \text{ составлено из (+) и }(\cdot)\}, (\preceq)\rangle$.
  Заметим, что если $p,q \in L'$, то $p \star_{\mathcal{L}} q \in L'$ 
  (так как $\varphi_p(\overrightarrow{a})\star\varphi_q(\overrightarrow{a}) = \psi(\overrightarrow{a})$). Также ясно,
  что если $\sup_L\{p,q\} \in L'$ (или $\inf_L\{p,q\} \in L'$), то $p \star_{\mathcal{L}} q = p \star_{\mathcal{L'}} q$.
  Значит, $\mathcal{L'}$ также дистрибутивна. Построим <<ДНФ>>:
  $$\varphi(a_1,\dots,a_n) = \sum_{\text{Кн} \in \text{ДНФ}(\varphi)}\prod_{i \in \text{Кн}}a_i$$
  
  
  Всего не больше $2^n$ возможных компонент и $2^{2^n}$ возможных формул $\varphi(\overrightarrow{a})$.
  \end{proof}

  Язык $\mathcal{L} \subseteq A^*$ разрешим, если существует алгоритм, который
  завершает работу при любом $a \in A^*$, поданном на вход,
  причём алгоритм возвращает <<истину>> при $a \in \mathcal{L}$, 
  и возвращает <<ложь>> при $a \notin \mathcal{L}$.

  Теория \textbf{неразрешима}, если язык всех истинных (доказуемых) формул неразрешим.

  (Лемма) Если $\not\vdash \alpha$ в ИИВ, то существует $\mathcal{G}$,
  что $\mathcal{G} \not\models \alpha$, причём $|\mathcal{G}| \le 2^{2^{|\alpha|+2}}$.

  \begin{proof}Если $\not\vdash \alpha$, то по 
    полноте найдётся алгебра Гейтинга $\mathcal{H}$, что
    $\mathcal{H} \not\models \alpha$. 
    
    Пусть $\varphi_1, \dots, \varphi_n$ --- подформулы $\alpha$.
    Пусть $\mathcal{G}$ --- дистрибутивная подрешётка $\mathcal{H}$, 
    построенная по $\llbracket \varphi_1 \rrbracket, \dots, \llbracket \varphi_n \rrbracket$, $0$ и $1$. 
    
    Очевидно, что $\mathcal{G}$ --- алгебра Гейтинга, и можно показать, 
    что $\mathcal{G} \not\models \alpha$ (псевдодополнения не обязаны сохраниться).
    Тогда по лемме, $|\mathcal{G}| \le 2^{2^{n+2}}$. 
    \end{proof}

    \textbf{(Теорема)}ИИВ разрешимо.

    \begin{proof}По формуле $\alpha$ построим все возможные алгебры Гейтинга $\mathcal{G}$ размера не больше $2^{2^{|\alpha|+2}}$,
      если $\mathcal{G}\models\alpha$, то $\vdash\alpha$.
      \end{proof}

\section{Категорические силлогизмы. Термины, предикат, субъект, фигуры, модусы (сильные, слабые, неправильные),
ограничения, контрпримеры на ограничения.
Исчисление предикатов. Язык исчисления предикатов.
Метаязык, сокращения записи.
Вхождения, свободные вхождения, подстановка, свобода для подстановки.
Теория доказательств для исчисления предикатов, выводимость.
Доказательства свойств категорических силлогизмов (формулировка свойств сильных и слабых
силлогизмов на языке исчисления предикатов, их доказательство).}

\textbf{Силлогизм} --- «подытоживание, подсчёт, умозаключение»

 \textbf{Категорический} --- потому, что речь идёт о категориях (в философском смысле).

Определяем некоторые стандартные мыслительные блоки, с которыми у образованной аудитории есть навык работы.

Цель --- сделать неформальный человеческий язык чуть более формальным. 

\newpage

\textbf{Термины} 

\begin{tabular}{l}
  предикат (больший термин, P)\\
  субъект (меньший термин, S)\\
  средний термин (M). 
  \end{tabular}
  
  \vspace{0.3cm}
  На основании соотношений P и M, а также M и S строим соотношение P и S.
  \vspace{0.3cm}
  Возможные соотношения:
  
  \begin{tabular}{lll}
  A & Affirmato (общеутвердительное) & Матан есть раздел математики (SaP)\\
  I & affIrmato (частноутвердительное) & Некоторые разделы математики сложны (SiP)\\
  E & nEgo (общеотрицательное) & Никакой человек не знает всю математику\\
  O & negO (частноотрицательное) & Некоторые разделы математики --- не матан
  \end{tabular}

  \begin{tabular}{lcccc}
    & Фигура 1 & Фигура 2 & Фигура 3 & Фигура 4\\
    & 
    \tikz{
        \node at (0,1) (M1) { \tiny $M$ };
        \node at (1,1) (P)  { \tiny $P$ };
        \node at (0,0) (S)  { \tiny $S$ };
        \node at (1,0) (M2) { \tiny $M$ };
        \draw (0.85,0.85) -- (0.15,0.85) -- (0.85,0.15) -- (0.15,0.15);
    }
    &
    \tikz{
        \node at (1,1) (M1) { \tiny $M$ };
        \node at (0,1) (P)  { \tiny $P$ };
        \node at (0,0) (S)  { \tiny $S$ };
        \node at (1,0) (M2) { \tiny $M$ };
        \draw (0.15,0.85) -- (0.85,0.85) -- (0.85,0.15) -- (0.15,0.15);
    }
    &
    \tikz{
        \node at (0,1) (M1) { \tiny $M$ };
        \node at (1,1) (P)  { \tiny $P$ };
        \node at (1,0) (S)  { \tiny $S$ };
        \node at (0,0) (M2) { \tiny $M$ };
        \draw (0.85,0.85) -- (0.15,0.85) -- (0.15,0.15) -- (0.85,0.15);
    }
    &
    \tikz{
        \node at (1,1) (M1) { \tiny $M$ };
        \node at (0,1) (P)  { \tiny $P$ };
        \node at (1,0) (S)  { \tiny $S$ };
        \node at (0,0) (M2) { \tiny $M$ };
        \draw (0.15,0.85) -- (0.85,0.85) -- (0.15,0.15) -- (0.85,0.15);
    }
    
    \\
    
    
    Большая посылка: & M—P & P—M & M—P & P—M\\
    Меньшая посылка: & S—M & S—M & M—S & M—S\\
    Заключение: & S—P & S—P & S—P &S—P 
    \end{tabular}

Большинство модусов \textbf{неправильные}.

Список всех правильных модусов (из них выделяют \textbf{слабые}, выводящие частное соотношение при возможности общего --- указаны курсивом):

{\small
\begin{center}\begin{tabular}{llll}
Фигура 1 &Фигура 2 &Фигура 3 &Фигура 4\\
Barbara &Cesare &Darapti &Bramantip\\
Celarent &Camestres &Disamis &Camenes\\
Darii &Festino &Datisi &Dimaris\\
Ferio &Baroco &Felapton &Fesapo\\
\it Barbari &\it Cesaro &Bocardo &Fresison\\
\it Celaront &\it Camestros &Ferison &\it Camenos
\end{tabular}\end{center}}

Некоторые модусы требуют непустоты M: это все слабые модусы и 19 \textbf{сильных}.(Пример с единорогами - \textbf{ограничение}).

\textbf{Исчисление предикатов} создано, чтобы расширить формализованную часть языка(включает в себя предикаты).

\textbf{Язык исчисления предикатов}

  
  Предметные выражения: метапеременная {\color{blue}$\theta$}. \pause
  \begin{itemize}
  \item Предметные переменные: {\color{blue}$a$}, {\color{blue}$b$}, {\color{blue}$c$}, \dots, метапеременные {\color{blue}$x$}, {\color{blue}$y$}. \pause
  \item Функциональные выражения: {\color{blue}$f(\theta_1,\dots,\theta_n)$}, метапеременные {\color{blue}$f$}, {\color{blue}$g$}, \dots(константы - нульместные функции).\pause
  \end{itemize}\pause
  Логические выражения: метапеременные {\color{blue}$\alpha$}, {\color{blue}$\beta$}, {\color{blue}$\gamma$}, \dots
  \begin{itemize}
  \item Предикатные выражения: {\color{blue}$P(\theta_1,\dots,\theta_n)$}, метапеременная {\color{blue}$P$}.\\\pause
  Имена: {\color{blue}$A$}, {\color{blue}$B$}, {\color{blue}$C$}, \dots (пропозициональные перемнные - нульместные предикаты).\pause
  \item Связки: {\color{blue}$(\varphi\vee\psi)$}, {\color{blue}$(\varphi\with\psi)$}, {\color{blue}$(\varphi\rightarrow\psi)$}, 
     {\color{blue}$(\neg\varphi)$}.\pause
  \item Кванторы: {\color{blue}$(\forall x.\varphi)$} и {\color{blue}$(\exists x.\varphi)$}(жадные).
  \end{itemize}


\textbf{Вхождение} подформулы в формулу --- это позиция первого символа этой подформулы в формуле.


\vspace{-0.4cm}
$$\text{Вхождения }{\color{blue}x}\text{ в формулу:}\quad (\forall {\color{blue}x}.A({\color{blue}x}) \vee \exists {\color{blue}x}.B({\color{blue}x})) \vee C({\color{blue}x})$$
\vspace{-0.7cm}

\textbf{Вхождение $x$ в $\psi$ свободное}, если не находится в области действия никакого квантора по $x$.
Переменная входит свободно в $\psi$, если имеет хотя бы одно свободное вхождение. $FV(\psi), FV(\Gamma)$ --- множества свободных
переменных в $\psi$, в $\Gamma$


Терм $\theta$ \textbf{свободен для подстановки} вместо $x$ в $\psi$ ($\psi[x := \theta]$), если 
ни одно свободное вхождение переменных в $\theta$ не станет связанным после подстановки.

\begin{center}\begin{tabular}{c|c}
Свобода есть & Свободы нет\\\hline
$(\forall x.P(y)) [y := z]$ & $(\forall x.P(y)) [y := x]$\\
$(\forall y.\forall x.P(x)) [x := y]$ & $(\forall y.\forall x.P(t)) [t := y]$
\end{tabular}\end{center}

\textbf{Теория доказательств}

Рассмотрим язык исчисления предикатов. Возьмём все схемы аксиом классического исчисления высказываний и добавим ещё две схемы аксиом 
(здесь везде $\theta$ свободен для подстановки вместо $x$ в $\varphi$):

\begin{tabular}{ll}
11. & $(\forall x.\varphi) \rightarrow \varphi[x:=\theta]$\\
12. & $\varphi[x:=\theta] \rightarrow \exists x.\varphi$ 
\end{tabular}

Добавим ещё два правила вывода (здесь везде $x$ не входит свободно в $\varphi$):
$$\infer[\text{Правило для }\forall]{\varphi\rightarrow\forall x.\psi}{\varphi\rightarrow\psi}$$
$$\infer[\text{Правило для }\exists]{(\exists x.\psi)\rightarrow\varphi}{\psi\rightarrow\varphi}$$

\textbf{Доказуемость, выводимость, полнота, корректность} --- аналогично исчислению высказываний.

\textbf{Доказательства свойств категорических силлогизмов }

%TODO
\section{Теория моделей исчисления предикатов (предметное множество, оценка).
Функции (предикаты) и функциональные (предикатные) символы.
Общезначимость, следование.
Теорема о дедукции в исчислении предикатов. Отличия от исчисления высказываний.
Лемма о перестановке подстановки и оценки. Теорема о корректности исчисления предикатов.}

\textbf{Предметные выражения}: метапеременная {\color{blue}$\theta$}. \pause
  \begin{itemize}
  \item Предметные переменные: {\color{blue}$a$}, {\color{blue}$b$}, {\color{blue}$c$}, \dots, метапеременные {\color{blue}$x$}, {\color{blue}$y$}. \pause
  \item Функциональные выражения: {\color{blue}$f(\theta_1,\dots,\theta_n)$}, метапеременные {\color{blue}$f$}, {\color{blue}$g$}, \dots(константы - нульместные функции).\pause
  \end{itemize}\pause
  \textbf{Логические выражения}: метапеременные {\color{blue}$\alpha$}, {\color{blue}$\beta$}, {\color{blue}$\gamma$}, \dots
  \begin{itemize}
  \item Предикатные выражения: {\color{blue}$P(\theta_1,\dots,\theta_n)$}, метапеременная {\color{blue}$P$}.\\\pause
  Имена: {\color{blue}$A$}, {\color{blue}$B$}, {\color{blue}$C$}, \dots (пропозициональные перемнные - нульместные предикаты).\pause
  \item Связки: {\color{blue}$(\varphi\vee\psi)$}, {\color{blue}$(\varphi\with\psi)$}, {\color{blue}$(\varphi\rightarrow\psi)$}, 
     {\color{blue}$(\neg\varphi)$}.\pause
  \item Кванторы: {\color{blue}$(\forall x.\varphi)$} и {\color{blue}$(\exists x.\varphi)$}(жадные).
  \end{itemize}

  \textbf{Оценка} --- упорядоченная четвёрка $\langle D, F, P, E \rangle$, где:\pause

\begin{enumerate}
\item $D \neq \varnothing$ --- предметное множество;\pause
\item $F$ --- оценка для функциональных символов; пусть $f_n$ --- $n$-местный функциональный символ:
 $$F_{f_n}: D^n \rightarrow D$$\pause

\item $P$ --- оценка для предикатных символов; пусть $T_n$ --- $n$-местный предикатный символ:
 $$P_{T_n}: D^n \rightarrow V\quad\quad\quad\pause V = \{\text{И}, \text{Л}\}$$\pause

\item $E$ --- оценка для предметных переменных.
 $$E(x) \in D$$
\end{enumerate}\end{dfn}

\textbf{Правила оценок}

\begin{itemize}
  \item Правила для связок $\vee$, $\with$, $\neg$, $\rightarrow$ остаются прежние;\pause
  \item $\llbracket f_n (\theta_1, \theta_2, \dots, \theta_n) \rrbracket = F_{f_n} (\llbracket\theta_1\rrbracket,
            \llbracket\theta_2\rrbracket, \dots, \llbracket\theta_n\rrbracket)$\pause
  \item $\llbracket P_n (\theta_1, \theta_2, \dots, \theta_n) \rrbracket = P_{T_n} (\llbracket\theta_1\rrbracket,
            \llbracket\theta_2\rrbracket, \dots, \llbracket\theta_n\rrbracket)$\pause
  \item $$\llbracket \forall x.\phi \rrbracket = \left\{\begin{array}{ll}
     \text{И}, & \text{если } \llbracket\phi\rrbracket^{x := t} = \text{И}\text{ при всех } t \in D\\
     \text{Л}, & \text{если найдётся } t \in D, \text{ что } \llbracket\phi\rrbracket^{x := t} = \text{Л}
    \end{array}\right.$$\pause
  \item $$\llbracket \exists x.\phi \rrbracket = \left\{\begin{array}{ll}
     \text{И}, & \text{если найдётся } t \in D, \text{ что } \llbracket\phi\rrbracket^{x := t} = \text{И}\\
     \text{Л}, & \text{если } \llbracket\phi\rrbracket^{x := t} = \text{Л}\text{ при всех } t \in D
    \end{array}\right.$$
  \end{itemize}

  Формула исчисления предикатов общезначима, если истинна при любой оценке:
$$\models\phi$$

\textbf{Следование}: $\gamma_1,\gamma_2,\dots,\gamma_n\models\alpha$, если $\alpha$ выполнено всегда, когда выполнено $\gamma_1,\gamma_2,\dots,\gamma_n$.\\

\textbf{Теорема о дедукции} 
Если $\Gamma\vdash\alpha\rightarrow\beta$, то $\Gamma,\alpha\vdash\beta$.
Если $\Gamma,\alpha\vdash\beta$ и в доказательстве не применяются правила для кванторов 
по свободным переменным из $\alpha$, то $\Gamma\vdash\alpha\rightarrow\beta$.

\begin{proof}$(1)$ --- как в КИВ \pause $(2)$ --- та же схема, два новых случая. \pause

  Перестроим: $\delta_1, \delta_2, \dots, \delta_n \equiv \beta$ в $\alpha\rightarrow\delta_1, \alpha\rightarrow\delta_2, \dots, \alpha\rightarrow\delta_n$.
  
  Дополним: обоснуем $\alpha\rightarrow\delta_n$, если предыдущие уже обоснованы.\pause
  
  Два новых похожих случая: правила для $\forall$ и $\exists$. Рассмотрим $\forall$.
  
  Доказываем $(n)\ \ \alpha\rightarrow\psi\rightarrow\forall x.\varphi$ (правило для $\forall$), значит, доказано  
  $(k)\ \ \alpha\rightarrow\psi\rightarrow\varphi$. \pause\\

  \begin{tabular}{lll}
  $(n-0.9) \dots (n-0.8)$ & $(\alpha\rightarrow\psi\rightarrow\varphi)\rightarrow(\alpha\with\psi)\rightarrow\varphi$ & Т. о полноте КИВ\\
  $(n-0.6)$ & $(\alpha\with\psi)\rightarrow\varphi$ & M.P. $k$,$n-0.8$\\\pause
  $(n-0.4)$ & $(\alpha\with\psi)\rightarrow\forall x.\varphi$ & Правило для $\forall$, $n-0.6$\\\pause
  $(n-0.3) \dots (n-0.2)$ & $((\alpha\with\psi)\rightarrow\forall x.\varphi)\rightarrow(\alpha\rightarrow\psi\rightarrow\forall x.\varphi)$ & Т. о полноте КИВ\\
  $(n)$ & $\alpha\rightarrow\psi\rightarrow\forall x.\varphi$ & M.P. $n-0.4$, $n-0.2$
  \end{tabular}
  
  \end{proof}

  (\textbf{Лемма о перестановке подстановки и оценки})
  Если $\theta$ свободен для подстановки 
вместо $x$ в $\varphi$, то $\llbracket\varphi\rrbracket^{x := \llbracket\theta\rrbracket} = \llbracket\varphi[x := \theta]\rrbracket$.

\begin{proof}[Доказательство (индукция по структуре $\varphi$)](заметено под ковер все, кроме кванторов, потому что оно очевидно)
  \begin{itemize}
  \item База: $\varphi$ не имеет кванторов. Очевидно.
  \item Переход: пусть справедливо для $\psi$. Покажем для $\varphi = \forall y.\psi$. 
  \begin{itemize}
  \item $x=y$ либо $x \notin FV(\psi)$. Тогда: 
  $\llbracket\forall y.\psi\rrbracket^{x := \llbracket\theta\rrbracket} = \llbracket\forall y.\psi\rrbracket = \llbracket(\forall y.\psi)[x := \theta]\rrbracket$(оценка игнорируется)
  
  \item $x \ne y$. Тогда: $\llbracket\forall y.\psi\rrbracket^{x := \llbracket\theta\rrbracket} = 
    \llbracket\psi\rrbracket^{y \in D; x := \llbracket\theta\rrbracket} = \dots$(если квантор оценивается в истинну, то в можем взять любую оценку, если в ложь, то возьмем этот контрпример).
  %\vspace{-0.3cm}
  
  {\color{olive}Свобода для подстановки: $y\notin\theta$.}
  \vspace{-0.3cm}
   $$\dots = \llbracket\psi\rrbracket^{x := \llbracket\theta\rrbracket; y \in D} = \dots$$
  \vspace{-0.8cm}
  
  {\color{olive}Индукционное предположение.}
  \vspace{-0.3cm}
  
   $$\dots = \llbracket\psi[x := \theta]\rrbracket^{y \in D} = 
  \llbracket\forall y.(\psi[x := \theta])\rrbracket = \dots$$
  \vspace{-0.5cm}

  (если оценка совпадает на всех y по предположению, то совпадет и на кванторе).
  
  {\color{olive}Но $\forall y.(\psi[x := \theta]) \equiv (\forall y.\psi) [x := \theta]$ (как текст). Отсюда:}
  \vspace{-0.3cm}
  
  $$\dots = \llbracket(\forall y.\psi)[ x := \theta]\rrbracket$$
  \vspace{-0.5cm}
  \end{itemize}
  \end{itemize} \end{proof}

  \textbf{Корректность} Если $\Gamma \vdash \alpha$ и в доказательстве не используются кванторы по свободным переменным из $FV(\Gamma)$, то $\Gamma \models \alpha$.

  \begin{proof}Фиксируем $D, F, P$. Индукция по длине доказательства $\alpha$: при любом $E$ выполнено $\Gamma\models\alpha$ 
    при длине доказательства $n$, покажем для $n+1$. 
    \begin{itemize}
    \item Схемы аксиом (1)..(10), правило M.P.: аналогично И.В.
    \item Схемы (11) и (12), например, схема $(\forall x.\varphi) \rightarrow \varphi [x := \theta]$: \vspace{-0.6cm}
    
    $$\llbracket (\forall x.\varphi) \rightarrow \varphi [x := \theta]\rrbracket = \llbracket ((\forall x.\varphi) \rightarrow \varphi) [x := \theta] \rrbracket =
      \llbracket (\forall x.\varphi) \rightarrow \varphi \rrbracket ^ { x := \llbracket\theta\rrbracket } = \text{И}$$(по лемме)
    
    \item Правила для кванторов: например, введение $\forall$:
    
      Пусть $\llbracket \psi \rightarrow \varphi \rrbracket = \text{И}$. Причём $x \notin FV(\Gamma)$ и $x \notin FV(\psi)$. То есть,
      при любом $\mathcal{x}$ выполнено $\llbracket \psi \rightarrow \varphi \rrbracket^{x := \mathcal{x}} = \text{И}$. Тогда
      $\llbracket \psi \rightarrow (\forall x.\varphi) \rrbracket = \text{И}$.(по таблице истинности)
    
    \end{itemize}
    \end{proof}
%\section{Непротиворечивые множества формул (с кванторами и бескванторные).
%Пополнение множества формул.
%Существование моделей у непротиворечивых множеств формул в бескванторном исчислении предикатов.}

%\section{Поверхностные кванторы (предварённая форма). Эквивалентность формул формулам с поверхностными кванторами
%(формулировка теоремы).
%Сколемизация.
%Теорема Гёделя о полноте исчисления предикатов. 
%Полнота исчисления предикатов.}

%\section{Машина Тьюринга. Разрешимость теории, примеры. Задача об останове, её неразрешимость. 
%Неразрешимость исчисления предикатов.}

%\section{ Представление чисел через натуральные (целые, рациональные, вещественные).
%Аксиоматика Пеано. Арифметические операции (сложение, умножение, возведение в степень) в аксиоматике Пеано.
%Доказательство коммутативности сложения.
%Порядок теории (0, 1, 2). Теории первого порядка. Формальная арифметика. Доказательство $a=a$.
%Арифметизация математики, формализация категорических силлогизмов, предложенная Лейбницем.}

%\section{Примитивно-рекурсивные и рекурсивные функции. 
%Функции вычисления простых чисел. Частичный логарифм.
%Выразимость отношений и представимость функций в формальной арифметике. Характеристические функции.
%Функция Аккермана. Доказательство невозможности выражения функции Аккермана в примитивно-рекурсивных функциях.}

%\section{Представимость примитивов $N$, $Z$, $U$, $S$. Бета-функция Гёделя. 
%Представимость $R$ и $M$, представимость рекурсивных функций в формальной арифметике.}

%\section{Гёделева нумерация. Рекурсивность представимых в формальной арифметике функций.
%Функции $W_1$ и $W_2$, их представления $\omega_1$, $\omega_2$, самоприменимость.}

%\section{ Непротиворечивость (эквивалентные определения), $\omega$-не\-про\-ти\-во\-ре\-чи\-вость. 
%Первая теорема Гёделя о неполноте арифметики.
%Формулировка первой теоремы Гёделя о неполноте арифметики в форме Россера. 
%Синтаксическая и семантическая неполнота арифметики.
%Неполнота расширений формальной арифметики.
%Ослабленные варианты: арифметика Пресбургера, система Робинсона.}

%\section{ Вторая теорема Гёделя о неполноте арифметики, $Consis$. 
%Лемма об автоссылках. Условия Гильберта-Бернайса-Лёба. Неразрешимость формальной арифметики. 
%Теорема Тарского о невыразимости истины.}

%%\section{ Теория множеств. Определения равенства. Парадокс брадобрея. Аксиоматика Цермело-Френкеля. 
%Конструктивные аксиомы (пустого, пары, объединения, множества подмножеств, выделения).
%Частичный, линейный, полный порядок. Ординальные числа, аксиома бесконечности.
%Конечные ординалы, предельные ординалы, доказательство существования ординала $\omega$, 
%операции над ординалами (варианты определения), факты об операциях над ординалами 
%(выполнены ли ассоциативность и коммутативность операций). Связь ординалов и упорядочений.}

%\section{Аксиомы фундирования и подстановки. Кардинальные числа, мощность множеств, операции над 
%кардинальными числами (сложение, умножение, возведение в степень). Теорема Кантора-Бернштейна, 
%теорема Кантора. }


\section{Мощность модели. Элементарные подмодели. Теорема Лёвенгейма-Сколема, парадокс Сколема.}


\begin{dfn} Пусть задана модель $\langle D, F_n, P_n \rangle$ для некоторой теории первого порядка. 
Её мощностью будем считать мощность $D$.
\end{dfn}


\begin{dfn}$\mathcal{M}' = \langle D', F'_n, P'_n \rangle$ --- элементарная подмодель $\mathcal{M} = \langle D, F_n, P_n \rangle$, 
если: \begin{enumerate}
\item $D' \subseteq D$,  $F'_n$, $P'_n$ --- сужение $F_n$, $P_n$ (замкнутое на $D'$). \item $\mathcal{M}\models \varphi(x_1,\dots,x_n)$ тогда и только тогда, когда $\mathcal{M}'\models \varphi(x_1,\dots,x_n)$
при $x_i \in D'$. \end{enumerate}
\end{dfn}

\begin{exm}Когда сужение $M$ не является элементарной подмоделью? 
$\forall x.\exists y.x \ne y$. Истинно в $\mathbb{N}$.  Но пусть $D' = \{ 0 \}$.
\end{exm}


\begin{thm}Пусть $T$ --- множество всех формул теории первого порядка. 
Пусть теория имеет некоторую модель $\mathcal{M}$.
Тогда найдётся элементарная подмодель $\mathcal{M'}$, причём $|\mathcal{M'}| \leq \max(\aleph_0, |T|)$.
\end{thm}
\begin{proof} (Схема доказательства)
\begin{enumerate} 
\item Построим $D_0$ --- множество всех значений, которые упомянуты в языке теории. \item Будем последовательно пополнять $D_i$: $D_0 \subseteq D_1 \subseteq D_2 \dots$, следя за мощностью.
$D' = \cup D_i$.
\item Покажем, что $\langle D', F_n, P_n\rangle$ --- требуемая подмодель.
\end{enumerate}
\end{proof}


Пусть $\{f^0_k\}$ --- все 0-местные функциональные символы теории. \begin{enumerate}
\item $D_0 = \{ \llbracket f^0_k \rrbracket \}$, если есть хотя бы один $f^0_k$. \item Если таких $f^0_k$ нет, возьмём какое-нибудь одно значение из $D$. \end{enumerate}
Очевидно, $|D_0| \le |T|$.



Фиксируем некоторый $D_k$. Напомним, $T$ --- множество всех формул теории. Рассмотрим $\varphi \in T$.\begin{enumerate}
\item $\varphi$ не имеет свободных переменных --- пропустим. \item $\varphi$ имеет хотя бы одну свободную переменную $y$. \begin{enumerate}
\item $\varphi (y, x_1, \dots, x_n)$ при $y,x_i \in D_k$ бывает истинным и ложным --- ничего не меняем \item $\varphi (y, x_1, \dots, x_n)$ при $y \in D$ и $x_i \in D_k$ либо всегда истинен, либо всегда ложен --- ничего не меняем \item $\varphi (y, x_1, \dots, x_n)$ при $y,x_i \in D_k$ тождественно истинен или ложен, но при 
$y' \in D \setminus D_k$ отличается --- добавим $y'$ к $D_{k+1}$. Вместе добавим всевозможные $\llbracket\theta(y')\rrbracket$.
\end{enumerate}
\end{enumerate}


\begin{enumerate}
\item Всего добавили не больше $|T| \cdot |T|$ (для каждой формулы $\varphi$, возможно, будет добавлен $y$ --- 
и всевозможные выражения $\theta(y)$, допустимые в языке), и $|D_0| \le |T| \le |T|\cdot|T|$,
отсюда $|D_k| \le |T| \cdot |T|$.
\item $|D'| = |\bigcup D_i| \le |T| \cdot |T| \cdot \aleph_0$.
\item Тогда $|T| \cdot |T| \cdot \aleph_0 = \max(|T|,\aleph_0)$. Разберём случаи:

\begin{enumerate}
\item Если $|T| < \aleph_0$, тогда $(|T| \cdot |T|) \cdot \aleph_0 = \aleph_0$
\item Если $|T| \ge \aleph_0$, тогда $(|T| \cdot |T|) \cdot \aleph_0 = |T| \cdot \aleph_0 = |T|$.

\end{enumerate}
\item Итого, $|D'| \le \max(|T|,\aleph_0)$.
\end{enumerate}



Индукцией по структуре формул $\tau \in T$ покажем, 
что все формулы можно вычислить, и что $\llbracket \varphi \rrbracket_\mathcal{M'} = \llbracket \varphi \rrbracket_\mathcal{M}$.
\begin{enumerate}
\item База, 0 связок. $\tau \equiv P(f_1(x_1,\dots,x_n),\dots,f_n(x_1,\dots,x_n))$.  Если $x_i \in D'$, то значит,
добавлены на некоторых шагах (максимальный пусть $t$). Поэтому в $D_{t+1}$ можно вычислить формулу, и её значение сохранилось. \item Переход. Пусть формулы из $k$ связок сохраняют значения. Рассмотрим $\tau$ с $k+1$ связкой. \begin{enumerate}
\item $\tau \equiv \rho \star \sigma$ --- очевидно. \item $\tau\equiv\forall y.\varphi(y,x_1,\dots,x_n)$. Каждый $x_i$ добавлен на каком-то шаге --- максимум $t$. Если $\varphi(y,x_1,\dots,x_n)$ бывает истинен и ложен при $y_t, y_f \in D$, то $y_t, y_f \in D_{t+1}$ (по построению). Поэтому, если $\mathcal{M}\not\models\forall y.\varphi(y,x_1,\dots,x_n)$, то и 
$\mathcal{M'}\not\models\forall y.\varphi(y,x_1,\dots,x_n)$. Если же $\varphi(y,x_1,\dots,x_n)$ не меняется от $y$, то тем более
$\llbracket \varphi \rrbracket_\mathcal{M'} = \llbracket \varphi \rrbracket_\mathcal{M}$. \item $\tau\equiv\exists y.\varphi(y,x_1,\dots,x_n)$ --- аналогично.
\end{enumerate}
\end{enumerate}



\begin{enumerate}
\item Как известно, $|\mathbb{R}| = |\mathcal{P}(\mathbb{N})| > |\mathbb{N}| = \aleph_0$.  Однако, ZFC --- теория со счётным
количеством формул. Значит, существует счётная модель ZFC, то есть $|\mathbb{R}| = \aleph_0$.  В чём ошибка? \item У равенств разный смысл, первое --- в предметном языке, второе --- в метаязыке. 
\end{enumerate}

\section{Аксиома выбора, альтернативные формулировки (лемма Цорна, теорема Цермело, существование
частичной обратной), доказательство переходов (кроме доказательства леммы Цорна).}

\begin{axm}[выбора]
Из любого семейства дизъюнктных непустых множеств $\mathcal{A}$ можно выбрать непустую трансверсаль --- 
множество $S$, что $|S \cap A| = 1$ для каждого $A\in\mathcal{A}$. Иначе, $S \in \times \mathcal{A}$.
\end{axm}

\begin{thm}[функциональный вариант аксиомы выбора]
Пусть $\mathcal{A}$ --- семейство непустых множеств. Тогда существует
$f : \mathcal{A} \rightarrow \cup \mathcal{A}$, причём $\forall a.a \in \mathcal{A} \rightarrow f(a) \in a$
\end{thm}

\begin{proof}
Пусть $X(A) = \{ \langle A, a \rangle \ |\ a \in A \}$, 
по семейству $\mathcal{A}$ рассмотрим $\{X(A)\ |\ A\in\mathcal{A}\}$
\begin{itemize}
\item непустых: если $A\in\mathcal{A}$, $A \ne \varnothing$, то $X(A) \ne \varnothing$;
\item дизъюнктное: если $A_0,A_1\in\mathcal{A}$, $A_0 \ne A_1$, то $X(A_0) \cap X(A_1) = \varnothing$
\end{itemize}
тогда по аксиоме выбора $\exists f.f \in \times \mathcal{A}$.
\end{proof}
Обратное утверждение также легко показать.

\subsection{Аксиома выбора: альтернативные формулировки}

\begin{thm}[Лемма Цорна]
Если задано $\langle M, (\preceq) \rangle$ и для всякого линейно упорядоченного $S \subseteq M$ выполнено
$\text{upb}_M S \ne \varnothing$, то в $M$ существует максимальный элемент.
\end{thm}
\begin{thm}[Теорема Цермело]
На любом множестве можно задать полный порядок.
\end{thm}
\begin{thm}
У любой сюръективной функции существует частичная обратная.
\end{thm}

\begin{thm}
Аксиома выбора $\Rightarrow$ лемма Цорна: без доказательства
\end{thm}

\subsection{Начальный отрезок}

\begin{dfn}
Назовём (для данного раздела) упорядоченным множеством пару $\langle S, (\prec_S)\rangle$.
Отношение порядка $(\prec_S)$ может быть как строгим, так и нестрогим.
Будем говорить, что $\langle S, (\prec_S)\rangle$ --- начальный отрезок $\langle T, (\prec_T) \rangle$,
если:\begin{itemize}
\item $S \subseteq T$;
\item если $a,b \in S$, то $a \prec_S b$ тогда и только тогда, когда $a \prec_T b$;
\item если $a \in S$, $b \in T\setminus S$, то $a \prec_T b$.
\end{itemize}
Будем обозначать это как $\langle S, (\prec_S)\rangle\sqsubseteq\langle T, (\prec_T)\rangle$ или как $S \sqsubseteq T$, если порядок на множествах понятен из контекста.
\end{dfn}

\begin{thm}
Отношение <<быть начальным отрезком>> является отношением нестрогого порядка.
\end{thm}

\subsection{Верхняя грань семейства упорядоченных множеств}
\begin{thm}[о верхней грани]
Если семейство упорядоченных множеств $X$ линейно упорядочено отношением <<быть начальным отрезком>>, то у него есть верхняя грань.
\end{thm}

\begin{proof}
Пусть $M = \cup \{ T | \langle T, (\prec) \rangle \in X \}$ и
$(\prec)_M = \cup \{ (\prec) | \langle T, (\prec) \rangle \in X \}$.

Покажем, что если $\langle A, (\prec_A)\rangle \in X$, то $A \sqsubseteq M$. Рассмотрим определение:
\begin{itemize}
\item $A \subseteq M$ --- выполнено по построению $M$;
\item если $a,b \in A$, то $a \prec_A b$ влечёт $a \prec_M b$ (по построению $M$). Если же $a \prec_M b$, но $a \not\prec_A b$,
то существует $A'$, что $a,b \in A'$ и $a \prec_{A'} b$. Тогда $A\not\sqsubseteq A'$ и $A'\not\sqsubseteq A$, что невозможно
по линейности порядка;
\item если $a \in A$, $b \in M\setminus A$, то найдётся $B$, что $b\in B$, отчего $a \prec_B b$ (так как $A \sqsubseteq B$) 
и $a \prec_M b$ (по построению $M$).
\end{itemize}
Тогда $\langle M, (\prec_M)\rangle$ --- требуемая верхняя грань.
\end{proof}

\subsection{Лемма Цорна $\Rightarrow$ теорема Цермело}
Пусть выполнена лемма Цорна и дано некоторое $X$. Покажем, что на нём можно ввести полный порядок.
\begin{itemize}
\item Пусть $S = \{ \langle P, (\prec)\rangle \ |\ P \subseteq X, (\prec)\text{ --- полный порядок} \}$.
{\color{gray}Например, для $X = \{0,1\}$ множество
$S = \{
\langle\varnothing,\varnothing\rangle,
\langle \{0\},\varnothing\rangle,
\langle\{1\},\varnothing\rangle,
\langle X, 0 \prec 1\rangle,
\langle X, 1 \prec 0\rangle
\}$}

\item Введём порядок на $S$ как $(\sqsubseteq)$. Заметим, что это --- частичный, но не линейный порядок. 
{\color{gray}Например, $\langle X, 0 \prec 1\rangle$ несравним с $\langle X, 1 \prec 0\rangle$.}

\item По теореме о верхней грани любое линейно упорядоченное подмножество 
$\langle T, (\sqsubseteq) \rangle$ (где $T \subseteq S$) имеет
верхнюю грань.

{\color{gray}Например, 
для $\{\langle\varnothing,\varnothing\rangle,
\langle \{0\},\varnothing\rangle,
\langle X, 0 \prec 1\rangle\}$ это $\langle X, 0 \prec 1\rangle$.}

\item По лемме Цорна тогда есть $\langle R, (\sqsubseteq_R)\rangle = \max S$. Заметим, что $R = X$, потому что иначе пусть
$a \in X\setminus R$. Тогда положив $M = \langle R\cup\{a\}, (\sqsubseteq_R)\cup\{x\prec a\ |\ x \in R\} \rangle$
получим, что $M$ тоже вполне упорядоченное (и потому $M \in S$), значит, $R$ не максимальное.
\end{itemize}

\subsection{Теорема Цермело $\Rightarrow$ существование обратной $\Rightarrow$ аксиома выбора}
\begin{thm}Теорема Цермело $\Rightarrow$ у сюръективных функций существует частичная обратная.\end{thm}
\begin{proof}
Рассмотрим сюръективную $f: A \rightarrow B$. Рассмотрим семейство $R_b = \{ a \in A\ |\ f(a) = b \}$.
Построим полный порядок на каждом из $R_b$. Тогда $f^{-1}(b) = \min R_b$.
\end{proof}
\begin{thm}Существует частичная обратная у сюръективных функций $\Rightarrow$ существует трансверсаль у семейства непустых дизъюнктных множеств.\end{thm}
\begin{proof}
Пусть дано семейство непустых дизъюнктных множеств $\mathcal{A}$. 
Рассмотрим $f: \cup \mathcal{A} \rightarrow \mathcal{A}$, что
$f(a) = \cup\{ A \in \mathcal{A}\ |\ a \in A \}$. Поскольку элементы $\mathcal{A}$ дизъюнктны,
$f(a) \in \mathcal{A}$ при всех $a$. Тогда существует $f^{-1}: \mathcal{A} \rightarrow \cup\mathcal{A}$. Тогда 
$\{ f^{-1}(A)\ |\ A\in\mathcal{A} \} \in \times \mathcal{A}$.
\end{proof}


\section{Применение аксиомы выбора: эквивалентность определений пределов (по Коши и по Гейне).
Теорема Диаконеску. Ослабленные варианты (счётный выбор и зависимый выбор), универсум фон Неймана.
Аксиома конструктивности.}

\begin{dfn}Пределом функции $f$ в точке $x_0$ по \emph{Коши} называется такой $y$, что
$$\forall \varepsilon\in\mathbb{R}^+.\exists \delta\in\mathbb{R}^+.\forall x.|x-x_0| < \delta \rightarrow |f(x) - y| < \varepsilon$$
\end{dfn}

\vspace{-0.5cm}
\begin{dfn}Пределом функции $f$ в точке $x_0$ по \emph{Гейне} называется такой $y$, что
для любой $x_n \rightarrow x_0$ выполнено $f(x_n) \rightarrow y$.
\end{dfn}


\begin{thm}
Если $\lim\limits_{x \rightarrow x_0}f(x) = y$ по Гейне, то
$\forall \varepsilon>0.\exists \delta>0.\forall x.|x-x_0|<\delta \rightarrow |f(x)-y| < \varepsilon$.
\end{thm}

\begin{proof}
Пусть не так:
$\exists \varepsilon>0.\forall \delta>0.\exists x_\delta.|x_\delta-x_0|<\delta \with |f(x_\delta)-y| \ge \varepsilon$.
Фиксируем $\varepsilon$ и возьмём $\delta_n = \frac{1}{n}$ и $p_n = x_{\delta_n}$. 
$p_n \rightarrow x_0$, так как $|x_\frac{1}{n} - x_0| < \frac{1}{n}$, 
по определению предела по Гейне $f(p_n) \rightarrow y$, 
но по предположению $\forall n\in\mathbb{N}.|f(p_n) - y| \ge \varepsilon$.
\end{proof}
\begin{snote}
Для применения предела по Гейне нужна $p_n$: то есть $p: \mathbb{N}\rightarrow\mathbb{R}$.
%где $\Gamma(p) \subseteq \mathbb{N}\times\mathbb{R}$
%$\langle x_\frac{1}{1}: |x_\frac{1}{1}-x_0|<1 \with |f(x_\frac{1}{1})-y| \ge \varepsilon$; $x_\frac{1}{2}: |x_\frac{1}{2}-x_0|<\frac{1}{2} \with |f(x_\frac{1}{2})-y| \ge \varepsilon; ...\rangle$ 
%
%\vspace{0.3cm}
... %$\exists \varepsilon.\forall \delta.\exists x_\delta.|x_\delta-x_0|<\delta \with |f(x_\delta)-y| \ge \varepsilon$.\\
Фиксируем $\varepsilon$ и рассмотрим $X_\delta = \{ x \ |\ |x-x_0| <\delta \with |f(x)-y| \ge \varepsilon\}$.
Отрицание предела по Коши означает, что $X_\delta \ne \varnothing$ при любом $\delta > 0$.
%Возьмём $\delta_n = \frac{1}{n}$ и $x_{\frac{1}{n}} \in X_\frac{1}{n}$.

... То есть, по семейству $Q:=\{ X_1, X_\frac{1}{2}, X_\frac{1}{4}, \dots \}$ 
по аксиоме выбора построим $q: Q \rightarrow \cup Q$, что $q(X_\frac{1}{n}) \in X_\frac{1}{n}$.
Далее, взяв композицию $p_n := q(X_{\delta_n})$, получаем $p_n \rightarrow x_0$, что $\forall n\in\mathbb{N}.|f(p_n) - y| \ge \varepsilon$.
\end{snote}



\begin{thm}Пусть $\lim\limits_{x \rightarrow x_0} f(x) = y$ и дана $x_n \rightarrow x_0$.
Тогда $f(x_n) \rightarrow y$.\end{thm}

\begin{proof}
%Пусть $\lim_{x \rightarrow x_0} f(x) = y$ и дана $x_n \rightarrow x_0$. 
%То есть, $\forall \varepsilon>0.\exists \delta>0.\forall x.|x-x_0| < \delta \rightarrow |f(x) - y| < \varepsilon$
Фиксируем $\varepsilon > 0$.
\begin{itemize}
%\item Заметим, что $\forall $ и $|x_n - x_0| < \delta\rightarrow|f(x_n) - y| < \varepsilon$.
\item $\exists \delta > 0.\exists N\in\mathbb{N}.(\forall x.|x - x_0| < \delta \rightarrow |f(x) - y| < \varepsilon) \with
(\forall n\in\mathbb{N}.n > N \rightarrow |x_n - x_0|<\delta)$
\item $(\forall x.|x - x_0| < \delta \rightarrow |f(x) - y| < \varepsilon) \rightarrow (|x_n - x_0| < \delta \rightarrow |f(x_n) - y| < \varepsilon)$ \ (сх. 11).
\item $\exists \delta > 0.\exists N\in\mathbb{N}.\forall n\in\mathbb{N}.n > N\rightarrow |f(x_n) - y| < \varepsilon$.
\item Поскольку $\delta$ не используется в формуле, $\exists \delta > 0$ можно устранить.
\item Отсюда $\exists N\in\mathbb{N}.\forall n\in\mathbb{N}.n > N\rightarrow |f(x_n) - y| < \varepsilon$
\end{itemize}
\end{proof}
Почему здесь не потребовалась аксиома выбора? Потому что нам нужен единственный $\delta$, а для него --- 
единственный $N$


\begin{exm}
Пусть $A_0 = \{0,1,3,5\}$ и $A_1 = \{3,5,1,0,0,5,3\}$.
Верно ли, что $A_0 = A_1$?
Да, так как $\forall x.x \in \{0,1,3,5\} \leftrightarrow x \in \{3,5,1,0,0,5,3\}$.\end{exm}
\begin{thm}[конгруэнтность]
Если $f: A \rightarrow B$, также $a,b\in A$ и $a=b$, то $f(a) = f(b)$.
\end{thm}

\begin{proof}
Пусть $F \subseteq A\times B$ --- график функции $f$.

%Легко показать, что если $a=b$ и $y_1 = y_2$, то $\langle a, y_1\rangle = \langle b,y_2\rangle$.\\
%Значит (по аксиоме равенства), $\langle a,x\rangle \in F$ тогда и только тогда,
%когда $\langle b,x\rangle \in F$. 
По определению функции, $\forall x.\forall y_1.\forall y_2.\langle x,y_1\rangle \in F \with \langle x,y_2 \rangle \in F \rightarrow y_1 = y_2$.\\
Также, если $f(a) = y_1$, $f(b) = y_2$, то $\langle a,y_1 \rangle \in F$ и $\langle b,y_2 \rangle \in F$.\\
Тогда: $\langle a,y_1\rangle = \langle b,y_1\rangle = \langle b,y_2 \rangle = \langle a,y_2\rangle$,
то есть $f(a) = y_2 = f(b)$.

%Пусть $\langle a,x \rangle \in F$ (поскольку $f$ --- функция, такое $x$ должно существовать).
%Тогда из $a=b$ следует $\langle b,x \rangle = \langle a,x \rangle$ (по свойствам упорядоченной пары), значит, $f(b) = x$.
\end{proof}
% следует $f(A_0) = f(A_1)$ 
%по определению функционального бинарного отношения:
%$$\forall x.\exists y.F(x, y) \with \forall y_0.\forall y_1.F(x,y_0) \with F(x,y_1) \rightarrow y_0=y_1$$.
%\end{exm}


\begin{thm}Если рассмотреть ИИП с ZFC, то для любого $P$ выполнено $\vdash P \vee \neg P$.\end{thm}
\begin{proof}Рассмотрим $\mathcal{B} = \{0,1\}$, $A_0 = \{ x \in \mathcal{B} | x = 0 \vee P \}$ и 
$A_1 = \{ x \in \mathcal{B} | x = 1 \vee P\}$.
$\{A_0,A_1\}$ --- семейство непустых множеств, и по акс. выбора существует
$f: \{A_0,A_1\} \rightarrow \cup A_i$, что $f(A_i) \in A_i$. (Если $P$, то $A_0 = A_1$ и $\{A_0,A_1\} = \{\mathcal{B}\}$).

\vspace{0.3cm}
\begin{tabular}{ll}
$\vdash f(A_0) \in A_0 \with f(A_1) \in A_1$ & а.выбора: $f(A_i) \in A_i$\\
$\vdash {\color{olive}f(A_0) \in \mathcal{B}} \with (f(A_0) = 0 \vee P) \with {\color{olive}f(A_1) \in \mathcal{B}} \with (f(A_1) = 1 \vee P)$ & а.выделения\\
%$\vdash(f(A_0) = 0 \vee P) \with (f(A_1) = 1 \vee P)$ & Удал. $(\with)$\\
$\vdash (f(A_0) = 0 \with f(A_1) = 1) \vee P$ & Удал. $(\with)$ + дистр.\\
$\vdash P\vee{\color{blue}f(A_0) \ne f(A_1)}$ & $0 \ne 1$ и транз.\\$\vdash P \rightarrow A_0 = A_1$ & Определение $A_i$\\
$\vdash A_0 = A_1 \rightarrow f(A_0) = f(A_1)$ & Конгруэнтность\\
$\vdash \color{blue} f(A_0) \ne f(A_1) \rightarrow \neg P$ & Контрапозиция\\
$\vdash P \vee \neg P$ & Подставили
\end{tabular}
\end{proof}




\begin{thm}[конечного выбора]
Если $X_1\ne\varnothing, \dots, X_n\ne\varnothing$, $X_i\cap X_j = \varnothing$ при $i \ne j$, то $\times \{X_1, \dots, X_n\} \ne \varnothing$.
\end{thm}

\begin{proof}
\begin{itemize}\item База: $n=1$. Тогда $\exists x_1.x_1 \in X_1$, поэтому $\exists x_1.\{x_1\} \in \times \{X_1\}$.

\item Переход: %если $\exists v.v \in \times \{X_{1,n}\}$ и $\exists x_{n+1}.x_{n+1} \in X_{n+1}$, то
$\exists v.v \in \times \{X_{1,n}\}\rightarrow\exists x_{n+1}.x_{n+1} \in X_{n+1}\rightarrow
v \cup \{x_{n+1}\} \in \times (X_{1,n}\cup\{X_{n+1}\})$
\end{itemize}\vspace{-0.3cm}\end{proof}

%Построим явно: $(\exists x_1.x_1 \in X_1) \rightarrow \exists f.\exists x_1.f = \{\langle X_1, x_1 \rangle\}\with x_1 \in X_1$

%Построим явно: $$\exists x_1.\dots\exists x_n.x_1 \in X_1 \with \dots \with x_n \in X_1 \rightarrow \varphi(\langle X_1, x_1\rangle, \dots, \langle X_n, x_n\rangle)$$
%И потом:
%$$X_1 \ne \varnothing \with \dots \with X_n \ne \varnothing \rightarrow \exists f.\varphi(f)$$

%Докажем явным выписыванием: 
%$x_1 \in X_1 \with \dots \with x_n \in X_n \rightarrow \varphi(\{\langle X_1, x_1\rangle, \dots, \langle X_n, x_n\rangle\})$\\
%$\exists x_1 \in X_1 \with \dots \with (\exists x_n \in X_n)\rightarrow \exists f.\varphi(f)$
%$$(x_1 \in X_1) \with \dots \with (x_n \in X_n) \rightarrow (f = \{\langle X_1, x_1 \rangle, \dots, \langle X_n, x_n \rangle\} \rightarrow f(X_1) = x_1 \with \dots \with f(X_n) = x_n)$$
%$$(f = \{\langle X_1, x_1 \rangle, \dots, \langle X_n, x_n \rangle\} \rightarrow f(X_1) = x_1 \with \dots \with f(X_n) = x_n)$$
%$$(\exists x_1. x_1 \in X_1)\with\dots\with(\exists x_n.x_n \in X_n)\rightarrow\exists f.f(X_1) \in X_1 \with \dots \with f(X_n) \in X_n$$

\begin{axm}[счётного выбора]
Для счётного семейства непустых множеств существует функция, каждому из которых сопоставляющая один из своих элементов
\end{axm}

\begin{axm}[зависимого выбора]
если $\forall x \in E.\exists y \in E. x R y$, то существует последовательность $x_n: \forall n.x_n R x_{n+1}$
\end{axm}



Заметим, что семейство $\{A_0, A_1\}$ из теоремы Диаконеску в ИИП не является конечным (равно как и бесконечным).
\begin{dfn}Конечное множество --- равномощное некоторому конечному кардинальному числу.\end{dfn}

\begin{itemize}
\item Какова мощность семейства? 
\item 1, если $P$, и 2, если $\neg P$. 
\item Но поскольку $P \vee \neg P$ не выполнено в ИИП, мы не можем
доказать, что мощность семейства 1 или 2.
\item Поэтому мы не можем воспользоваться теоремой конечного выбора.
\end{itemize}


\begin{dfn}Наследственным свойством множества назовём такое свойство, которым обладает как само
множество, так и все его подмножества.
\end{dfn}

\begin{dfn}Фундированным множеством назовём такое, которое не пересекается хотя бы с одним своим элементом.\end{dfn}

\begin{dfn}Аксиома фундирования. 
В каждом непустом множестве найдется элемент, не пересекающийся с исходным множеством.
$$\forall x .x = \varnothing \vee \exists y .y \in x \with \forall z.z \in x \rightarrow z \notin y$$
\end{dfn}

Иными словами, в каждом множестве есть элемент, минимальный по отношению $(\in)$.


\begin{dfn}\emph{Универсум фон Неймана} $V$ --- все наследственные фундированные множества.\end{dfn}

При наличии аксиомы фундирования можно показать, что $V = \cup_a V_a$, где:
$$V_a = \left\{\begin{array}{ll}
    \varnothing, & a=0\\
    \mathcal{P}(V_b), & a = b'\\
    \bigcup_{b < a}(V_b), & a \text{ --- предельный}
\end{array}\right.$$

\begin{dfn}
\emph{Конструктивный универсум} $L = \cup_a L_a$, где:
$$L_a = \left\{\begin{array}{ll}
    \varnothing, & a=0\\
    \{ \{ x\in L_b\ |\ \varphi(x,t_1,\dots,t_k) \}\ |\ \varphi\text{ --- формула}, t_i \in L_b\}, & a = b'\\
    \bigcup_{b < a}(L_b), & a \text{ --- пред.}
\end{array}\right.$$
\end{dfn}



\begin{dfn}
Аксиома конструктивности: $V=L$, то есть допустимы только те фундированные множества, которые задаются формулами.
\end{dfn}

\begin{thm}Аксиома выбора и континуум-гипотеза следуют из аксиомы конструктивности\end{thm}

Для некоторых теорий аксиома слишком сильна.



\section{Индукция и полная индукция. Наследственные множества. Трансфинитная индукция
(аналоги полного и обычного варианта математической индукции). Доказательство $a \cdot a = a$ при $a \ge \aleph_0$.}

\subsection{Два вида индукции}

\begin{dfn}[принцип математической индукции]
Какое бы ни было $\varphi(x)$, если $\varphi(0)$ и при всех $x$ выполнено $\varphi(x)\rightarrow \varphi(x')$, то
при всех $x$ выполнено и само $\varphi(x)$.
\end{dfn}

\begin{dfn}[принцип полной математической индукции]
Какое бы ни было $\psi(x)$, если $\psi(0)$ и при всех $x$ выполнено $(\forall t.t \leq x \rightarrow \psi(t))\rightarrow \psi(x')$, то
при всех $x$ выполнено и само $\psi(x)$.
\end{dfn}

\begin{thm}Принципы математической индукции эквивалентны\end{thm}
\begin{proof}
$(\Rightarrow)$ взяв $\varphi := \psi$, имеем выполненность $\varphi(x)\rightarrow\varphi(x')$, значит, $\forall x.\psi(x)$. \\
$(\Leftarrow)$ возьмём $\psi(x) := \forall t.t\le x\rightarrow\varphi(t)$.
\end{proof}

\subsection{Наследственные подмножества}

\begin{dfn} Назовём вполне упорядоченное отношением $(\in)$ множество $S$ наследственным подмножеством $A$, если 
$\forall x.x \in A \rightarrow (\forall t.t \in x \rightarrow t \in S) \rightarrow x \in S$.
\end{dfn}
\begin{thm}Единственным наследственным подмножеством вполне упорядоченного множества является оно само.\end{thm}
\begin{proof}Пусть $B \subseteq A$ --- наследственное и $B \ne A$.
Тогда существует $a = \min (A \setminus B)$. Тогда $(\forall t.t \in a \rightarrow t \in B) \rightarrow a \in B$ по наследственности $B$,
и выполнено $\forall t.t \in a \rightarrow t \in B$ (по минимальности $a$). Значит, $a \in B$.
\end{proof}

\subsection{Трансфинитная индукция}

\begin{thm}[ограниченная трансфинитная индукция] Если для $\varphi(x)$ (некоторого утверждения
теории множеств) и некоторого ординала $\varepsilon$ (ограничения) выполнено
$\forall x.x \in \varepsilon \rightarrow (\forall t.t \in x \rightarrow \varphi(t)) \rightarrow \varphi(x)$,
то $\forall x.x \in \varepsilon \rightarrow \varphi(x)$.
\end{thm}
\begin{proof}Рассмотрим $S = \{ x\in \varepsilon\ |\ \varphi(x) \}$. Тогда $x \in S$ равносильно 
$x\in\varepsilon\with\varphi(x)$.
Тогда перепишем: $\forall e.e \in \varepsilon \rightarrow (\forall x.x \in e \rightarrow x \in S) \rightarrow e \in S$.
Отсюда по теореме о наследственных множествах $S = \varepsilon$.\end{proof}

\begin{thm}[неограниченная трансфинитная индукция] Если для $\varphi(x)$ (некоторого утверждения
теории множеств) выполнено
$\forall x.\text{ординал}(x) \rightarrow (\forall t.t \in x \rightarrow \varphi(t)) \rightarrow \varphi(x)$,
то $\forall x.\text{ординал}(x) \rightarrow \varphi(x)$.
\end{thm}

\subsection{Альтернативная формулировка трансфинитной индукции}

\begin{thm}Для ординала $\varepsilon$ подмножество $S \in \varepsilon$ --- наследственное, если и только если одновременно:
\begin{enumerate}
    \item Если $x \in \varepsilon$ и $x = \varnothing$, то $x \in S$;
    \item Если $x \in \varepsilon$ и существует $y$: $y' = x$, то $y \in S \rightarrow x \in S$;
    \item Если $x \in \varepsilon$ и $x$ --- предельный, то $(\forall t.t \in x \rightarrow t \in S) \rightarrow (x \in S)$.
\end{enumerate}
\end{thm}

\begin{proof}
$(\Rightarrow)$ очевидно. Докажем $(\Leftarrow)$: пусть $S$ не наследственное: 
$E := \{e \in \varepsilon \ |\  (\forall t.t \in e \rightarrow t \in S) \with e \notin S \}$
и $E \ne \varnothing$. Тогда пусть $e = \min E$.

\begin{enumerate}
\item $e = \varnothing$ или предельный. Тогда $(\forall t.t \in e \rightarrow t \in S) \rightarrow (e \in S)$.
\item $e = y'$. Тогда $y \in \varepsilon$ ($\varepsilon$ --- ординал) и 
$(\forall t.t \in y \rightarrow t \in S) \rightarrow (y \in S)$ (так как $e$ минимальный, для которого $S$ не наследственное).
По условию, $(y \in S) \rightarrow (e \in S)$, отсюда $(\forall t.t \in e \rightarrow t \in S) \rightarrow (e \in S)$.

\begin{center}\tikz{\draw[thick,-stealth] (0,0) -- (7,0); 
\filldraw (2,0) circle (1pt);
\filldraw (1,0) circle (1pt);
\filldraw (3,0) node[above] {$t \in y$} circle (2pt);
\filldraw[red] (4,0) node[above] {$y$} circle (2pt) ;
\filldraw (5,0) node[above] {$e\vphantom{y}$} circle (2pt) ; }\end{center}
\end{enumerate}
\end{proof}

\subsection{Пример применения: $\alpha\cdot\alpha = \alpha$ при $\alpha \ge \aleph_0$}

\begin{thm}Если $\alpha$ --- кардинальное число и $\alpha \ge \aleph_0$, 
то $\alpha\cdot\alpha = \alpha$.\end{thm}
\begin{proof}Формализуем свойство <<быть кардинальным числом>>:
$|x|=x$ и утверждение теоремы: $\varphi(x) := |x|=x\rightarrow x<\omega\vee|x\times x|=x$.

Транфинитная индукция: при $\forall y.y < x \rightarrow \varphi(y)$ покажем $\varphi(x)$,
разберём варианты:

\begin{enumerate}
\item $|x|\ne x$ или $|x| < \omega$, тогда $\varphi(x)$ истинно (из лжи следует любое утверждение).
\item $|x|=x$ и $|x| = \omega$, тогда надо показать $\omega < \omega \vee |\omega\times\omega|=\omega$
(утверждение можно показать без индукции, рассмотрим отдельно).
\item $|x|=x$ и $|x| > \omega$, тогда надо показать $x < \omega \vee |x\times x|=x$ (рассмотрим отдельно).
\end{enumerate}
\end{proof}

\subsubsection{Счётный случай: $|\omega \times \omega| = \omega$}
Тогда $\omega \times \omega$ упорядочим так: $\langle p,q \rangle \prec \langle s,t \rangle$,
если \begin{enumerate}
\item $\max(p,q) < \max(s,t)$
\item $\max(p,q) = \max(s,t)$ и $q < t$
\item $\max(p,q) = \max(s,t)$, $q = t$ и $p < s$
\end{enumerate}
Очевидно, можно построить биекцию между так упорядоченными значениями и $\omega$.

\begin{center}\begin{tikzpicture}

\filldraw[gray!20] (0,0) -- (4.5, 0) -- (4.5, 3) -- (0, 3) -- cycle;
\filldraw[gray!50] (0,0) -- (3, 0) -- (3, 2) -- (0, 2) -- cycle;
\filldraw[gray] (0,0) -- (1.5, 0) -- (1.5, 1) -- (0, 1) -- cycle;

\foreach \x in {0, 1, 2, 3} {
	\foreach \y in {0, 1, 2, 3} {
                \node at (1.5*\x + 1, \y + 0.6) {
			\pgfmathparse{(max(\x,\y))*(max(\x,\y)) + \y + ((\y+1)==(max(\x,\y)+1))*\x}%
			\pgfmathprintnumber{\pgfmathresult}%
		};
		\node at (1.5*\x + 0.4, \y + 0.2) {\footnotesize $\langle \x,\y \rangle$};
		\draw (1.5*\x, \y) -- (1.5*\x +1.5, \y) -- (1.5*\x +1.5, \y +1) -- (1.5*\x, \y +1) -- cycle;
	}
}

\end{tikzpicture}\end{center}

\subsubsection{Общий случай: $\alpha$ --- кардинал, $\alpha > \omega$ и $|\alpha \times \alpha| = \alpha$}
Аналогично счётному случаю, $\alpha \times \alpha$ упорядочим так: $\langle p,q \rangle \prec \langle s,t \rangle$,
если \begin{enumerate}
\item $p \cup q < s \cup t$
\item $p \cup q = s \cup t$ и $q < t$
\item $p \cup q = s \cup t$, $q = t$ и $p < s$
\end{enumerate}
\begin{itemize}
\item Легко заметить, что это --- линейный порядок (показав, что $p \not\prec q$ и $q \not\prec p$ влечёт $p = q$)
\item ... и полный порядок. Найти наименьший в $S \ne \varnothing$ возможно, рассмотрев $m_1 := \min \{ p \cup q\ |\ \langle p,q \rangle \in S\}$ и
$M_1 := \{ \langle p,q\rangle\ |\ \langle p,q \rangle \in S, p \cup q = m_1\}$,
затем $m_2 := \min \{q\ |\ \langle p,q \rangle \in M_1 \}$,
$M_2 := \{\langle p,q\rangle\ |\ \langle p,q \rangle \in M_1, q = m_1\}$.
Тогда требуемым наименьшим в $S$ будет $\min \{ p\ |\ \langle p,q \rangle \in M_2\}$
\item Тогда $\langle \alpha\times\alpha, (\prec)\rangle$ соответствует какой-то ординал $\tau$ 
и сохраняющая порядок биекция $t: \tau\rightarrow\alpha\times\alpha$. 
\item Заметим, что $x < \omega$ тогда и только тогда, когда $\cup(\cup t(x)) < \omega$
(очевидно из того, что $|\{z\ |\ \text{ординал}(z), z < x\}|=|\{p\ |\ p \prec t(x)\}|$).
\item Покажем, что $|\tau| = \alpha$.
\end{itemize}

\subsection{Докажем $\tau = \alpha$}

Очевидно, что $\tau \ge \alpha$ (так как $|\tau| = |\alpha\times\alpha| \ge \alpha$). Но пусть $\tau > \alpha$.
\begin{itemize}
\item Тогда $t(\alpha) = \langle\zeta,\eta\rangle$ определено (у $\alpha$ есть образ).
\item Пусть $\sigma := \zeta \cup \eta$. Очевидно, $\langle \zeta, \eta \rangle \preceq \langle \sigma,\sigma \rangle$
и $\sigma \in \alpha$.
\item Каков образ $t$ на этом начальном отрезке?
$\{t(x)\ |\ x < \alpha\} \subseteq \{\langle p,q\rangle\ |\ p,q \le \sigma\}$.
Поэтому $\alpha \le |(\sigma+1)\times(\sigma+1)|$. 
\item С другой стороны, $\sigma < \alpha$. Поскольку $\alpha$ --- кардинал (т.е., в частности, предельный ординал), 
то $\sigma+1 < \alpha$ и $|\sigma+1| < \alpha$. 
\item По предположению индукции, $|\sigma+1|<\omega \vee |\sigma+1| = |\sigma+1|\cdot|\sigma+1|$,
по свойствам $(\prec)$ имеем $\sigma\ge\omega$.
\item Отсюда $\alpha \le |(\sigma+1)\times(\sigma+1)| = |\sigma+1| < \alpha$, что невозможно.
\end{itemize}



\section{Система $S_\infty$, степень и порядок доказательства. 
Правило сечения, теорема об устранении сечений. Доказательство непротиворечивости формальной арифметики.}


\begin{enumerate}
\item Язык: связки $\neg$, $\vee$, $\forall$, $=$; нелогические символы: $(+)$,$(\cdot)$,$(')$,$0$; переменные: $x$.
\item Аксиомы: все истинные формулы вида $\theta_1=\theta_2$ $(0 = 0, 5 = 5)$; все истинные отрицания формул вида $\neg\theta_1=\theta_2 $$(\neg 1 = 0, \neg 2 \cdot 2 = 5)$
($\theta_i$ --- термы без переменных).
\item Структурные (слабые) правила:
$$\infer{\zeta\vee\beta\vee\alpha\vee\delta}{\zeta\vee\alpha\vee\beta\vee\delta} \quad\quad
\infer{\alpha\vee\delta}{\alpha\vee\alpha\vee\delta}$$

сильные правила
$$\infer{\alpha\vee\beta}{\beta}\quad
\infer{\neg(\alpha\vee\beta)\vee\delta}{\neg\alpha\vee\delta\quad\neg\beta\vee\delta}\quad
\infer{\neg\neg\alpha\vee\delta}{\alpha\vee\delta}\quad
\infer{(\neg\forall x.\alpha)\vee\delta}{\neg\alpha[x := \theta]\vee\delta}\quad$$

Формулы в правилах, обозначенные буквами $\zeta$ и $\delta$, называются боковыми и могут отсутствовать.
\item и ещё два правила 
\end{enumerate}



Бесконечная индукция:
$$\infer{(\forall x.\alpha)\vee\delta}{\alpha[x:=\overline{0}]\vee\delta
                                  \quad\alpha[x:=\overline{1}]\vee\delta
                                  \quad\alpha[x:=\overline{2}]\vee\delta\quad\dots}$$

Сечение:
$$\infer{\zeta\vee\delta}{\zeta\vee\alpha\quad\quad\neg\alpha\vee\delta} $$ (аналог M.P. если нет $\zeta$, то в точности он)
Здесь $\alpha$ --- секущая формула, число связок в $\neg\alpha$ --- степень сечения.\\
В отличие от других правил, в правиле сечения хотя бы одна из боковых формул $\zeta$ или $\delta$ должна присутствовать.


\begin{enumerate}
\item Доказательства образуют деревья.
\item Каждой формуле в дереве сопоставим порядковое число (ординал).
\item Порядковое число заключения любого неструктурного правила строго больше порядкового числа его посылок
(больше или равно в случае структурного правила).

%$$%\infer{(\forall a.a = a)_\omega}{
%  % \infer{(0 = 0)_1}{}\quad
%  % \infer{(0'= 0')_2}{\infer{\dots\vphantom{0}}{\infer{(0= 0)_1}{}}}\quad
%  % \infer{0''= 0''}{\infer{\dots\vphantom{0}}{\infer{0'= 0'}{\infer{\dots\vphantom{0}}{\infer{0= 0}{}}}}}\quad\dots
%\infer{(\forall a.a = a)_\omega}{
%   \infer{(0 = 0)_1}{}\quad
%   \infer{(0'= 0')_2}{}\quad
%   \infer{(0''= 0'')_3}{}\quad\dots
%}\quad\quad
%\infer{(\forall a.a = a)_1}{
%   \infer{(0 = 0)_0}{}\quad
%   \infer{(0'= 0')_0}{}\quad
%   \infer{(0''= 0'')_0}{}\quad\dots
%}$$

$$\infer{(\neg\neg\forall x.\neg x'=0)_{\omega+1}}{\infer{(\forall x.\neg x' = 0)_\omega}{(\neg 1=0)_1\quad (\neg 2=0)_2 \quad (\neg 3=0)_4 \quad (\neg 4 = 0)_8 \dots}}$$

\item Существует конечная максимальная степень сечения в дереве (назовём её степенью вывода).
\end{enumerate}



\begin{thm}Если $\vdash_\text{фа}\alpha$, то $\vdash_\infty|\alpha|_\infty$ \end{thm}
\begin{exm}Обратное неверно: $$\infer{\forall x.\neg\omega_1(x,\overline{\ulcorner\sigma\urcorner})}
{\neg\omega_1(\overline{0},\overline{\ulcorner\sigma\urcorner})\quad\quad
 \neg\omega_1(\overline{1},\overline{\ulcorner\sigma\urcorner})\quad\quad
 \neg\omega_1(\overline{2},\overline{\ulcorner\sigma\urcorner})\quad\quad\dots}$$
\end{exm}
\begin{thm}Если Ф.А. противоречива, то противоречива и $S_\infty$\end{thm}



\begin{thm}\vspace{-0.3cm}
$$\infer{\neg\alpha\vee\delta\quad\neg\beta\vee\delta\vphantom{overline{1}]}}{\neg(\alpha\vee\beta)\vee\delta}\quad
  \infer{\alpha\vee\delta\vphantom{overline{1}]}}{\neg\neg\alpha\vee\delta}\quad
  \infer{\alpha[x:=\overline{0}]\vee\delta
          \quad\alpha[x:=\overline{1}]\vee\delta
          \quad\alpha[x:=\overline{2}]\vee\delta\quad\dots}{(\forall x.\alpha)\vee\delta}
$$\vspace{-0.5cm}
%Если формула $\alpha$ доказана и имеет вид, похожий на заключение правил де Моргана, 
%отрицания и бесконечной индукции --- то посылки соответствующих правил могут быть получены из самой 
%формулы $\alpha$ доказательством, причём доказательством с не большей степенью и не большим порядком.
\end{thm}
\begin{proof}
\begin{tabular}{ll}\begin{minipage}{0.5\linewidth}
Например, формула вида $\neg\neg \alpha\vee\delta$. 
\vspace{0.2cm}Проследим историю $\neg\neg\alpha$; она могла быть получена:
\begin{enumerate}
\item ослаблением --- заменим $\neg\neg\alpha$ на $\alpha$ в этом узле и последующих.
\item отрицанием --- выбросим правило, заменим $\neg\neg\alpha$ на $\alpha$ в последующих.
\end{enumerate}
%Изменённый вывод --- доказательство требуемого.

\end{minipage} &
\begin{minipage}{0.5\linewidth}
\tikz{
  \node at (-1.5,3) (J1) { $\delta(0)$ };
  \node at (1.5,3) (J2) { $\alpha\vee\delta(2)$ };
  \node at (-1.5,1.5) (I1) { ${\color{red}\neg\neg}\alpha\vee\delta(0)$ };
  \node at (1.5,1.5) (I2) { $\color{red}\neg\neg\alpha\vee\delta(2)$ };
  \node at (1.5,0) (C) { ${\color{red}\neg\neg}\alpha\vee\forall x.\delta(x)$ }; 
  \node at (3.5,1.5) (D) { $\dots$ };
  \draw[->] (J1) -- (I1); \draw[->] (I1) -- (C);
  \draw[red,->] (J2) -- (I2); \draw[red,->] (I2) -- (C);
  \draw[->] (D) -- (C);
  \draw[blue,->,bend right=20] (J2) .. controls (0,1.5) .. (C);
}\end{minipage}
\end{tabular}
\end{proof}

\begin{thm}Если $\alpha$ имеет вывод степени $m>0$ порядка $t$, то
можно найти вывод степени строго меньшей $m$ с порядком $2^t$.
\end{thm}

\begin{proof}Трансфинитная индукция. Пусть для всех деревьев порядка $t_1 < t$ 
условие выполнено. Покажем, что оно выполнено для порядка $t$.
Рассмотрим заключительное правило. Это может быть...

\begin{enumerate}
\item Не сечение.
\item Сечение, секущая формула --- элементарная.
\item Сечение, секущая формула --- $\neg\alpha$.
\item Сечение, секущая формула --- $\alpha\vee\beta$.
\item Сечение, секущая формула --- $\forall x.\alpha$.
\end{enumerate}
\end{proof}



$$\infer{(\alpha)_{t}}{(\pi_0)_{t_0}\quad(\pi_1)_{t_1}\quad(\pi_2)_{t_2}\quad\dots}$$
Заменим доказательства посылок $(\pi_i)_{t_i}$ на $(\pi'_i)_{2^{t_i}}$ по индукционному предположению. (Что если ti $=$ t?)

\begin{enumerate}
\item Поскольку степени посылок $m'_i < m_i$, то $\max m'_i < \max m_i$.
\item Поскольку $t_i \le t$, то $2^{t_i} \le 2^t$.
\end{enumerate}


\vspace{-0.1cm}$$\infer{\zeta\vee\delta}{\zeta\vee\forall x.\alpha\quad\quad(\neg\forall x.\alpha)\vee\delta}$$\vspace{-0.1cm}
Причём степень и порядок выводов компонент, соответственно, $(m_1,t_1)$ и $(m_2,t_2)$.\vspace{-0.1cm}
\begin{enumerate}
\item По индукции, вывод $\zeta\vee\forall x.\alpha$ можно упростить до $(m_1',2^{t_1})$.
\item По обратимости, можно построить вывод $\zeta\vee\alpha[x := \theta]$ за $(m_1',2^{t_1})$ [предыдущая теорема предполагает, что степень и порядок при перестроении не увеличиваются].
\item В формуле $(\neg \forall x. \alpha)\vee\delta$ формула $\neg\forall x.\alpha$ получена
либо ослаблением, либо квантификацией из $\neg\alpha[x := \theta_k]\vee\delta_k$. 
\begin{enumerate}
\item Каждое правило квантификации заменим на:
$$\infer{\zeta\vee\delta_k}{\zeta\vee\alpha[x := \theta_k]\quad\quad(\neg\alpha[x := \theta_k])\vee\delta_k}$$
\item Остальные вхождения $\neg\forall x.\alpha$ заменим на $\zeta$ (в правилах ослабления).
\end{enumerate}
\item В получившемся дереве меньше степень --- так как в $\neg\alpha[x := \theta]$ меньше связок, чем в $\neg\forall x.\alpha$.
%\item Нумерацию можно также перестроить.
\end{enumerate}



\begin{center}\tikz{
    \node (RRW) at (6,5.5) {$\delta_l$};
    \node (RR) at (6,4) {${\color{red}(\neg\forall x.\alpha)}\vee\delta_l$};
    \node (RRnew) at (6.5,3) {$\color{blue}\zeta\vee\delta_l$};
    \node (LL) at (-2,5) {$\dots$};

    \node (R) at (3.2,3) {${\color{red}(\neg\forall x.\alpha)}\vee\delta_k$};
    \node (Rnew) at (1,3) {$\color{blue}\zeta\vee\delta_k$};
    \node (RQ) at (3,5) {$\neg\alpha[x := \theta]\vee\delta_k$};
    \node (L1) at (0,4) {$\color{blue}\zeta\vee\alpha[x := \theta]$};
    \node (L) at (-2,2) {$\color{red}\zeta\vee\forall x.\alpha$};

    \node (CR) at (4.5,2) {${\color{red}(\neg\forall x.\alpha)}\vee\delta$};
    \node (CRnew) at (5,1) {$\color{blue}\zeta\vee\delta$};
    \node (C) at (0,0) {$\color{red}\zeta\vee\delta$};
    \draw[red,->] (L) -- (C);
    \draw[red,->] (LL) -- (L);
    \draw[red,->] (CR) -- (C);
    \draw[blue,->,bend right=30] (LL) to (L1);
    \draw[dashed,->] (R) -- (CR);
    \draw[dashed,->] (RR) -- (CR);
    \draw[double,->,blue] (RR) -- (RRnew);
    \draw[double,->,blue] (CR) -- (CRnew);
    \draw[double,->,blue] (R) -- (Rnew);
    \draw[->] (RRW) -- (RR);
    \draw[->] (RQ) -- (R);
    \draw[blue,->,bend right=20] (RQ) to (Rnew);
    \draw[blue,->] (L1) -- (Rnew);
}\end{center}



\begin{dfn}Итерационная экспонента
$$(a\uparrow)^m(t
) = 
  \left\{
    \begin{array}{ll}     t,&m=0\\
                          a^{(a\uparrow)^{m-1}(t)},&m > 0
    \end{array}
  \right.
$$
\end{dfn}
\begin{thm}Если $\vdash_\infty\sigma$ степени $m$ порядка $t$, то найдётся доказательство без сечений
порядка $(2\uparrow)^m(t)$
\end{thm}
\begin{proof}
В силу конечности $m$ воспользуемся индукцией по $m$ и теоремой об уменьшении степени.
\end{proof}


\begin{dfn}$\varepsilon_0$ --- неподвижная точка $\varepsilon_0 = \omega^{\varepsilon_0}$\end{dfn}

Иначе говоря, $\varepsilon_0 = \{ \omega, \omega^\omega, \omega^{\omega^\omega}, (\omega \uparrow)^3(\omega), (\omega\uparrow)^4(\omega), \dots \}$.

Очевидно, что теорема об устранении сечений может быть доказана трансфинитной индукцией до ординала $\varepsilon_0$
(максимальный порядок дерева вывода, при правильной нумерации вершин).



\begin{lmm}Если $\vdash_\infty\alpha$ и $\vdash_\infty\neg\alpha$, тогда $\vdash_\infty\neg 0=0$.
\end{lmm}

\begin{thm}$\not\vdash_\infty\neg 0=0$\end{thm}
\begin{proof} Пусть $\vdash_\infty\neg 0=0$, устраним сечения и рассмотрим заключительное правило.
\begin{enumerate}
\item Правило де Моргана?  Нет отрицаний дизъюнкции ($\neg(\alpha\vee\beta)\vee\delta$). \item Отрицание?  Нет двойного отрицания ($\neg\neg\alpha\vee\delta$). \item Бесконечная индукция или квантификация?  Нет квантора. \item Ослабление?  Нет дизъюнкции ($\alpha \vee \beta$), хотя $\beta$ обязана присутствовать. \item Сечение?  Исключено по условию.
\end{enumerate}

То есть, неизбежно, $\neg 0=0$ --- аксиома, что также неверно.
\end{proof}

\section{Сколемизация. Эрбранов универсум, основные термы, эрбранова интерпретация,
система дизъюнктов, основные примеры, система основных примеров, теорема Гёделя о компактности,
теорема Эрбрана. Правило резолюции (для исчисления высказываний и для исчисления предикатов),
задачи унификации, уравнения в алгебраических термах, наибольший общий унификатор.
Общая формулировка метода резолюции. SMT-решатели.}

\section{Лямбда-исчисление. Пред- и лямбда-термы. Альфа-эквивалентность, бета-редукция, бета-эквивалентность.
Теорема Чёрча-Россера (формулировка). Нормальная форма и её единственность (с доказательством).
Представление истины и лжи, чёрчевские нумералы, арифметические функции (сложение, умножение, вычитание).
Комбинатор неподвижной точки. Импликационный фрагмент ИИВ. Замкнутость импликационного фрагмента ИИВ 
(формулировка). Типизация лямбда-исчисления по Чёрчу и по Карри. Гильбертов вывод и комбинаторы.}

\section{Модальная логика, системы K, K4, T, S4, S5. Линейная темпоральная логика. Построение формулы для выбранного
инварианта (например, условия на семафоры критической секции для двух потоков).
Проверка на моделях, постановка задачи. Система переходов. Автоматы Бюхи. 
Построение автомата Бюхи по данной формуле ЛТЛ. Схема алгоритма, проверяющего с помощью моделей 
соответствие алгоритма утверждению в ЛТЛ.}

\end{document}
