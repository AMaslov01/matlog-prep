\documentclass[11pt,a4paper,oneside]{scrartcl}
\usepackage[utf8]{inputenc}
\usepackage[english,russian]{babel}
\usepackage[top=1cm,bottom=1cm,left=1cm,right=1cm]{geometry}
\usepackage{amsmath}
\usepackage{amssymb}
\usepackage{stmaryrd}
\usepackage{cmll}
\usepackage{xcolor}
\usepackage{proof}
\usepackage{comment}
\usepackage{titletoc}
\usepackage{amsthm}
\newtheorem*{thm}{Теорема}
\newtheorem*{dfn}{Определение}
\theoremstyle{definition}
\newtheorem*{exm}{Пример}
\newtheorem*{axm}{Аксиома}
\newtheorem*{lmm}{Лемма}
\newtheorem*{snote}{Замечание}
\setcounter{tocdepth}{1}
\usepackage{hyperref}
\usepackage{tikz}
\usetikzlibrary{hobby,fit,backgrounds,calc,shapes.geometric,patterns}
\hypersetup{
    colorlinks=true,
    linkcolor=blue,
    filecolor=magenta,
    urlcolor=cyan,
}
\usepackage{graphicx} % Required for inserting images
\newcommand{\pause}{} % Compatibility with beamer presentations
\newcommand{\mdoubleplus}{\mathbin{+\!\!+}} % String concatenation

\title{Matlog Exam}
\author{Artemiy Maslov, Aleksandr Tarasov}
\date{January 2026}

\begin{document}

    \maketitle

    \newpage

    \tableofcontents

    \newpage

    \section{Исчисление высказываний. Предметный язык и язык исследователя (метаязык).
    Язык исчисления высказываний. Оценка высказываний, общезначимость, следование.
    Доказуемость, гипотезы (контекст), выводимость. Корректность, полнота, противоречивость и
    непротиворечивость (эквивалентные формулировки). Теорема о дедукции для исчисления высказываний.}

    Для задания\textbf{ исчисления высказываний} необходим предметный язык, теория моделей и теория доказательств.

    \textbf{Предметный язык}(формальный язык, тексты на котором мы будем анализировать) КИВ состоит из высказываний.

    \textbf{Высказывание} - это строка, которая является либо атомарным высказыванием(пропозициональной переменной), либо составным высказыванием, построенным с помощью отрицания, конъюнкции, дизъюнкции или импликации.

    \textbf{Метаязык} - язык, с помощью которого мы анализируем предметный язык.

    \text {Для построения }\textbf{оценки высказываний} необходимо задать множество истинностных значений $V = \{\textit{И},\textit{Л}\,\}$ \text{ и определить функцию оценки} $f: \mathcal{P} \rightarrow V$. \text{Оценки высказываний задаются рекурсивно, интуитивным способом.}

    Если $\alpha$ истинна при любой оценке переменных, то она \textbf{общезначима}: $\models \alpha$.

    Если $\alpha$ истинна при любой оценке переменных, при которой истинны
    высказывания $\gamma_1, \dots, \gamma_n$, то $\alpha$ --- \textbf{следствие} этих высказываний: $\gamma_1, \dots, \gamma_n \models \alpha$.

    Схема высказываний - высказывания с метапеременными.

    Будем говорить, что высказывание $\sigma$ строится по схеме $\textit{Ш}$,
    если существует такая замена метапеременных $\textit{ч}_1$, $\textit{ч}_2$, ..., $\textit{ч}_n$
    в схеме $\textit{Ш}$ на какие-либо выражения $\varphi_1$, $\varphi_2$, ..., $\varphi_n$,
    что после её проведения получается высказывание $\sigma$: $\sigma = \textit{Ш}[\textit{ч}_1 := \varphi_1][\textit{ч}_2 := \varphi_2]...[\textit{ч}_n := \varphi_n]$.

    Аксиомы ИВ - следующие схемы высказываний:

    \begin{tabular}{ll}
        (1) & $\alpha \rightarrow \beta \rightarrow \alpha$ \\
    (2) & $(\alpha \rightarrow \beta) \rightarrow (\alpha \rightarrow \beta \rightarrow \gamma) \rightarrow (\alpha \rightarrow \gamma)$ \\
    (3) & $\alpha \rightarrow \beta \rightarrow \alpha \& \beta$\\
    (4) & $\alpha \& \beta \rightarrow \alpha$\\
    (5) & $\alpha \& \beta \rightarrow \beta$\\
    (6) & $\alpha \rightarrow \alpha \vee \beta$\\
    (7) & $\beta \rightarrow \alpha \vee \beta$\\
    (8) & $(\alpha \rightarrow \gamma) \rightarrow (\beta \rightarrow \gamma) \rightarrow (\alpha \vee \beta \rightarrow \gamma)$\\
    (9) & $(\alpha \rightarrow \beta) \rightarrow (\alpha \rightarrow \neg \beta) \rightarrow \neg \alpha$\\
    (10) & $\neg \neg \alpha \rightarrow \alpha$
    \end{tabular}

    Правило вывода MP - Если имеет место $\alpha$ и $\alpha\rightarrow\beta$, то имеет место $\beta$.\\

    \textbf{Доказательством в ИВ} назовём конечную последовательность высказываний $\delta_1, \delta_2, \dots, \delta_n$, причём каждое $\delta_i$ либо:
    \begin{itemize}
        \item является аксиомой(существует замена переменных для какой-то схемы)
        \item получается из $\delta_1,\dots,\delta_{i-1}$ по правилу Modus Ponens.
    \end{itemize}

    \textbf{Доказательство формулы} $\alpha$ - такое доказательство $\delta_1, \delta_2, \dots, \delta_n$, что $\alpha\equiv\delta_n$.\\

    Формула \textbf{доказуема}, если существует доказательство.\\

    \textbf{Вывод} формулы $\alpha$ из гипотез $\gamma_1,\dots,\gamma_k$ - последовательность
    $\delta_1,\dots,\delta_n$, причём каждое $\delta_i$ либо:
    \begin{itemize}
        \item является аксиомой;
        \item либо получается по правилу Modus Ponens из предыдущих;
        \item либо является одной из гипотез.
    \end{itemize}

    Формула $\alpha$ \textbf{выводима} из гипотез $\gamma_1,\dots,\gamma_k$, если существует её вывод.\\

    \textbf{Контекст} - список формул.\\

    Теория \textbf{корректна}, если любое доказуемое в ней утверждение общезначимо.\\

    Теория \textbf{полна}, если любое общезначимое в ней утверждение доказуемо.\\

    Теория \textbf{противоречива}, если найдётся такая формула $\alpha$, что $\vdash\alpha$ и $\vdash\neg\alpha$.

    Эквивалентно в КИВ и ИИВ:
    \begin{itemize}
        \item теория непротиворечива;
        \item $\not\vdash A \with \neg A$;
        \item Найдётся $\alpha$, что $\not\vdash\alpha$.
    \end{itemize}

    (\textbf{Теорема о дедукции}) $\Gamma,\alpha\vdash\beta$ выполнено тогда и только тогда, когда выполнено $\Gamma\vdash\alpha\rightarrow\beta$.

    $\Leftarrow$

    Пусть $\Gamma \vdash \alpha \rightarrow \beta$. Тогда существует вывод
    \[
        \delta_1, \delta_2, \dots, \delta_{n-1}, \alpha \rightarrow \beta,
    \]
    где каждая $\delta_i$ либо принадлежит $\Gamma$, либо получена по правилам вывода.
    Добавим к этому выводу формулу $\alpha$ (как гипотезу) и применим \textit{modus ponens}:
    \[
        \delta_1, \dots, \delta_{n-1},\; \alpha \rightarrow \beta,\; \alpha,\; \beta.
    \]
    Теперь $\beta$ выводима из $\Gamma \cup \{ \alpha \}$, т.е. $\Gamma, \alpha \vdash \beta$.

    $\Rightarrow$

    (индукция по длине вывода). Если $\delta_1, \dots, \delta_n$ --- вывод
    $\Gamma,\alpha\vdash\delta_n$, то найдётся вывод $\zeta_k$ для $\Gamma\vdash\alpha\rightarrow\delta_n$,
    причём $\zeta_1 \equiv \alpha\rightarrow\delta_1, \dots, \zeta_n \equiv \alpha\rightarrow\delta_n$.

    \begin{itemize}
        \item База $(n=1)$: частный случай перехода (без M.P.).

        \item Переход. Пусть $\delta_1, \dots, \delta_{n+1}$ --- исходный вывод. И пусть (по индукционному предположению)
        уже по начальному фрагменту $\delta_1, \dots, \delta_n$ построен вывод $\zeta_k$ утверждения
        $\Gamma\vdash\alpha\rightarrow\delta_n$.

        Но $\delta_{n+1}$ как-то был обоснован --- разберём случаи:
        \begin{enumerate}
            \item $\delta_{n+1}$ --- аксиома или $\delta_{n+1} \in \Gamma$ %(выполнено без доказательства в новом выводе)\pause
            \item $\delta_{n+1}\equiv\alpha$\pause
            \item $\delta_{n+1}$ --- Modus Ponens из $\delta_j$ и
            $\delta_k \equiv \delta_j\rightarrow\delta_{n+1}$.
        \end{enumerate}

        В каждом из случаев можно дополнить черновик до полноценного вывода.
    \end{itemize}

        \section{Теорема о полноте исчисления высказываний. Условное отрицание. 14 лемм о связках.
        Лемма об устранении посылок. Доказательство теоремы.}

        (\textbf{Теорема о полноте исчисления высказываний}) Если $\models\alpha$, то $\vdash\alpha$.\\

        \begin{proof}
            Пусть $\llbracket\alpha\rrbracket = x$. Тогда \textbf{условное отрицание} задается как $\llparenthesis\alpha\rrparenthesis$:
            $$\llparenthesis\alpha\rrparenthesis = \left\{\begin{array}{ll}\alpha, & x = \textnormal{И}\\
                \neg\alpha, & x = \textnormal{Л}\end{array}\right.$$

            (\textbf{Лемма о связках}) $\llparenthesis\varphi\rrparenthesis, \llparenthesis\psi\rrparenthesis \vdash \llparenthesis\varphi\star\psi\rrparenthesis$

            \begin{center}\begin{tabular}{rclp{1cm}rcl}
                              $\neg\varphi, \neg\psi$&$ \vdash $&$\neg (\varphi \with \psi)$& & $\neg\varphi, \neg\psi$&$ \vdash $&$     (\varphi \rightarrow  \psi)$ \\
                              $\neg\varphi,     \psi$&$ \vdash $&$\neg (\varphi \with \psi)$& &$\neg\varphi,     \psi$&$ \vdash $&$     (\varphi \rightarrow  \psi)$ \\
                              $    \varphi, \neg\psi$&$ \vdash $&$\neg (\varphi \with \psi)$& &$ \varphi, \neg\psi$&$ \vdash $&$\neg (\varphi \rightarrow  \psi)$ \\
                              $    \varphi,     \psi$&$ \vdash $&$     (\varphi \with \psi)$& &$    \varphi,     \psi$&$ \vdash $&$     (\varphi \rightarrow  \psi)$ \\
                              $\neg\varphi, \neg\psi$&$ \vdash $&$\neg (\varphi \vee  \psi)$& &$    \varphi          $&$ \vdash $&$     \neg\neg\varphi$ \\
                              $\neg\varphi,     \psi$&$ \vdash $&$     (\varphi \vee  \psi)$& &$\neg\varphi          $&$ \vdash $&$         \neg\varphi$\\
                              $    \varphi, \neg\psi$&$ \vdash $&$     (\varphi \vee  \psi)$ \\
                              $    \varphi,     \psi$&$ \vdash $&$     (\varphi \vee  \psi)$
                \end{tabular}\end{center}

            (\textbf{Лемма об условном отрицании формул}) Пусть пропозициональные переменные $\Xi := \{X_1, \dots, X_n\}$ ---
            все переменные, которые используются в формуле $\alpha$. И пусть
            задана некоторая оценка переменных.

            Тогда, $\llparenthesis \Xi \rrparenthesis \vdash\llparenthesis\alpha\rrparenthesis$

            Индукция по длине формулы $\alpha$.
            \begin{itemize}
                \item База: формула $\alpha$ --- атомарная, т.е. $\alpha \equiv X_i$. Тогда при любом $\Xi$ выполнено
                $\llparenthesis\Xi\rrparenthesis^{X_i := \text{И}} \vdash X_i$ и $\llparenthesis\Xi\rrparenthesis^{X_i := \text{Л}} \vdash \neg X_i$.
                \item Переход: $\alpha \equiv \varphi\star\psi$, причём $\llparenthesis\Xi\rrparenthesis\vdash\llparenthesis\varphi\rrparenthesis$
                и $\llparenthesis\Xi\rrparenthesis\vdash\llparenthesis\psi\rrparenthesis$
                Тогда построим вывод:

                \begin{tabular}{lll}
                    $(1)\dots(n)$ & $\llparenthesis\varphi\rrparenthesis$ & индукционное предположение\\
                    $(n+1)\dots(k)$ & $\llparenthesis\psi\rrparenthesis$ & индукционное предположение\\
                    $(k+1)\dots(l)$ & $\llparenthesis\varphi\star\psi\rrparenthesis$ &
                    лемма о связках: $\llparenthesis\varphi\rrparenthesis$ и $\llparenthesis\psi\rrparenthesis$ доказаны выше,\\
                    & & значит, их можно использовать как гипотезы
                \end{tabular}
            \end{itemize}


            (\textbf{Лемма об устранении посылок}) Пусть при всех оценках переменных
            $\llparenthesis\Xi\rrparenthesis \vdash \alpha$, тогда
            $\vdash\alpha$.

            Индукция по количеству переменных $n$.

            \begin{itemize}
                \item База: $n=0$. Тогда $\vdash\alpha$ есть из условия.\item Переход: пусть $\llparenthesis X_1, X_2,  \dots X_{n+1} \rrparenthesis \vdash \alpha$.
                Рассмотрим $2^n$ пар выводов: $$\llparenthesis X_1, X_2, \dots X_n\rrparenthesis,X_{n+1} \vdash \alpha\quad\quad\llparenthesis X_1, X_2, \dots X_n\rrparenthesis,\neg X_{n+1} \vdash \alpha$$
                По лемме об исключении допущения тогда
                $$\llparenthesis X_1, X_2, \dots X_n \rrparenthesis \vdash \alpha$$
            \end{itemize}

            В виду общезначимости, $\llparenthesis\Xi\rrparenthesis\vdash\alpha$.
            При этом, $\llparenthesis X_1, X_2, \dots X_n \rrparenthesis  \vdash \alpha$ при всех оценках
            переменных $X_1, \dots X_n$. Значит, $\vdash\alpha$ по индукционному предположению.

        \end{proof}

        \section{Топологические пространства. Определение. Примеры (топология стрелки, Зарисского,
            топология на деревьях). Открытые и замкнутые множества. Связность. Компактность. Непрерывные функции.}

        \textbf{Топологическим пространством} называется упорядоченная пара $\langle X, \Omega \rangle$,
        где $X$ --- некоторое множество, а $\Omega \subseteq \mathcal{P}(X)$, причём:
        \begin{enumerate}
            \item $\varnothing, X \in \Omega$
            \item если $A_1, \dots, A_n \in \Omega$, то $A_1 \cap A_2 \cap \dots \cap A_n \in \Omega$;
            \item если $\{A_\alpha\}$ --- семейство множеств из $\Omega$, то и $\bigcup_\alpha A_\alpha \in \Omega$.
        \end{enumerate}


        Множество $\Omega$ называется топологией. Элементы $\Omega$ называются \textbf{открытыми множествами}.\\

        Внутренность множества $A^\circ$ --- наибольшее $T$, что $T \in \Omega$ и $T \subseteq A$.\\

        Функция $f: X \rightarrow Y$ \textbf{непрерывна}, если прообраз любого открытого множества открыт.\\

        Множество \textbf{компактно}, если из любого его открытого покрытия можно выбрать конечное
        подпокрытие.\\

        Пространство $\langle X, \Omega\rangle$ \textbf{связно}, если нет $A,B \in \Omega$, что $A\cup B = X$,
        $A \cap B = \varnothing$ и $A,B \ne \varnothing$.\\
        \newpage

        \textbf{Примеры}
        \begin{itemize}
            \item Топология стрелки

            X = $\mathbb{R}$,

            $\Omega = \{\, U \subset \mathbb{R} \mid U = \bigcup_{\alpha} [a_\alpha,b_\alpha),\ a_\alpha<b_\alpha \,\}.$
            \item Топология Зарисского

            X = k,

            $\Omega = \{\, U \subset X \mid X \setminus U$ \text{ — конечное множество} $\,\}.$
            \item Топология на деревьях

            Пусть некоторый лес задан конечным множеством вершин $V$ и
            отношением $(\preceq)$, связывающим предков и потомков ($a \preceq b$, если $b$ --- потомок $a$). Тогда подмножество его вершин $X\subseteq V$ назовём открытым,
            если из $a \in X$ и $a \preceq b$ следует, что $b \in X$.

        \end{itemize}

        \section{ Гильбертов вывод и натуральный вывод. Интуиционистское исчисление высказываний.
        Доказательства чистого существования. BHK-интерпретация.
        Закон исключённого третьего, принцип взрыва, связь с КИВ и ИИВ.
        Решётки. Дистрибутивная решётка. Псевдодополнение. Булевы и псевдобулевы алгебры.}

        \textbf{Натуральный вывод, связь с КИВ}
        \begin{itemize}
            \item Формулы языка (секвенции) имеют вид: $\Gamma\vdash\alpha$.
            \item Аксиома:\\$\infer[\text{(акс.)}]{\Gamma,\alpha\vdash\alpha}{\vphantom{\Gamma}}$

            \item Правила введения связок:\\$\infer{\Gamma\vdash\alpha\rightarrow\beta}{\Gamma,\alpha\vdash\beta}\quad\quad\infer{\Gamma\vdash\alpha\vee\beta}{\Gamma\vdash\alpha}$, $\infer{\Gamma\vdash\alpha\vee\beta}{\Gamma\vdash\beta}\quad\quad\infer{\Gamma\vdash\alpha\with\beta}{\Gamma\vdash\alpha\quad\quad\Gamma\vdash\beta}$

            \item Правила удаления связок:\\$\infer{\Gamma\vdash\beta}{\Gamma\vdash\alpha\quad\Gamma\vdash\alpha\rightarrow\beta}\quad\quad\infer{\Gamma\vdash\gamma}{\Gamma\vdash\alpha\rightarrow\gamma\quad\Gamma\vdash\beta\rightarrow\gamma\quad\Gamma\vdash\alpha\vee\beta}$
            $\infer{\Gamma\vdash\alpha}{\Gamma\vdash\alpha\with\beta}\quad\quad\infer{\Gamma\vdash\beta}{\Gamma\vdash\alpha\with\beta}\quad\quad\infer{\Gamma\vdash\alpha}{\Gamma\vdash\bot}$
            \item Пример доказательства:\vspace{-0.3cm}
            $$\infer[(\text{введ}\with)]{A\with B\vdash B \with A}{\infer[(\text{удал}\with)]{A \with B \vdash B}{\infer[(\text{акс.})]{A \with B\vdash A \with B}{}}
            \quad\quad\infer[(\text{удал}\with)]{A \with B \vdash A}{\infer[(\text{акс.})]{A \with B\vdash A \with B}{}}}$$
        \end{itemize}

        Из важного - гипотезы включены в формулу, вместо $\neg$ нульместный $\bot$.\\
        Классический нормальный вывод получится при замене принципа взрыва на снятие двойного отрицания: $$\infer{\Gamma\vdash A}{\Gamma\vdash (A\rightarrow\bot)\rightarrow\bot}.$$
        Нормальный и гильбертов вывод эквивалентны в любой логике. КИВ строго сильнее ИИВ(например, в ИИВ нельзя доказать исключенное третье).

        Можно построить аксиоматику ИИВ в гильбертовском стиле, заменив аксиому  $\neg \neg \alpha \rightarrow \alpha$ на $\alpha \rightarrow \neg\alpha \rightarrow \beta$.\\

        \textbf{Интуиционистское исчисление высказываний}

        Основные положения:
        \begin{enumerate}
            \item Математика не формальна.
            \item Математика независима от окружающего мира.
            \item Математика не зависит от логики — это логика зависит от математики.
        \end{enumerate}

        В интуиционизме \textbf{доказательства чистого существования} не конструктивны, поскольку не позволяют построить сам объект. Применяется \textbf{ВНК-интерпретация}:
        \begin{itemize}
            \item $\alpha\ \&\ \beta$ построено, если построены $\alpha$ и $\beta$ \item $\alpha \vee \beta$ построено, если построено $\alpha$ или $\beta$,
            и мы знаем, что именно \item $\alpha\rightarrow\beta$ построено, если есть способ перестроения
            $\alpha$ в $\beta$\item $\bot$ — конструкция, не имеющая построения\item $\neg\alpha$ построено, если построено $\alpha\rightarrow\bot$
        \end{itemize}

        Возьмём за $\alpha$ нерешённую проблему, тогда конструкция $\alpha\vee\neg\alpha$ не имеет построения.\\

        Множество нижних граней $X\subseteq\mathcal{U}$: $\mbox{\upshape lwb}_\mathcal{U} X = \{ y\in \mathcal{U}\ |\ y \preceq x$\text{ при всех }$ x \in X\}$.\\
        Минимальный ($m \in X$): нет меньшего --- при всех $y \in X$, $y \preceq m$ влечёт $y = m$ \\
        Наименьший ($m \in X$): меньше всех --- при всех $y \in X$ выполнено $m \preceq y$\\
        Инфимум: наибольшая нижняя грань --- $\inf_\mathcal{U} X = \mbox{\upshape наиб}(\mbox{\upshape lwb}_\mathcal{U} X)$\\

        \textbf{Решёткой} называется упорядоченная пара: $\langle X, (\preceq)\rangle$,
        где $X$ --- некоторое множество, а $(\preceq)$ --- частичный порядок на $X$, такой,
        что для любых $a,b \in X$ определены $a + b = \sup\{a,b\}$ и $a \cdot b = \inf\{a,b\}$.

        \textbf{Псевдодополнением} $a \rightarrow b$ называется наибольший из $\{ x \ |\ a \cdot x \preceq b\}$.

        \begin{center}\tikz{
            \node[circle,inner sep=0.3] (A) at (1,-1.3) {$a$};
            \node[fill=cyan!0.2,circle,inner sep=0.3] (B) at (0.5,-0.5) {$b$};
            \node[circle,inner sep=0.3] (C) at (1.5,-0.5) {$c$};
            \node[circle,inner sep=0.3] (D) at (1,0.3) {$d$};
            \foreach \b/\e in {A/B, A/C, B/D, C/D} {
                \draw[stealth-, line width=1, color=black!50!green] (\b) to (\e);
            }
            \draw[fill=cyan,fill=cyan,opacity=0.2](C.north west)
            to[closed,curve through={
                (A.south west) .. (A.south east)
            }] (C.north east);
        }\end{center}
        Здесь $b \rightarrow c = \text{наиб}\{x\ |\ b \cdot x \preceq c\} = \text{наиб}\{ a, c \} = c$.\\

        \textbf{Дистрибутивной решёткой} называется такая, что для любых $a,b,c$ выполнено
        $a \cdot (b + c) = a \cdot b + a \cdot c$.

        Импликативная решётка --- такая, в которой для любых элементов есть псевдодополнение.(Любая импликативная - дистрибутивная.

        0 --- наименьший элемент решётки, а 1 --- наибольший элемент решётки.\\

        Импликативная решётка с 0 --- \textbf{псевдобулева алгебра} (алгебра Гейтинга).
        В такой решётке определено $\sim a := a \rightarrow 0$.

        \textbf{Булева алгебра} --- псевдобулева алгебра, в которой $a\ + \sim a = 1$ для всех $a$.\\


        Символы булевой алгебры: $(\with),(\vee),(\neg),$\text{F},\text{T}.\\
        Символы решёток: $(+),(\cdot),(\rightarrow),(\sim),0,1$\\
        Упорядочивание: $\text{F} \le \text{T}$.

        \begin{enumerate}
            \item $a \with b = \min(a,b)$, $a \vee b = \max(a,b)$
            (анализ таблицы истинности), отсюда $a \cdot b = a \with b$ и $a + b = a \vee b$.

            \item $a \rightarrow b = \neg a \vee b$, так как:
            $$a \rightarrow b = \text{max}\{ c | c \with a \le b\} = \left\{\begin{array}{ll}\neg a,& b = \text{F}\\
                \text{T},& b = \text{T}\end{array}\right.$$

            \item $0 = \min\{\text{T},\text{F}\} = \text{F}$, $1 = \max\{\text{T},\text{F}\} = \text{T}$, $\sim a = a \rightarrow 0 = \neg a \vee \text{F} = \neg a$.
            Заметим, что $a\ + \sim a = a \vee \neg a = \text{T}$.
        \end{enumerate}
        Итого: булева алгебра --- импликативная решётка с 0 и с $a\ + \sim a = 1$.\\

        Пусть некоторое исчисление высказываний оценивается значениями из некоторой решётки.
        Назовём оценку согласованной с исчислением, если
        $\llbracket\alpha\with\beta\rrbracket = \llbracket\alpha\rrbracket\cdot\llbracket\beta\rrbracket$,
        $\llbracket\alpha\vee\beta\rrbracket = \llbracket\alpha\rrbracket+\llbracket\beta\rrbracket$,
        $\llbracket\alpha\rightarrow\beta\rrbracket = \llbracket\alpha\rrbracket\rightarrow\llbracket\beta\rrbracket$,
        $\llbracket\neg\alpha\rrbracket =\ \sim\llbracket\alpha\rrbracket$,
        $\llbracket A \with\neg A\rrbracket = 0$, $\llbracket A\rightarrow A \rrbracket = 1$. \\

        Любая псевдобулева алгебра, являющаяся согласованной оценкой интуиционистского исчисления высказываний,
        является его корректной моделью: если $\vdash\alpha$, то $\llbracket\alpha\rrbracket = 1$.\\

        Любая булева алгебра, являющаяся согласованной оценкой классического исчисления высказываний,
        является его корректной моделью: если $\vdash\alpha$, то $\llbracket\alpha\rrbracket = 1$.

        \section{Алгебра Линденбаума. Полнота интуиционистского исчисления высказываний в псевдобулевых
        алгебрах. Модели Крипке. Вынужденность. Сведение моделей Крипке к псевдобулевым алгебрам.
        Нетабличность ИИВ (формулировка теоремы).}

        Определим предпорядок на высказываниях: $\alpha \preceq \beta := \alpha \vdash \beta$ в интуиционистском исчислении высказываний.
        Также $\alpha\approx\beta$, если $\alpha\preceq\beta$ и $\beta\preceq\alpha$.

        Пусть $L$ --- множество всех высказываний. Тогда \textbf{алгебра Линденбаума} $\mathcal{L} = L/_\approx$.\\

        (Теорема) Интуиционистское исчисление высказываний \textbf{полно в псевдобулевых алгебрах}:
        если $\models\alpha$ во всех псевдобулевых алгебрах, то $\vdash\alpha$.

        \begin{proof}Возьмём в качестве модели исчисления алгебру Линденбаума:
            $\llbracket \alpha \rrbracket = [\alpha]_\mathcal{L}$.

            Пусть $\models\alpha$. Тогда $\llbracket\alpha\rrbracket = 1$ во всех псевдобулевых алгебрах, в том числе
            и $\llbracket\alpha\rrbracket = 1_\mathcal{L}$. То есть $[\alpha]_\mathcal{L} = [A\rightarrow A]_\mathcal{L}$.
            То есть $A \rightarrow A \approx \alpha$. Значит, в частности, $A \rightarrow A \vdash \alpha$.
            Значит, $\vdash\alpha$.\end{proof}

        \textbf{Модель Крипке} $\langle \mathcal{W}, (\preceq), (\Vdash)\rangle$:
        \begin{itemize}
            \item $\mathcal{W}$ --- множество миров, $(\preceq)$ --- нестрогий частичный порядок на $\mathcal{W}$;
            \item $(\Vdash)\subseteq \mathcal{W}\times P$ --- отношение вынуждения
            между мирами и переменными(завуалированная "истинность"), причём, если $W_i \preceq W_j$ и $W_i \Vdash X$, то $W_j \Vdash X$.
        \end{itemize}

        Доопределим вынужденность:
        \begin{itemize}
            \item $W \Vdash \alpha\with\beta$, если $W \Vdash \alpha$ и $W \Vdash \beta$;
            \item $W \Vdash \alpha\vee\beta$, если $W \Vdash \alpha$ или $W \Vdash \beta$;
            \item $W \Vdash \alpha\rightarrow\beta$, если всегда при $W \preceq W_1$ и $W_1 \Vdash \alpha$ выполнено $W_1 \Vdash \beta$
            \item $W \Vdash \neg\alpha$, если всегда при $W \preceq W_1$ выполнено $W_1 \not\Vdash \alpha$.
        \end{itemize}

        Будем говорить, что $\Vdash\alpha$, если $W\Vdash\alpha$ при всех $W \in \mathcal{W}$.
        Будем говорить, что $\models_\kappa\alpha$, если $\Vdash\alpha$ во всех моделях Крипке.

        (\textbf{Сведение к псевдобулевым алгебрам})Пусть $\langle \mathcal{W}, (\preceq), (\Vdash)\rangle$ ---
        некоторая модель Крипке.
        Тогда она есть корректная модель интуиционистского исчисления высказываний.

        \begin{proof}
            Доказательство для древовидного $(\preceq)$, обобщение на произвольный порядок легко построить.

            Заметим, что $V(\alpha) := \{ w \in \mathcal{W}\ |\ w\Vdash\alpha\}$ открыто в топологии для деревьев(истинно в узле, значит истинно в потомке).
            Значит, положив $V = \{\ S\ |\ S \subseteq \mathcal{W}\ \with\ S $\text{ --- открыто }$\}$ и
            $\llbracket \alpha \rrbracket = V(\alpha)$, получим алгебру Гейтинга(все открыто, операции переходят).
        \end{proof}

        \textbf{Табличная модель}

        Пусть задано $V$, значение $T \in V$ (<<истина>>), функция $f_P: P \rightarrow V$,
        функции $f_\with, f_\vee, f_\rightarrow : V \times V \rightarrow V$,
        функция $f_\neg: V \rightarrow V$.

        Тогда оценка $\llbracket X \rrbracket = f_P(X)$,
        $\llbracket \alpha\star\beta \rrbracket = f_\star(\llbracket \alpha \rrbracket, \llbracket \beta \rrbracket)$,
        $\llbracket \neg\alpha \rrbracket = f_\neg(\llbracket\alpha\rrbracket)$ --- табличная.

        Если $\vdash \alpha$ влечёт $\llbracket\alpha\rrbracket = T$ при всех оценках пропозициональных переменных $f_P$,
        то $\mathcal{M} := \langle V, T, f_\with, f_\vee, f_\rightarrow, f_\neg\rangle$ --- табличная модель.

        \textbf{(Лемма)}Табличная модель конечна, если $V$ конечно.

        \textbf{(Нетабличность ИИВ)}Не существует полной конечной табличной модели для интуиционистского исчисления высказываний.

        \begin{proof}
            Пусть существует полная конечная табличная модель $\mathcal{M}$, $V = \{v_1, v_2, \dots, v_n\}$(по лемме).
            То есть, если $\models_\mathcal{M}\alpha$, то $\vdash\alpha$.

            Рассмотрим $$\alpha_n =
            \bigvee_{1 \le p < q \le n+1} A_p \rightarrow A_q
            $$
            Рассмотрим оценку $f_P: \{A_1 \dots A_{n+1}\} \rightarrow \{v_1 \dots v_n\}$.
            По принципу Дирихле существуют $p \ne q$, что $\llbracket A_p \rrbracket = \llbracket A_q \rrbracket$.
            С другой стороны, $\vdash X \rightarrow X$ --- поэтому $f_\rightarrow(\llbracket X \rrbracket, \llbracket X \rrbracket) = T$,
            значит, $$\llbracket A_p \rightarrow A_q \rrbracket = f_\rightarrow(v,v) = f_\rightarrow (\llbracket A_p \rrbracket, \llbracket A_q \rrbracket) = f_\rightarrow(\llbracket X \rrbracket, \llbracket X \rrbracket) = T$$

            Аналогично, $\vdash \sigma \vee (X \rightarrow X) \vee \tau$, отсюда $\llbracket \alpha_n \rrbracket = \llbracket \sigma \vee (X \rightarrow X) \vee \tau \rrbracket = T$.

            Однако, в такой модели $\not\Vdash \alpha_n$:

            \begin{center}\tikz{
                \node at (0,0)   (R) {$W_R$};
                \node at (3,1.5) (A1) {$W_1$}; \node[right] at (3.5,1.5) (A11) {\color{black!50!red} $\Vdash A_1$};
                \draw[red,fill=red,opacity=0.2](A1.south west)
                to[closed,curve through={($(A1.south west)!0.5!(A1.south east)$) .. (A1.north east)}] (A1.north west);

                \node at (3,0.5) (A2) {$W_2$}; \node[right] at (3.5,0.5) (A21) {\color{black!50!magenta} $\Vdash A_2$};
                \draw[red,fill=magenta,opacity=0.2](A2.north west)
                to[closed,curve through={(A2.south west) .. (A2.south east)}] (A2.north east);
                \node at (3,-0.2) (A3) {$\dots$};
                \node at (3,-0.9) (A4) {$W_n$}; \node[right] at (3.5,-0.9) (A41) {\color{teal} $\Vdash A_n$};
                \draw[red,fill=teal,opacity=0.2](A4.north west)
                to[closed,curve through={($(A4.north west)!0.5!(A4.north east)$) .. (A4.south east)}] (A4.south west);

                \draw[->] (R) to (A1);
                \draw[->] (R) to (A2);
                \draw[->] (R) to (A4);

                \node[right] at (6,1.5) {Если $q > 1$, то}; \node[right] at (8.6, 1.5) {$W_1 \not\Vdash A_q$ и $W_1 \not\Vdash A_1 \rightarrow A_q$};
                \node[right] at (6,0.5) {Если $q > 2$, то}; \node[right] at (8.6, 0.5) { $W_2 \not\Vdash A_q$ и $W_2 \not\Vdash A_2 \rightarrow A_q$};
                \node[right] at (6,-0.5) {Если $q > n$, то}; \node[right] at (8.6,-0.5) {$W_n \not\Vdash A_{n+1}$; $W_n \not\Vdash A_n \rightarrow A_{n+1}$};
                \node[right] at (6,-1.5) {Если $p < q$, то}; \node[right] at (8.6, -1.5) { $W_p \not\Vdash A_q$ и $W_p \not\Vdash A_p \rightarrow A_q$};
            }
            \end{center}

            Если $p < q$, то $W_p \not\Vdash A_p \rightarrow A_q$, то есть $W_R \not\Vdash A_p \rightarrow A_q$(недоказуема в детях, значит недоказуема в потомках).

            Отсюда: $W_R \not\Vdash \bigvee_{p < q} A_p \rightarrow A_q$, $W_R \not\Vdash \alpha_n$,
            потому $\not\models \alpha_n$ и $\not\vdash \alpha_n$.

        \end{proof}

        \section{Гёделева алгебра. Операция $\Gamma(A)$. Дизъюнктивность ИИВ.
        Подрешётка. Разрешимость интуиционистского исчисления высказываний.}

        Исчисление \textbf{дизъюнктивно}, если при любых $\alpha$ и $\beta$ из $\vdash\alpha\vee\beta$ следует $\vdash\alpha$ или $\vdash\beta$.

        Решётка \textbf{гёделева}, если $a + b = 1$ влечёт $a = 1$ или $b = 1$.

        Для алгебры Гейтинга $\mathcal{A} = \langle A, (\preceq) \rangle$ определим операцию <<гёделевизации>>:
        $\Gamma(\mathcal{A}) = \langle A\cup\{\omega\}, (\preceq_{\Gamma(\mathcal{A})}) \rangle$, где
        отношение $(\preceq_{\Gamma(\mathcal{A})})$ --- минимальное отношение порядка,
        удовлетворяющее условиям:

        \vspace{-0.5cm}
        \begin{center}\begin{tabular}{cc}
                          \begin{minipage}{9cm}
                              \begin{itemize}
                                  \item $a \preceq_{\Gamma(\mathcal{A})} b$, если $a \preceq_\mathcal{A} b$ и $a,b \notin \{\omega,1\}$;
                                  \item $a \preceq_{\Gamma(\mathcal{A})} \omega$, если $a \ne 1$;
                                  \item $\omega \preceq_{\Gamma(\mathcal{A})} 1$
                              \end{itemize}
                          \end{minipage}
                          &
                          \begin{minipage}{4cm}\begin{center}
                                                   \tikz{
                                                       \filldraw[pattern=north west lines,pattern color=gray] (1,-1) circle (1cm);
                                                       \node[right] at (2.2,-1) (A) {$A \setminus \{1\}$};
                                                       \node[circle,fill,inner sep=2pt, outer sep=0pt,label=right:$1$] at (1,1) (Max) {};
                                                       \node[circle,fill,inner sep=2pt, outer sep=0pt,label=above right:$\omega$] at (1,0) (Omega) {};
                                                       \draw[-stealth,line width=1] (Max) to (Omega);
                                                   }\end{center}
                          \end{minipage}
            \end{tabular}\end{center}

        (Оценка)Определим $\llbracket\cdot\rrbracket_{\Gamma(\mathcal{L})} : \mathcal{F} \rightarrow \Gamma(\mathcal{L})$.
        Положим $\llbracket X \rrbracket_{\Gamma(\mathcal{L})} := \llbracket X \rrbracket_\mathcal{L}$.
        Связки определим естественным образом:
        $\llbracket \alpha\with\beta \rrbracket_{\Gamma(\mathcal{L})} := \llbracket \alpha\rrbracket_{\Gamma(\mathcal{L})}\cdot\llbracket\beta \rrbracket_{\Gamma(\mathcal{L})}$
        и т.п.


        \textbf{(Теорема) }Оценка является алгеброй Гейтинга, согласованной с ИИВ.

        Пусть $\mathcal{A}, \mathcal{B}$ --- алгебры Гейтинга. Тогда $g: \mathcal{A} \rightarrow \mathcal{B}$ --- гомоморфизм,
        если $g(a \star b) = g(a) \star g(b)$, $g(0_\mathcal{A}) = 0_\mathcal{B}$ и $g(1_\mathcal{A}) = 1_\mathcal{B}$.

        Будем говорить, что оценка $\llbracket\cdot\rrbracket_\mathcal{A}$ согласована
        с $\llbracket\cdot\rrbracket_\mathcal{B}$ и гомоморфизмом $g$, если $g(\mathcal{A}) = \mathcal{B}$ и
        $g(\llbracket\alpha\rrbracket_\mathcal{A}) = \llbracket\alpha\rrbracket_\mathcal{B}$.

        $$\mathcal{G}(a) = \left\{\begin{array}{ll} a, & a \ne \omega\\
            1, & a = \omega\end{array}\right.$$

        (Лемма)$\mathcal{G}$ --- гомоморфизм $\Gamma(\mathcal{L})$ и $\mathcal{L}$, причём
        оценка $\llbracket\cdot\rrbracket_{\Gamma(\mathcal{L})}$ согласована с $\mathcal{G}$
        и $\llbracket\cdot\rrbracket_\mathcal{L}$.

        \textbf{Дизъюнктивность ИИВ} Если $\vdash \alpha\vee\beta$, то либо $\vdash\alpha$, либо $\vdash\beta$.

        \begin{proof}Пусть $\vdash\alpha\vee\beta$. Тогда $\llbracket\alpha\vee\beta\rrbracket_{\Gamma(\mathcal{L})} = 1$
            (так как данная оценка согласована с ИИВ). Тогда $\llbracket\alpha\rrbracket_{\Gamma(\mathcal{L})} = 1$ или
            $\llbracket\beta\rrbracket_{\Gamma(\mathcal{L})} = 1$ (так как $\Gamma(\mathcal{L})$ гёделева).

            Пусть $\llbracket\alpha\rrbracket_{\Gamma(\mathcal{L})} = 1$,
            тогда $\mathcal{G}(\llbracket\alpha\rrbracket_{\Gamma(\mathcal{L})}) = \llbracket\alpha\rrbracket_\mathcal{L} = 1$,
            тогда $\vdash\alpha$ (по полноте $\mathcal{L}$).
        \end{proof}

        (в геделевой алгебре легко показывается дизъюнктивность, а из-за согласованности все переносится на псевдобулеву алгебру).

        Решётка $\mathcal{L'} = \langle L', \preceq \rangle$ --- \textbf{подрешётка решётки} $\mathcal{L} = \langle L, \preceq \rangle$,
        если $L' \subseteq L$, $(\preceq') \subseteq (\preceq)$ и
        при $a,b \in L'$ выполнено $a +_{\mathcal{L'}} b = a +_{\mathcal{L}} b$ и $a \cdot_{\mathcal{L'}} b = a \cdot_{\mathcal{L}} b$.

        (Лемма) Существует дистрибутивная подрешётка $\mathcal{L'}$, содержащая
        $a_1, \dots, a_n$, что $|L'| \le 2^{2^n}$.

        \begin{proof}
            Пусть $\mathcal{L'} = \langle\{ \varphi(a_1,\dots,a_n)\ |\ \varphi \text{ составлено из (+) и }(\cdot)\}, (\preceq)\rangle$.
            Заметим, что если $p,q \in L'$, то $p \star_{\mathcal{L}} q \in L'$
            (так как $\varphi_p(\overrightarrow{a})\star\varphi_q(\overrightarrow{a}) = \psi(\overrightarrow{a})$). Также ясно,
            что если $\sup_L\{p,q\} \in L'$ (или $\inf_L\{p,q\} \in L'$), то $p \star_{\mathcal{L}} q = p \star_{\mathcal{L'}} q$.
            Значит, $\mathcal{L'}$ также дистрибутивна. Построим <<ДНФ>>:
            $$\varphi(a_1,\dots,a_n) = \sum_{\text{Кн} \in \text{ДНФ}(\varphi)}\prod_{i \in \text{Кн}}a_i$$


            Всего не больше $2^n$ возможных компонент и $2^{2^n}$ возможных формул $\varphi(\overrightarrow{a})$.
        \end{proof}

        Язык $\mathcal{L} \subseteq A^*$ разрешим, если существует алгоритм, который
        завершает работу при любом $a \in A^*$, поданном на вход,
        причём алгоритм возвращает <<истину>> при $a \in \mathcal{L}$,
        и возвращает <<ложь>> при $a \notin \mathcal{L}$.

        Теория \textbf{неразрешима}, если язык всех истинных (доказуемых) формул неразрешим.

        (Лемма) Если $\not\vdash \alpha$ в ИИВ, то существует $\mathcal{G}$,
        что $\mathcal{G} \not\models \alpha$, причём $|\mathcal{G}| \le 2^{2^{|\alpha|+2}}$.

        \begin{proof}Если $\not\vdash \alpha$, то по
        полноте найдётся алгебра Гейтинга $\mathcal{H}$, что
            $\mathcal{H} \not\models \alpha$.

            Пусть $\varphi_1, \dots, \varphi_n$ --- подформулы $\alpha$.
            Пусть $\mathcal{G}$ --- дистрибутивная подрешётка $\mathcal{H}$,
            построенная по $\llbracket \varphi_1 \rrbracket, \dots, \llbracket \varphi_n \rrbracket$, $0$ и $1$.

            Очевидно, что $\mathcal{G}$ --- алгебра Гейтинга, и можно показать,
            что $\mathcal{G} \not\models \alpha$ (псевдодополнения не обязаны сохраниться).
            Тогда по лемме, $|\mathcal{G}| \le 2^{2^{n+2}}$.
        \end{proof}

        \textbf{(Теорема)}ИИВ разрешимо.

        \begin{proof}По формуле $\alpha$ построим все возможные алгебры Гейтинга $\mathcal{G}$ размера не больше $2^{2^{|\alpha|+2}}$,
        если $\mathcal{G}\models\alpha$, то $\vdash\alpha$.
        \end{proof}

        \section{Категорические силлогизмы. Термины, предикат, субъект, фигуры, модусы (сильные, слабые, неправильные),
            ограничения, контрпримеры на ограничения.
            Исчисление предикатов. Язык исчисления предикатов.
            Метаязык, сокращения записи.
            Вхождения, свободные вхождения, подстановка, свобода для подстановки.
            Теория доказательств для исчисления предикатов, выводимость.
            Доказательства свойств категорических силлогизмов (формулировка свойств сильных и слабых
            силлогизмов на языке исчисления предикатов, их доказательство).}

        \textbf{Силлогизм} --- «подытоживание, подсчёт, умозаключение»

        \textbf{Категорический} --- потому, что речь идёт о категориях (в философском смысле).

        Определяем некоторые стандартные мыслительные блоки, с которыми у образованной аудитории есть навык работы.

        Цель --- сделать неформальный человеческий язык чуть более формальным.

        \newpage

        \textbf{Термины}

        \begin{tabular}{l}
            предикат (больший термин, P)\\
            субъект (меньший термин, S)\\
            средний термин (M).
        \end{tabular}

        \vspace{0.3cm}
        На основании соотношений P и M, а также M и S строим соотношение P и S.
        \vspace{0.3cm}
        Возможные соотношения:

        \begin{tabular}{lll}
            A & Affirmato (общеутвердительное) & Матан есть раздел математики (SaP)\\
            I & affIrmato (частноутвердительное) & Некоторые разделы математики сложны (SiP)\\
            E & nEgo (общеотрицательное) & Никакой человек не знает всю математику\\
            O & negO (частноотрицательное) & Некоторые разделы математики --- не матан
        \end{tabular}

        \begin{tabular}{lcccc}
            & Фигура 1 & Фигура 2 & Фигура 3 & Фигура 4\\
            &
            \tikz{
                \node at (0,1) (M1) { \tiny $M$ };
                \node at (1,1) (P)  { \tiny $P$ };
                \node at (0,0) (S)  { \tiny $S$ };
                \node at (1,0) (M2) { \tiny $M$ };
                \draw (0.85,0.85) -- (0.15,0.85) -- (0.85,0.15) -- (0.15,0.15);
            }
            &
            \tikz{
                \node at (1,1) (M1) { \tiny $M$ };
                \node at (0,1) (P)  { \tiny $P$ };
                \node at (0,0) (S)  { \tiny $S$ };
                \node at (1,0) (M2) { \tiny $M$ };
                \draw (0.15,0.85) -- (0.85,0.85) -- (0.85,0.15) -- (0.15,0.15);
            }
            &
            \tikz{
                \node at (0,1) (M1) { \tiny $M$ };
                \node at (1,1) (P)  { \tiny $P$ };
                \node at (1,0) (S)  { \tiny $S$ };
                \node at (0,0) (M2) { \tiny $M$ };
                \draw (0.85,0.85) -- (0.15,0.85) -- (0.15,0.15) -- (0.85,0.15);
            }
            &
            \tikz{
                \node at (1,1) (M1) { \tiny $M$ };
                \node at (0,1) (P)  { \tiny $P$ };
                \node at (1,0) (S)  { \tiny $S$ };
                \node at (0,0) (M2) { \tiny $M$ };
                \draw (0.15,0.85) -- (0.85,0.85) -- (0.15,0.15) -- (0.85,0.15);
            }

            \\


            Большая посылка: & M—P & P—M & M—P & P—M\\
            Меньшая посылка: & S—M & S—M & M—S & M—S\\
            Заключение: & S—P & S—P & S—P &S—P
        \end{tabular}

        Большинство модусов \textbf{неправильные}.

        Список всех правильных модусов (из них выделяют \textbf{слабые}, выводящие частное соотношение при возможности общего --- указаны курсивом):

            {\small
        \begin{center}\begin{tabular}{llll}
                          Фигура 1 &Фигура 2 &Фигура 3 &Фигура 4\\
                          Barbara &Cesare &Darapti &Bramantip\\
                          Celarent &Camestres &Disamis &Camenes\\
                          Darii &Festino &Datisi &Dimaris\\
                          Ferio &Baroco &Felapton &Fesapo\\
                          \textit{Barbari} &\textit{Cesaro} &Bocardo &Fresison\\
                          \textit{Celaront} &\textit{Camestros} &Ferison &\textit{Camenos}
            \end{tabular}\end{center}}

        Некоторые модусы требуют непустоты M: это все слабые модусы и 19 \textbf{сильных}.(Пример с единорогами - \textbf{ограничение}).

        \textbf{Исчисление предикатов} создано, чтобы расширить формализованную часть языка(включает в себя предикаты).

        \textbf{Язык исчисления предикатов}


        Предметные выражения: метапеременная {\color{blue}$\theta$}. \pause
        \begin{itemize}
            \item Предметные переменные: {\color{blue}$a$}, {\color{blue}$b$}, {\color{blue}$c$}, \dots, метапеременные {\color{blue}$x$}, {\color{blue}$y$}. \pause
            \item Функциональные выражения: {\color{blue}$f(\theta_1,\dots,\theta_n)$}, метапеременные {\color{blue}$f$}, {\color{blue}$g$}, \dots(константы - нульместные функции).\pause
        \end{itemize}\pause
        Логические выражения: метапеременные {\color{blue}$\alpha$}, {\color{blue}$\beta$}, {\color{blue}$\gamma$}, \dots
        \begin{itemize}
            \item Предикатные выражения: {\color{blue}$P(\theta_1,\dots,\theta_n)$}, метапеременная {\color{blue}$P$}.\\\pause
            Имена: {\color{blue}$A$}, {\color{blue}$B$}, {\color{blue}$C$}, \dots (пропозициональные перемнные - нульместные предикаты).\pause
            \item Связки: {\color{blue}$(\varphi\vee\psi)$}, {\color{blue}$(\varphi\with\psi)$}, {\color{blue}$(\varphi\rightarrow\psi)$},
                {\color{blue}$(\neg\varphi)$}.\pause
            \item Кванторы: {\color{blue}$(\forall x.\varphi)$} и {\color{blue}$(\exists x.\varphi)$}(жадные).
        \end{itemize}


        \textbf{Вхождение} подформулы в формулу --- это позиция первого символа этой подформулы в формуле.


        \vspace{-0.4cm}
        $$\text{Вхождения }{\color{blue}x}\text{ в формулу:}\quad (\forall {\color{blue}x}.A({\color{blue}x}) \vee \exists {\color{blue}x}.B({\color{blue}x})) \vee C({\color{blue}x})$$
        \vspace{-0.7cm}

        \textbf{Вхождение $x$ в $\psi$ свободное}, если не находится в области действия никакого квантора по $x$.
        Переменная входит свободно в $\psi$, если имеет хотя бы одно свободное вхождение. $FV(\psi), FV(\Gamma)$ --- множества свободных
        переменных в $\psi$, в $\Gamma$


        Терм $\theta$ \textbf{свободен для подстановки} вместо $x$ в $\psi$ ($\psi[x := \theta]$), если
        ни одно свободное вхождение переменных в $\theta$ не станет связанным после подстановки.

        \begin{center}\begin{tabular}{c|c}
                          Свобода есть & Свободы нет\\\hline
                          $(\forall x.P(y)) [y := z]$ & $(\forall x.P(y)) [y := x]$\\
                          $(\forall y.\forall x.P(x)) [x := y]$ & $(\forall y.\forall x.P(t)) [t := y]$
            \end{tabular}\end{center}

        \textbf{Теория доказательств}

        Рассмотрим язык исчисления предикатов. Возьмём все схемы аксиом классического исчисления высказываний и добавим ещё две схемы аксиом
        (здесь везде $\theta$ свободен для подстановки вместо $x$ в $\varphi$):

        \begin{tabular}{ll}
            11. & $(\forall x.\varphi) \rightarrow \varphi[x:=\theta]$\\
            12. & $\varphi[x:=\theta] \rightarrow \exists x.\varphi$
        \end{tabular}

        Добавим ещё два правила вывода (здесь везде $x$ не входит свободно в $\varphi$):
        $$\infer[\text{Правило для }\forall]{\varphi\rightarrow\forall x.\psi}{\varphi\rightarrow\psi}$$
        $$\infer[\text{Правило для }\exists]{(\exists x.\psi)\rightarrow\varphi}{\psi\rightarrow\varphi}$$

        \textbf{Доказуемость, выводимость, полнота, корректность} --- аналогично исчислению высказываний.

        \textbf{Доказательства свойств категорических силлогизмов }

        \section{Теория моделей исчисления предикатов (предметное множество, оценка).
        Функции (предикаты) и функциональные (предикатные) символы.
        Общезначимость, следование.
        Теорема о дедукции в исчислении предикатов. Отличия от исчисления высказываний.
        Лемма о перестановке подстановки и оценки. Теорема о корректности исчисления предикатов.}

        \textbf{Предметные выражения}: метапеременная {\color{blue}$\theta$}. \pause
        \begin{itemize}
            \item Предметные переменные: {\color{blue}$a$}, {\color{blue}$b$}, {\color{blue}$c$}, \dots, метапеременные {\color{blue}$x$}, {\color{blue}$y$}. \pause
            \item Функциональные выражения: {\color{blue}$f(\theta_1,\dots,\theta_n)$}, метапеременные {\color{blue}$f$}, {\color{blue}$g$}, \dots(константы - нульместные функции).\pause
        \end{itemize}\pause
        \textbf{Логические выражения}: метапеременные {\color{blue}$\alpha$}, {\color{blue}$\beta$}, {\color{blue}$\gamma$}, \dots
        \begin{itemize}
            \item Предикатные выражения: {\color{blue}$P(\theta_1,\dots,\theta_n)$}, метапеременная {\color{blue}$P$}.\\\pause
            Имена: {\color{blue}$A$}, {\color{blue}$B$}, {\color{blue}$C$}, \dots (пропозициональные перемнные - нульместные предикаты).\pause
            \item Связки: {\color{blue}$(\varphi\vee\psi)$}, {\color{blue}$(\varphi\with\psi)$}, {\color{blue}$(\varphi\rightarrow\psi)$},
                {\color{blue}$(\neg\varphi)$}.\pause
            \item Кванторы: {\color{blue}$(\forall x.\varphi)$} и {\color{blue}$(\exists x.\varphi)$}(жадные).
        \end{itemize}

        \begin{dfn}\textbf{Оценка} --- упорядоченная четвёрка $\langle D, F, P, E \rangle$, где:\pause

        \begin{enumerate}
            \item $D \neq \varnothing$ --- предметное множество;\pause
            \item $F$ --- оценка для функциональных символов; пусть $f_n$ --- $n$-местный функциональный символ:
            $$F_{f_n}: D^n \rightarrow D$$\pause

            \item $P$ --- оценка для предикатных символов; пусть $T_n$ --- $n$-местный предикатный символ:
            $$P_{T_n}: D^n \rightarrow V\quad\quad\quad\pause V = \{\text{И}, \text{Л}\}$$\pause

            \item $E$ --- оценка для предметных переменных.
            $$E(x) \in D$$
        \end{enumerate}\end{dfn}

    \textbf{Правила оценок}

    \begin{itemize}
        \item Правила для связок $\vee$, $\with$, $\neg$, $\rightarrow$ остаются прежние;\pause
        \item $\llbracket f_n (\theta_1, \theta_2, \dots, \theta_n) \rrbracket = F_{f_n} (\llbracket\theta_1\rrbracket,
        \llbracket\theta_2\rrbracket, \dots, \llbracket\theta_n\rrbracket)$\pause
        \item $\llbracket P_n (\theta_1, \theta_2, \dots, \theta_n) \rrbracket = P_{T_n} (\llbracket\theta_1\rrbracket,
        \llbracket\theta_2\rrbracket, \dots, \llbracket\theta_n\rrbracket)$\pause
        \item $$\llbracket \forall x.\phi \rrbracket = \left\{\begin{array}{ll}
                                                                  \text{И}, & \text{если } \llbracket\phi\rrbracket^{x := t} = \text{И}\text{ при всех } t \in D\\
                                                                  \text{Л}, & \text{если найдётся } t \in D, \text{ что } \llbracket\phi\rrbracket^{x := t} = \text{Л}
        \end{array}\right.$$\pause
        \item $$\llbracket \exists x.\phi \rrbracket = \left\{\begin{array}{ll}
                                                                  \text{И}, & \text{если найдётся } t \in D, \text{ что } \llbracket\phi\rrbracket^{x := t} = \text{И}\\
                                                                  \text{Л}, & \text{если } \llbracket\phi\rrbracket^{x := t} = \text{Л}\text{ при всех } t \in D
        \end{array}\right.$$
    \end{itemize}

    Формула исчисления предикатов общезначима, если истинна при любой оценке:
    $$\models\phi$$

    \textbf{Следование}: $\gamma_1,\gamma_2,\dots,\gamma_n\models\alpha$, если $\alpha$ выполнено всегда, когда выполнено $\gamma_1,\gamma_2,\dots,\gamma_n$.\\

    \textbf{Теорема о дедукции}
    Если $\Gamma\vdash\alpha\rightarrow\beta$, то $\Gamma,\alpha\vdash\beta$.
    Если $\Gamma,\alpha\vdash\beta$ и в доказательстве не применяются правила для кванторов
    по свободным переменным из $\alpha$, то $\Gamma\vdash\alpha\rightarrow\beta$.

    \begin{proof}$(1)$ --- как в КИВ \pause $(2)$ --- та же схема, два новых случая. \pause

        Перестроим: $\delta_1, \delta_2, \dots, \delta_n \equiv \beta$ в $\alpha\rightarrow\delta_1, \alpha\rightarrow\delta_2, \dots, \alpha\rightarrow\delta_n$.

        Дополним: обоснуем $\alpha\rightarrow\delta_n$, если предыдущие уже обоснованы.\pause

        Два новых похожих случая: правила для $\forall$ и $\exists$. Рассмотрим $\forall$.

        Доказываем $(n)\ \ \alpha\rightarrow\psi\rightarrow\forall x.\varphi$ (правило для $\forall$), значит, доказано
        $(k)\ \ \alpha\rightarrow\psi\rightarrow\varphi$. \pause\\

        \begin{tabular}{lll}
            $(n-0.9) \dots (n-0.8)$ & $(\alpha\rightarrow\psi\rightarrow\varphi)\rightarrow(\alpha\with\psi)\rightarrow\varphi$ & Т. о полноте КИВ\\
            $(n-0.6)$ & $(\alpha\with\psi)\rightarrow\varphi$ & M.P. $k$,$n-0.8$\\\pause
            $(n-0.4)$ & $(\alpha\with\psi)\rightarrow\forall x.\varphi$ & Правило для $\forall$, $n-0.6$\\\pause
            $(n-0.3) \dots (n-0.2)$ & $((\alpha\with\psi)\rightarrow\forall x.\varphi)\rightarrow(\alpha\rightarrow\psi\rightarrow\forall x.\varphi)$ & Т. о полноте КИВ\\
            $(n)$ & $\alpha\rightarrow\psi\rightarrow\forall x.\varphi$ & M.P. $n-0.4$, $n-0.2$
        \end{tabular}

    \end{proof}

    (\textbf{Лемма о перестановке подстановки и оценки})
    Если $\theta$ свободен для подстановки
    вместо $x$ в $\varphi$, то $\llbracket\varphi\rrbracket^{x := \llbracket\theta\rrbracket} = \llbracket\varphi[x := \theta]\rrbracket$.

    \begin{proof}[Доказательство (индукция по структуре $\varphi$)](заметено под ковер все, кроме кванторов, потому что оно очевидно)
  \begin{itemize}
      \item База: $\varphi$ не имеет кванторов. Очевидно.
      \item Переход: пусть справедливо для $\psi$. Покажем для $\varphi = \forall y.\psi$.
      \begin{itemize}
          \item $x=y$ либо $x \notin FV(\psi)$. Тогда:
          $\llbracket\forall y.\psi\rrbracket^{x := \llbracket\theta\rrbracket} = \llbracket\forall y.\psi\rrbracket = \llbracket(\forall y.\psi)[x := \theta]\rrbracket$(оценка игнорируется)

          \item $x \ne y$. Тогда: $\llbracket\forall y.\psi\rrbracket^{x := \llbracket\theta\rrbracket} =
          \llbracket\psi\rrbracket^{y \in D; x := \llbracket\theta\rrbracket} = \dots$(если квантор оценивается в истинну, то в можем взять любую оценку, если в ложь, то возьмем этот контрпример).
          %\vspace{-0.3cm}

              {\color{olive}Свобода для подстановки: $y\notin\theta$.}
          \vspace{-0.3cm}
          $$\dots = \llbracket\psi\rrbracket^{x := \llbracket\theta\rrbracket; y \in D} = \dots$$
          \vspace{-0.8cm}

          {\color{olive}Индукционное предположение.}
          \vspace{-0.3cm}

          $$\dots = \llbracket\psi[x := \theta]\rrbracket^{y \in D} =
          \llbracket\forall y.(\psi[x := \theta])\rrbracket = \dots$$
          \vspace{-0.5cm}

          (если оценка совпадает на всех y по предположению, то совпадет и на кванторе).

              {\color{olive}Но $\forall y.(\psi[x := \theta]) \equiv (\forall y.\psi) [x := \theta]$ (как текст). Отсюда:}
          \vspace{-0.3cm}

          $$\dots = \llbracket(\forall y.\psi)[ x := \theta]\rrbracket$$
          \vspace{-0.5cm}
      \end{itemize}
  \end{itemize} \end{proof}

    \textbf{Корректность} Если $\Gamma \vdash \alpha$ и в доказательстве не используются кванторы по свободным переменным из $FV(\Gamma)$, то $\Gamma \models \alpha$.

    \begin{proof}Фиксируем $D, F, P$. Индукция по длине доказательства $\alpha$: при любом $E$ выполнено $\Gamma\models\alpha$
        при длине доказательства $n$, покажем для $n+1$.
        \begin{itemize}
            \item Схемы аксиом (1)..(10), правило M.P.: аналогично И.В.
            \item Схемы (11) и (12), например, схема $(\forall x.\varphi) \rightarrow \varphi [x := \theta]$: \vspace{-0.6cm}

            $$\llbracket (\forall x.\varphi) \rightarrow \varphi [x := \theta]\rrbracket = \llbracket ((\forall x.\varphi) \rightarrow \varphi) [x := \theta] \rrbracket =
            \llbracket (\forall x.\varphi) \rightarrow \varphi \rrbracket ^ { x := \llbracket\theta\rrbracket } = \text{И}$$(по лемме)

            \item Правила для кванторов: например, введение $\forall$:

            Пусть $\llbracket \psi \rightarrow \varphi \rrbracket = \text{И}$. Причём $x \notin FV(\Gamma)$ и $x \notin FV(\psi)$. То есть,
            при любом $\mathcal{x}$ выполнено $\llbracket \psi \rightarrow \varphi \rrbracket^{x := \mathcal{x}} = \text{И}$. Тогда
            $\llbracket \psi \rightarrow (\forall x.\varphi) \rrbracket = \text{И}$.(по таблице истинности)

        \end{itemize}
    \end{proof}
    \section{Непротиворечивые множества формул (с кванторами и бескванторные).
    Пополнение множества формул.
    Существование моделей у непротиворечивых множеств формул в бескванторном исчислении предикатов.}

    $\Gamma$ --- \textbf{непротиворечивое множество формул},
    если $\Gamma\not\vdash\alpha\with\neg\alpha$ для любого $\alpha$

    $\Gamma$ --- \textbf{полное непротиворечивое множество замкнутых бескванторных формул},
    если:
    \begin{enumerate}\item $\Gamma$ содержит только замкнутые бескванторные формулы;
        \item если $\alpha$ --- некоторая замкнутая бескванторная формула, то либо $\alpha\in\Gamma$, либо $\neg\alpha\in\Gamma$.
    \end{enumerate}

    $\Gamma$ --- \textbf{полное непротиворечивое множество замкнутых формул}, если:
    \begin{enumerate}\item $\Gamma$ содержит только замкнутые формулы;
        \item если $\alpha$ --- некоторая замкнутая формула, то либо $\alpha \in \Gamma$, либо $\neg\alpha \in \Gamma$.
    \end{enumerate}

    \textbf{(Пополнение непротиворечивого множества формул)}
    Пусть $\Gamma$ --- непротиворечивое множество замкнутых (бескванторных) формул. Тогда, какова бы ни была
    замкнутая (бескванторная) формула $\varphi$, хотя бы $\Gamma \cup \{\varphi\}$ или $\Gamma \cup \{\neg\varphi\}$ ---
    непротиворечиво.

    \begin{proof}
        Пусть это не так и найдутся такие $\Gamma$, $\varphi$ и $\alpha$, что
        $$\begin{array}{rl}\Gamma,\varphi & \vdash \alpha\with\neg\alpha\\
            \Gamma,\neg\varphi & \vdash \alpha \with\neg\alpha\end{array}$$\pause\vspace{-0.3cm}

        Тогда по лемме об исключении гипотезы
        $$\Gamma\vdash \alpha\with\neg\alpha$$\pause\vspace{-0.4cm}

        То есть $\Gamma$ не является непротиворечивым. Противоречие.
    \end{proof}

    \textbf{(Дополнение непротиворечивого множества формул)}
    Пусть $\Gamma$ --- непротиворечивое множество замкнутых (бескванторных) формул. Тогда
    найдётся полное непротиворечивое множество замкнутых (бескванторных) формул $\Delta$, что
    $\Gamma \subseteq \Delta$

    \begin{proof}
        \begin{enumerate}
            \item Занумеруем все формулы (их счётное количество): $\varphi_1, \varphi_2, \dots$\pause
            \item Построим семейство множеств $\{\Gamma_i\}$:
            \begin{tabular}{cc}
                $\Gamma_0 = \Gamma$  &
                \begin{minipage}{12cm}
                    $$\Gamma_{i+1} = \left\{\begin{array}{ll}\Gamma_i \cup \{\varphi_i\},& \mbox{ если } \Gamma_i \cup \{\varphi_i\} \mbox{ непротиворечиво}\\
                        \Gamma_i \cup \{\neg\varphi_i\},& \mbox{ иначе}\end{array}\right.$$
                \end{minipage}\end{tabular}\pause
            \item Итоговое множество $$\Delta = \bigcup_i \Gamma_i$$\pause\vspace{-0.2cm}
            \item Непротиворечивость $\Delta$ не следует из индукции --- индукция гарантирует непротиворечивость
            только $\Gamma_i$ при натуральном (т.е. \emph{конечном}) $i$, потому\dots
        \end{enumerate}

        \begin{enumerate}\setcounter{enumi}{3}
            \item $\Delta$ непротиворечиво:
            \begin{enumerate}
                \item Пусть $\Delta$ противоречиво, то есть $$\Delta \vdash \alpha\with\neg\alpha$$\pause\vspace{-0.2cm}
                \item Доказательство конечной длины и использует конечное количество гипотез $\{\delta_1, \delta_2, \dots, \delta_n\} \subset \Delta$,
                то есть $$\delta_1, \delta_2, \dots, \delta_n \vdash \alpha\with\neg\alpha$$\pause\vspace{-0.2cm}
                \item Пусть $\delta_i \in \Gamma_{d_i}$, тогда $$\Gamma_{d_1}\cup \Gamma_{d_2}\cup \dots\cup \Gamma_{d_n} \vdash \alpha\with\neg\alpha$$\pause\vspace{-0.2cm}
                \item Но $\Gamma_{d_1} \cup \Gamma_{d_2} \cup \dots \cup \Gamma_{d_n} = \Gamma_{\max(d_1,d_2,\dots,d_n)}$,
                которое непротиворечиво, и потому $$\Gamma_{d_1}\cup \Gamma_{d_2}\cup \dots\cup \Gamma_{d_n} \not\vdash \alpha\with\neg\alpha$$
            \end{enumerate}\pause\vspace{-0.5cm}
        \end{enumerate}
    \end{proof}

    \textbf{Моделью} для множества формул $F$ назовём такую модель $\mathcal{M}$, что
    при всяком $\varphi \in F$ выполнено $\llbracket\varphi\rrbracket_\mathcal{M} = \text{И}$.\pause

    Альтернативное обозначение: $\mathcal{M}\models\varphi$.\\

    Пусть $M$ --- полное непротиворечивое множество замкнутых бескванторных формул. Тогда
    \textbf{модель $\mathcal{M}$} задаётся так:\pause
    \begin{enumerate}
        \item $D$ --- множество всевозможных предметных выражений без предметных переменных. Оно непусто (язык обычно содержит много нуль-местных функций),
        но если пусто --- добавим туда какое-нибудь одно значение, пусть ``z''.
        \pause
        \item $\llbracket f(\theta_1,\dots,\theta_n) \rrbracket = \mbox{``f(''} \mdoubleplus \llbracket\theta_1\rrbracket \mdoubleplus \mbox{ ``,'' }
        \mdoubleplus \dots \mdoubleplus \mbox{ ``,'' } \mdoubleplus \llbracket\theta_n\rrbracket \mdoubleplus \mbox {``)'' } $\pause
        \item $\llbracket P(\theta_1,\dots,\theta_n)\rrbracket = \left\{
                                                                     \begin{array}{ll} \mbox{И}, &\mbox{ если } P(\theta_1,\dots,\theta_n) \in M\\
                                                                         \mbox{Л}, &\mbox{ иначе}\end{array}\right.$\pause
        \item Так как $D \ne \varnothing$, то найдётся $z \in D$. Тогда $\llbracket x \rrbracket = z$. Это ничему не помешает, так как формулы замкнуты.
    \end{enumerate}

    \textbf{(Лемма)}Пусть $\varphi$ --- бескванторная формула, тогда $\mathcal{M}\models\varphi$ тогда и только тогда, когда $\varphi\in M$.


    \begin{proof}[Доказательство (индукция по длине формулы $\varphi$)]
  \begin{enumerate}
      \item База. $\varphi$ --- предикат. Требуемое очевидно по определению $\mathcal{M}$.\pause
      \item Переход. Пусть $\varphi = \alpha\star\beta$ (или $\varphi=\neg\alpha$), причём $\mathcal{M}\models\alpha$ ($\mathcal{M}\models\beta$)
      тогда и только тогда, когда $\alpha\in M$ ($\beta\in M$).\pause

      Тогда покажем требуемое для каждой связки в отдельности. А именно, для каждой связки покажем два утверждения:
      \begin{enumerate}
          \item если $\mathcal{M}\models\alpha\star\beta$, то $\alpha\star\beta \in M$.

          \item если $\mathcal{M}\not\models\alpha\star\beta$, то $\alpha\star\beta \notin M$.
      \end{enumerate}
  \end{enumerate}
    \end{proof}

    \textbf{(Утверждение для связок)}
    Если $\varphi = \alpha\to\beta$ и для любой формулы $\zeta$, более короткой, чем $\varphi$, выполнено
    $\mathcal{M}\models\zeta$ тогда и только тогда, когда $\zeta\in M$, тогда:
    \begin{enumerate}
        \item если $\mathcal{M}\models\alpha\to\beta$, то $\alpha\to\beta\in M$;
        \item если $\mathcal{M}\not\models\alpha\to\beta$, то $\alpha\to\beta\notin M$.
    \end{enumerate}

    \begin{proof}[Доказательство (разбором случаев)]
  \begin{enumerate}\pause
      \item $\mathcal{M}\models\alpha\to\beta$: $\llbracket\alpha\rrbracket = \text{Л}$. \pause
      Тогда по предположению $\alpha\notin M$, потому по полноте
      $\neg\alpha\in M$. \pause И, поскольку в ИВ $\neg\alpha\vdash\alpha\to\beta$, то $M \vdash \alpha\to\beta$. \pause
      Значит, $\alpha\to\beta \in M$, иначе по полноте $\neg(\alpha\to\beta) \in M$, что делает $M$ противоречивым.\pause
      \item $\mathcal{M}\models\alpha\to\beta$: $\llbracket\alpha\rrbracket = \text{И}$ и $\llbracket\beta\rrbracket = \text{И}$. Рассуждая аналогично,
      используя $\alpha,\beta\vdash\alpha\to\beta$, приходим к $\alpha\to\beta \in M$.\pause
      \item $\mathcal{M}\not\models\alpha\to\beta$. Тогда $\llbracket\alpha\rrbracket=\text{И}$,
      $\llbracket\beta\rrbracket=\text{Л}$, \pause то есть $\alpha\in M$ и $\neg\beta\in M$. \pause
      Также, $\alpha,\neg\beta\vdash\neg(\alpha\to\beta)$, отсюда $M\vdash\neg(\alpha\to\beta)$. \pause
      Предположим, что $\alpha\to\beta\in M$, то $M\vdash\alpha\to\beta$ --- отсюда
      $\alpha\to\beta\notin M$.
  \end{enumerate}
    \end{proof}

    \textbf{(Сущестование модели)}Любое непротиворечивое множество замкнутых бескванторных формул имеет модель.

    \begin{proof}
        Пусть $M$ --- непротиворечивое множество замкнутых бескванторных формул.\pause

        По теореме о пополнении существует $M'$ --- полное непротиворечивое множество замкнутых бескванторных формул,
        что $M \subseteq M'$.\pause

        По лемме $M'$ имеет модель, эта модель подойдёт для $M$.
    \end{proof}

    \section{Поверхностные кванторы (предварённая форма). Эквивалентность формул формулам с поверхностными кванторами
        (формулировка теоремы).
        Сколемизация.
        Теорема Гёделя о полноте исчисления предикатов.
        Полнота исчисления предикатов.}

    Формула $\varphi$ имеет \textbf{поверхностные кванторы} (находится в предварённой форме), если
    соответствует грамматике
    $$\varphi ::= \forall x.\varphi\ |\ \exists x.\varphi\ |\ \tau$$
    где $\tau$ --- формула без кванторов

    \textbf{(Теорема)}Для любой замкнутой формулы $\psi$ найдётся такая формула $\varphi$ с поверхностными кванторами,
    что $\vdash \psi\to\varphi$ и $\vdash\varphi\to\psi$

    \begin{proof}Индукция по структуре, применение теорем о перемещении кванторов.
    \end{proof}

    \textbf{Построение $M^*$}
    \begin{itemize}
        \item Пусть $M$ --- полное непротиворечивое множество замкнутых формул с поверхностными кванторами (очевидно, счётное). \pause
        Построим семейство непротиворечивых множеств замкнутых формул $M_k$.\pause
        \item Пусть $d^k_i$ --- семейство \emph{свежих} констант, в $M$ не встречающихся.\pause
        \item Индуктивно построим $M_k$:
        \begin{itemize}
            \item База: $M_0 = M$\pause
            \item Переход: положим $M_{k+1} = M_k \cup S$, где множество $S$ получается перебором всех формул $\varphi_i \in M_k$.\pause
            \begin{enumerate}
                \item $\varphi_i$ --- формула без кванторов, пропустим;\pause
                \item $\varphi_i = \forall x.\psi$ --- добавим к $S$ все формулы вида $\psi [x := \theta]$, где
                $\theta$ --- всевозможные замкнутые термы, использующие символы из $M_k$;\pause
                \item $\varphi_i = \exists x.\psi$ --- добавим к $S$ формулу $\psi [x := d^{k+1}_i]$, где $d^{k+1}_i$ --- некоторая
                свежая, ранее не использовавшаяся в $M_k$, константа.\pause
            \end{enumerate}
        \end{itemize}
    \end{itemize}

    (\textbf{Теорема})Если $M$ непротиворечиво, то каждое множество из $M_k$ --- непротиворечиво

    \begin{proof}Доказательство по индукции, база очевидна ($M_0 = M$). \pause
        Переход: \begin{itemize}
                     \item пусть $M_k$ непротиворечиво, но $M_{k+1}$ --- противоречиво: $M_k, M_{k+1}\setminus M_k \vdash A\with\neg A$. \pause
                     \item Тогда (т.к. доказательство конечной длины):
                     $M_k, \gamma_1, \gamma_2, \dots,\gamma_n\vdash A\with\neg A$
                     , где $\gamma_i \in M_{k+1}\setminus M_k$. \pause
                     \item По теореме о дедукции: $M_k\vdash \gamma_1\to\gamma_2\to\dots\to\gamma_n\to A\with\neg A$. \pause
                     \item Научимся выкидывать первую посылку: $M_k\vdash \gamma_2\to\dots\to\gamma_n\to A\with\neg A$. \pause
                     \item И по индукции придём к противоречию: $M_k \vdash A\with\neg A$.
        \end{itemize}

    \end{proof}

    \textbf{Устранение посылки} Если $M_k\vdash\gamma\to W$ и $\gamma\in M_{k+1}\setminus M_k$, то $M_k\vdash W$.

    \begin{proof}
        Покажем, как дополнить доказательство до $M_k\vdash W$, в зависимости от происхождения $\gamma$:
        \pause

        \begin{itemize}
            \item Случай $\forall x.\varphi$: $\gamma = \varphi[x:=\theta]$.
            \pause
            Допишем в конец доказательства:

            \begin{tabular}{ll}
                $\forall x.\varphi$ & (гипотеза)\\\pause
                $(\forall x.\varphi)\to(\varphi[x:=\theta])$ & (сх. акс. 11)\\\pause
                $\gamma$  & (M.P.) \\\pause
                $W$ & (M.P.)
            \end{tabular}


            \item Случай $\exists x.\varphi$


            \begin{itemize}
                \item $\gamma = \varphi[x := d^{k+1}_i]$\pause


                \item Перестроим доказательство $M_k\vdash \gamma\to W$:
                заменим во всём доказательстве $d^{k+1}_i$ на $y$.
                Коллизий нет: под квантором $d^{k+1}_i$ не стоит, переменной не является. \pause
                \item Получим доказательство $M_k\vdash \gamma[d^{k+1}_i := y]\to W$ и дополним его:

                \begin{tabular}{ll}
                    $\varphi[x := y]\to W$ & $\varphi[x := d^{k+1}_i][d^{k+1}_i := y]$\\\pause
                    $(\exists y.\varphi[x:=y])\to W$ & $y$ не входит в $W$ \\\pause
                    $(\exists x.\varphi)\to(\exists y.\varphi[x:=y])$ & доказуемо (упражнение)\\\pause
                    \dots \\
                    $(\exists x.\varphi)\to W$ & доказуемо как $(\alpha\to\beta)\to(\beta\to\gamma)\vdash\alpha\to\gamma$ \\\pause
                    $\exists x.\varphi$ & гипотеза\\\pause
                    $W$
                \end{tabular}
            \end{itemize}
        \end{itemize}


    \end{proof}

    \textbf{$M^*$} = $\bigcup_k M_k$

    (Лемма) $M^*$ непротиворечиво

    \begin{proof} От противного: доказательство противоречия конечной длины, гипотезы лежат в максимальном $M_k$, тогда $M_k$ противоречив.\pause
    \end{proof}

    (Определение) $M^\text{Б}$ --- множество всех бескванторных формул из $M^*$.\pause
    По непротиворечивому множеству $M$ можем построить $M^\text{Б}$ и для него построить модель $\mathcal{M}$.
    Покажем, что эта модель годится для $M^*$ (и для $M$, так как $M \subset M^*$).

    \textbf{(Лемма)} $\mathcal{M}$ есть модель для $M^*$.

    \begin{proof}
        Покажем, что при $\varphi\in M^*$ выполнено $\mathcal{M}\models\varphi$. Докажем индукцией по количеству кванторов в $\varphi$.\pause
        \begin{itemize}
            \item База: $\varphi$ без кванторов. Тогда $\varphi\in M^\text{Б}$, отсюда $\mathcal{M}\models\varphi$ по построению $\mathcal{M}$.\pause
            \item Переход: пусть утверждение выполнено для всех формул с $n$ кванторами. Покажем, что это выполнено и для $n+1$ кванторов.\pause
            \begin{itemize}
                \item Рассмотрим $\varphi = \exists x.\psi$, случай квантор всеобщности --- аналогично.\pause

                \item Раз $\exists x.\psi \in M^*$, то существует $k$, что $\exists x.\psi \in M_k$.\pause
                \item Значит, $\psi[x := d^{k+1}_i] \in M_{k+1}$. \pause
                \item По индукционному предположению, $\mathcal{M}\models\psi[x := d^{k+1}_i]$ --- в формуле $n$ кванторов.\pause
                \item Но тогда $\llbracket \psi \rrbracket^{x := \llbracket d^{k+1}_i\rrbracket} = \text{И}$.\pause
                \item Отсюда $\mathcal{M}\models\exists x.\psi$.
            \end{itemize}
        \end{itemize}
    \end{proof}

    \textbf{Теорема Гёделя о полноте исчисления предикатов} Если $M$ --- замкнутое непротиворечивое множество формул, то оно имеет модель.

    \begin{proof}
        \begin{itemize}
            \item Построим по $M$ множество формул с поверхностными кванторами $M'$.\pause
            \item По $M'$ построим непротиворечивое множество замкнутых бескванторных формул $M^\text{Б}$ ($M^\text{Б}\subseteq M^*$, теорема о непротиворечивости $M^*$).\pause
            \item Дополним его до полного, построим для него модель $\mathcal{M}$ (теорема о существовании модели).\pause
            \item $\mathcal{M}$ будет моделью и для $M'$ ($M'\subseteq M^*$, лемма о модели для $M^*$), и, очевидно, для $M$.
        \end{itemize}
    \end{proof}

    \textbf{Полнота исчисления предикатов} Исчисление предикатов полно.

    \begin{proof}
        \begin{itemize}
            \item Пусть это не так, и существует формула $\varphi$, что $\models\varphi$, но $\not\vdash\varphi$.\pause
            \item Тогда рассмотрим $M = \{\neg\varphi\}$. \pause
            \item $M$ непротиворечиво: если $\neg\varphi \vdash A\with\neg A$, то $\vdash \varphi$ (упражнение).\pause
            \item Значит, у $M$ есть модель $\mathcal{M}$, и $\mathcal{M}\models\neg\varphi$. \pause
            \item Значит, $\llbracket \neg\varphi \rrbracket = \text{И}$, поэтому $\llbracket \varphi \rrbracket = \text{Л}$,
            поэтому $\not\models\varphi$. Противоречие.
        \end{itemize}
    \end{proof}


    (Лемма) Если у множества формул $M$ есть модель $\mathcal{M}$, оно непротиворечиво.

    \begin{proof}Пусть противоречиво: $M\vdash A\with\neg A$, в доказательстве использованы гипотезы
        $\delta_1, \delta_2,\dots,\delta_n$. \pause Тогда $\vdash \delta_1\to\delta_2\to\dots\to\delta_n\to A\with\neg A$,
        то есть $\llbracket\delta_1\to\delta_2\to\dots\to\delta_n\to A\with\neg A\rrbracket = \text{И}$ (корректность).
        \pause Поскольку все $\llbracket \delta_i \rrbracket_\mathcal{M} = \text{И}$, то
        и $\llbracket A\with\neg A\rrbracket_\mathcal{M} = \text{И}$ (анализ таблицы истинности импликации). \pause
        Однако $\llbracket A \with\neg A \rrbracket = \text{Л}$. Противоречие.\end{proof}

    (Следствие) \textbf{Исчисление предикатов непротиворечиво}


\section{Машина Тьюринга. Разрешимость теории, примеры. Задача об останове, её неразрешимость.
Неразрешимость исчисления предикатов.}


\textbf{Машина Тьюринга:}

\begin{enumerate}
    \item Внешний алфавит $q_1, \dots, q_n$, выделенный символ-заполнитель $q_\varepsilon$
    \item Внутренний алфавит (состояний) $s_1, \dots, s_k$; $s_s$ --- начальное, $s_f$ --- допускающее, $s_r$ --- отвергающее.
    \item Таблица переходов $\langle k, s \rangle \Rightarrow \langle k', s', \leftrightarrow \rangle$
\end{enumerate}

Состояние машины Тьюринга:

\begin{enumerate}
    \item Бесконечная лента с символом-заполнителем $q_\varepsilon$, текст конечной длины.
    \item Головка над определённым символом.
    \item Символ состояния (состояние в узком смысле) --- символ внутреннего алфавита.
\end{enumerate}

Язык --- множество строк

Язык $L$ \textbf{разрешим}, если существует машина Тьюринга, которая для любого слова $w$ переходит в допускающее состояние, если $w \in L$,
и в отвергающее, если $w \notin L$.

Рассмотрим все возможные описания машин Тьюринга. Составим упорядоченные пары: описание машины Тьюринга и входная строка.
Из них выделим язык останавливающихся на данном входе машин Тьюринга.

\textbf{(Теорема)} Язык всех останавливающихся машин Тьюринга неразрешим.

\begin{proof}От противного. Пусть $S(x,y)$ --- машина Тьюринга, определяющая, остановится ли машина $x$, примененная к строке $y$.\pause
    \begin{center}W(x) = if (S(x,x)) \{ while (true); return 0; \} else \{ return 1; \}\end{center}\pause
    Что вернёт $S(code(W),code(W))$?
    \end{proof}

Кодируем состояние:

\begin{enumerate}
    \item внешний алфавит: $n$ 0-местных функциональных символов $q_1, \dots, q_n$; $q_\varepsilon$ --- символ-заполнитель.
    \item список: $\varepsilon$ и $c(l,s)$; <<abc>> представим как $c(q_a,c(q_b,c(q_c,\varepsilon)))$.
    \item положение головки: <<$ab\underline{p}q$>> как $(c(q_b,c(q_a,\varepsilon)), c(q_p,c(q_q,\varepsilon)))$.
    \item внутренний алфавит: $k$ 0-местных функциональных символов $s_1, \dots, s_k$. Из них выделенные $s_s$ --- начальное и
    $s_f$ --- допускающее состояние.
\end{enumerate}

Предикатный символ $F_{x,y}(w_l,w_r,s)$: если у машины $x$ с начальной строкой $y$ состояние $s$ достижимо на строке $rev(w_l) @ w_r$. 
Будем накладывать условия: семейство формул $C_m$. 
Очевидно, начальное состояние достижимо:
$$C_0 := F_{x,y}(\varepsilon,y,s_s)$$.

\begin{enumerate}
    \item Занумеруем переходы.\pause
    \item Закодируем переход $m$: $$\langle k, s \rangle \Rightarrow \langle k', s', \rightarrow \rangle, \text{ в случае } q_k \ne q_\varepsilon$$
    $C_m = \forall w_l.\forall w_r.F_{x,y}(w_l,c(q_k,w_r),s_s) \rightarrow F_{x,y}(c(q_{k'},w_l),w_r,s_{s'})$\\
    (здесь требуется, чтобы под головкой находился непустой символ $q_k$, потому мы обязательно требуем, чтобы лента была
    непуста)\pause
    
    \item Переход посложнее:
    $$\langle k, s \rangle \Rightarrow \langle k', s', \leftarrow \rangle, \text{ в случае } q_k \ne q_\varepsilon$$
    $C_m = \forall w_l.\forall w_r.\forall t.F_{x,y}(c(t,w_l),c(q_k,w_r),s_s) \rightarrow F_{x,y}(w_l,c(t,c(q_{k'},w_r)),s_{s'}) \with
    \forall w_l.\forall w_r.F_{x,y}(\varepsilon,c(q_k,w_r),s_s) \rightarrow F_{x,y}(\varepsilon,c(q_\varepsilon,c(q_{k'},w_r)),s_{s'})$\\
    \pause
    \item и т.п.
    \end{enumerate}

\textbf{(Теорема)} Состояние $s$ со строкой $rev(w_l)@w_r$ достижимо тогда и только тогда, когда
$C \vdash F_{x,y}(w_l,w_r,s)$

\begin{proof}
    $(\Leftarrow)$ Рассмотрим модель: предикат $F_{x,y}(w_l,w_r,s)$ положим истинным, если состояние достижимо. \pause
    Это --- модель для $C$ (по построению $C_m$). \pause
    Значит, доказуемость влечёт истинность (по корректности). \pause
    
    $(\Rightarrow)$ 

    Докажем, что если состояние $s$ со строкой $\text{rev}(w_l)@w_r$ достижимо, то 
\[
C \vdash F_{x,y}(w_l, w_r, s).
\]

Проведём \textbf{индукцию по длине лога исполнения} $k$ --- количеству шагов от начального состояния до состояния $s$.

\noindent \textbf{База индукции:} $k = 0$.

Состояние $s$ --- начальное. В этом случае $s = s_s$, $w_l = \varepsilon$, $w_r = y$.
По определению $C_0$ имеем:
\[
C_0 := F_{x,y}(\varepsilon, y, s_s).
\]
Так как $C_0$ входит в конъюнкцию $C$, то
\[
C \vdash F_{x,y}(\varepsilon, y, s_s).
\]
База доказана.

\noindent \textbf{Шаг индукции:} $k = m + 1$.

Предположим, что состояние $s$ достижимо за $m+1$ шагов. Тогда существует состояние $s'$ со строкой $\text{rev}(w_l')@w_r'$, которое достижимо за $m$ шагов, и переход из $s'$ в $s$ за один шаг.

По \textbf{предположению индукции} для состояния $s'$:
\[
C \vdash F_{x,y}(w_l', w_r', s').
\]

Поскольку переход из $s'$ в $s$ возможен, существует формула $C_t$ в семействе $C$, которая кодирует этот переход. Формально, $C_t$ имеет вид:
\[
F_{x,y}(w_l', w_r', s') \rightarrow F_{x,y}(w_l, w_r, s).
\]

Применяя \textit{modus ponens} к $C_t$ и выводимой формуле $F_{x,y}(w_l', w_r', s')$, получаем:
\[
C \vdash F_{x,y}(w_l, w_r, s).
\]

\end{proof}

\textbf{(Неразрешимость ИП)}

Язык всех доказуемых формул исчисления предикатов неразрешим.

\begin{proof} Пусть существует машина Тьюринга, разрешающая любую формулу. \pause%Тогда покажем, что задача останова разрешима.\pause
    На её основе тогда несложно построить некоторую машину Тьюринга, перестраивающую любую машину $S$ (с допускающим состоянием $s_f$ и входом $y$) 
    в её ограничения $C$ и разрешающую формулу ИП $C \rightarrow \exists w_l.\exists w_r.F_{S,y}(w_l,w_r,s_f)$. 
    Эта машина разрешит задачу останова.
    \end{proof}

\section{ Представление чисел через натуральные (целые, рациональные, вещественные).
Аксиоматика Пеано. Арифметические операции (сложение, умножение, возведение в степень) в аксиоматике Пеано.
Доказательство коммутативности сложения.
Порядок теории (0, 1, 2). Теории первого порядка. Формальная арифметика. Доказательство $a=a$.
Арифметизация математики, формализация категорических силлогизмов, предложенная Лейбницем.}

\textbf{(Представление чисел через натуральные)}

\begin{enumerate}
    \item Рациональные ($\mathbb{Q}$).\pause
    
          $Q = \mathbb{Z} \times \mathbb{N}$ --- множество всех простых дробей.\pause
    
          $\langle p,q \rangle$ --- то же, что $\frac{p}{q}$ \pause
    
          $\langle p_1,q_1 \rangle \equiv \langle p_2, q_2 \rangle$, если $p_1q_2 = p_2q_1$\pause
    
    \vspace{0.1cm}
          $\mathbb{Q} = Q/_\equiv$
    
    \item Вещественные ($\mathbb{R}$). \pause $X = \{ A, B \}$, где $A,B \subseteq \mathbb{Q}$ --- дедекиндово сечение, если:\pause
    \begin{enumerate}
    \item $A\cup B = \mathbb{Q}$\pause
    \item Если $a \in A$, $x \in \mathbb{Q}$ и $x \le a$, то $x \in A$\pause
    \item Если $b \in B$, $x \in \mathbb{Q}$ и $b \le x$, то $x \in B$\pause
    \item $A$ не содержит наибольшего.\pause
    \end{enumerate}

\end{enumerate}

$N$ (или, более точно, $\langle N, 0, (')\rangle$) \emph{соответствует} \textbf{аксиоматике Пеано}, 
  если следующее определено/выполнено:\pause
  \begin{enumerate}
     \item Операция <<штрих>> $('): N \to N$, причём нет $a,b \in N$, что $a \ne b$, но $a' = b'$.\pause
           Если $x = y'$, то $x$ назовём следующим за $y$, а $y$ --- предшествующим $x$.\pause
     \item Константа $0 \in N$: нет $x \in N$, что $x' = 0$.\pause
     \item Индукция. Каково бы ни было свойство (<<предикат>>) $P: N \to V$, если:
           \begin{enumerate}
           \item $P(0)$
           \item При любом $x\in N$ из $P(x)$ следует $P(x')$
           \end{enumerate}
           то при любом $x \in N$ выполнено $P(x)$.
  \end{enumerate}


(Теорема) 0 единственен: если $t$ таков, что при любом $y$ 
выполнено $y' \ne t$, то $t = 0$.\\

\begin{proof}
    \begin{itemize}
    \item Определим $P(x)$ как <<либо $x = 0$, либо $x = y'$ для некоторого $y \in N$>>.\pause
    \begin{enumerate}
    \item $P(0)$ выполнено, так как $0 = 0$.\pause
    \item Если $P(x)$ выполнено, то возьмём $x$ в качестве $y$: тогда для $P(x')$
    будет выполнено $x' = y'$.\pause
    \end{enumerate}
    Значит, $P(x)$ для любого $x \in N$.\pause
    
    \item Рассмотрим $P(t)$: <<либо $t = 0$, либо $t = y'$ для некоторого $y \in N$>>.
    Но так как такого $y$ нет, то неизбежно $t = 0$.
    \end{itemize}
\end{proof}

\textbf{(Определения)}

$1 = 0'$, $2 = 0''$, $3 = 0'''$, $4 = 0''''$, $5 = 0'''''$, $6 = 0''''''$,
$7 = 0'''''''$, $8 = 0''''''''$, $9 = 0'''''''''$

\vspace{-0.3cm}
$$a + b = \left\{ \begin{array}{ll} a, & \mbox{если } b = 0\\
                                    (a + c)', & \mbox{если } b = c'
                  \end{array}\right.$$

                  \vspace{-0.3cm}
$$a \cdot b = \left\{ \begin{array}{ll} 0, & \mbox{если } b = 0\\
                                    a \cdot c + a, & \mbox{если } b = c'
                  \end{array}\right.$$

\textbf{Коммутативность сложения}

(Лемма) $a + 0 = 0 + a$

\begin{proof} Пусть $P(x)$ --- это $x + 0 = 0 + x$.
    \begin{enumerate}\pause
    \item Покажем $P(0)$. $0 + 0 = 0 + 0$ \pause
    \item Покажем, что если $P(x)$, то $P(x')$. Покажем $P(x')$, то есть $x' + 0 = \dots$ \pause
    
    \begin{center}
      \begin{tabular}{lrl} %Равенство & обоснование \\\hline
                          $\dots = x'$ & $a=x',b=0$: & $x' + 0 \Rightarrow x'$ \\\pause
                          $\dots = (x)'$ & \\ \pause
                          $\dots = (x + 0)'$ & $a=x,b=0$: &$(x + 0) \Leftarrow (x)$ \\ \pause
                          $\dots = (0 + x)'$ & $P(x)$: &$(x + 0) \Rightarrow (0 + x)$ \\ \pause
                          $\dots = 0 + x'$ & $a=0,b=x'$: &$0 + x' \Leftarrow (0 + x)'$
       \end{tabular}
    \end{center}
    \end{enumerate}\pause
    Значит, $P(a)$ выполнено для любого $a \in N$.
    \end{proof}

(Лемма) $a + b' = a' + b$

\begin{proof} $P(x)$ --- это $a + x' = a' + x$\pause
    \begin{enumerate}
    \item $a + 0' = (a + 0)' = (a)' = a' = a' + 0$\pause
    \item Покажем, что $P(x')$ следует из $P(x)$: $a + x'' = (a + x')' = (a' + x)' = a' + x'$
    \end{enumerate}
\end{proof}

(Теорема) $a + b = b + a$

\begin{proof}[Доказательство индукцией по $b$: $P(x)$ --- это $a + x = x + a$]
    \begin{enumerate}
    \item $a + 0 = 0 + a$ (лемма 1)\pause
    \item $a + x' = (a + x)' = (x + a)' = x + a' = x' + a$
    \end{enumerate}
\end{proof}

\textbf{Теорией первого порядка} назовём исчисление предикатов с дополнительными (<<нелогическими>>
или <<математическими>>):
\begin{itemize}
\item предикатными и функциональными символами;
\item аксиомами.
\end{itemize}

Сущности, взятые из исходного исчисления предикатов, назовём логическими\\


\begin{tabular}{llll}
    Порядок & Кванторы & Формализует суждения\dots & Пример\\\hline
    нулевой & запрещены & об отдельных значениях & И.В.\\
    первый & по предметным переменным & о множествах & И.П.\\
    %    &   & $S = \{ t\ |\ \psi[x := t] \}$ \\
        &   \multicolumn{2}{l}{\color{olive}$\{2,3,5,7,\dots\} = \{ t\ |\ \forall p.\forall q.(p \ne 1 \with q \ne 1) \rightarrow (t \ne p\cdot q)\}$}\\
    второй & по предикатным переменным & о множествах множеств & Типы\\
        &   \multicolumn{2}{l}{\color{olive}$S = \{ \{t\ |\ P(t)\}\ |\ \varphi[p := P] \}$}\\
     & \dots 
    \end{tabular}
\pause\vspace{0.3cm}    

\textbf{Формальная арифметика} --- теория первого порядка, со следующими добавленными нелогическими \dots
\begin{itemize}
\item двухместными функциональными символами $(+)$, $(\cdot)$; одноместным функциональным символом $(')$, 
нульместным функциональным символом $0$;\pause
\item двухместным предикатным символом $(=)$;\pause
\item восемью нелогическими \emph{аксиомами}:\vspace{0.1cm}
\begin{tabular}{ll}
(A1) $a=b \to a=c \to b=c$             &(A5) $a+0 = a$                     \\
(A2) $a=b \to a'=b'$                   &(A6) $a+b' = (a+b)'$               \\
(A3) $a'=b' \to a=b$                   &(A7) $a\cdot 0 = 0$                \\
(A4) $\neg a' = 0$                     &(A8) $a\cdot b' = a \cdot b + a$
\end{tabular}\pause
\item нелогической схемой аксиом индукции $\psi[x:=0]\with(\forall x.\psi\to \psi[x:=x'])\to \psi$ с метапеременными $x$ и $\psi$.
\end{itemize}

\textbf{Доказательство a = a}

Пусть $\top ::= 0=0\to 0=0 \to 0=0$, тогда:

\begin{tabular}{lll}
(1) & $a=b\to a=c \to b=c$ & (Акс. А1)\\
(2) & $(a=b\to a=c \to b=c) \to \top \to (a=b\to a=c \to b=c)$ & (Сх. акс. 1)\\
(3) & $\top \to (a=b\to a=c \to b=c)$ & (M.P. 1, 2)\\
(4) & $\top \to (\forall c.a = b\to a = c \to b = c)$ & (Введ. $\forall$)\\\pause
(5) & $\top \to (\forall b.\forall c.a = b\to a = c \to b = c)$ & (Введ. $\forall$)\\
(6) & $\top \to (\forall a.\forall b.\forall c.a = b\to a = c \to b = c)$ & (Введ. $\forall$)\\\pause
(7) & $\top$ & (Сх. акс 1)\\
(8) & $(\forall a.\forall b.\forall c.a = b\to a = c \to b = c)$ & (M.P. 7, 6)\\\pause
(9) & $(\forall a.\forall b.\forall c.a = b\to a = c \to b = c) \to $\\
    & $\to (\forall b.\forall c.a+0 = b\to a+0 = c \to b = c)$ & (Сх. акс. 11)\\\pause
(10) & $\forall b.\forall c.a+0 = b\to a+0 = c \to b = c$ & (M.P. 8, 9)\\\pause
(12) & $\forall c.a+0 = a\to a+0 = c \to a = c$ & (M.P. 10, 11)\\\pause
(14) & $a+0 = a\to a+0 = a \to a = a$ & (M.P. 12, 13)\\\pause
(15) & $a+0 = a$ & (Акс. А5)\\
(16) & $a+0 = a \to a = a$ & (M.P. 15, 14)\\
(17) & $a = a$ & (M.P. 15, 16) 
\end{tabular}

\textbf{Арифметизация по Лейбницу}

Правила, по которым можно с помощью чисел судить о правильности выводов, о формах и модусах категорических силлогизмов (1679 г.)\end{center}

\begin{itemize}
\item Любой термин --- пара взаимно простых чисел $+a-b$. Например, мудрый --- $+70-33$, благочестивый --- $+10-3$.

\item Общеутвердительное предложение (каждый $+a-b$ есть $+c-d$): $a \text{делится на} c$ и $b \text{делится на} d$.\\
Всякий мудрый есть благочестивый ($70 = 10\cdot 7$, $33 = 3 \cdot 11$).

\item Частноотрицательное предложение --- не верно общеутвердительное.

\item Общеотрицательное предложение --- когда $a,d$ или $b,c$ имеют общий делитель, отличный от 1:\\
Ни один благочестивый ($+10-3$) не есть несчастный ($+5-14$), так как $10=2 \cdot 5$ и $14 =2 \cdot 7$.

\item Очевидно, силлогизм <<Barbara>> верен: $$\infer{e \text{делится на} c, f \text{делится на} d}{+a-b\text{ есть }+c-d\quad\quad+e-f\text{ есть }+a-b}$$

\item Неправильный силлогизм может быть верен иногда, скажем, <<AOO>> из фигуры 3:\\Всякий благочестивый ($+10-3$) есть счастливый
($+5-1$), некоторый благочестивый не есть богатый ($+8-11$), отсюда некоторый богатый не есть счастливый ($+8-11$ против $+5-1$).

\item Но для некоторых пар всё равно следствие данного неправильного силлогизма нарушается: $+12-5$, $+4-1$, $+8-11$
\end{itemize}

\section{Примитивно-рекурсивные и рекурсивные функции.
Функции вычисления простых чисел. Частичный логарифм.
Выразимость отношений и представимость функций в формальной арифметике. Характеристические функции.
Функция Аккермана. Доказательство невозможности выражения функции Аккермана в примитивно-рекурсивных функциях.}

\textbf{Примитивы}

\begin{enumerate}
    \item Примитив <<Ноль>> ($Z$) \vspace{-0.3cm}
    $$Z: \mathbb{N}_0\to\mathbb{N}_0,\ \ \ \ \ Z(x_1) = 0$$\vspace{-0.5cm}\pause
    \item Примитив <<Инкремент>> ($N$) \vspace{-0.3cm}
    $$N: \mathbb{N}_0\to\mathbb{N}_0,\ \ \ \ \ N(x_1) = x_1+1$$\vspace{-0.5cm}\pause
    \item Примитив <<Проекция>> ($U$) — семейство функций; пусть $k,n \in \mathbb{N}_0, k \le n$\vspace{-0.3cm}
    $$U^k_n: \mathbb{N}^n_0 \to \mathbb{N}_0,\ \ \ \ \ U^k_n(\overrightarrow{x}) = x_k$$\vspace{-0.5cm}\pause
    \item Примитив <<Подстановка>> ($S$) --- семейство функций; пусть $g: \mathbb{N}^k_0 \to \mathbb{N}_0,\ \ f_1,\dots,f_k: \mathbb{N}^n_0 \to \mathbb{N}_0$
     \vspace{-0.3cm}
    $$S\langle g,f_1,f_2,\dots,f_k \rangle (\overrightarrow{x}) = g(f_1(\overrightarrow{x}),\dots,f_k(\overrightarrow{x}))$$
    \item Примитив <<Примитивная рекурсия>> ($R$) --- семейство функций;
    пусть $f: \mathbb{N}^n_0\to\mathbb{N}_0$ и $g: \mathbb{N}^{n+2}_0 \to\mathbb{N}_0$.
    Тогда $R\langle f,g\rangle: \mathbb{N}^{n+1}_0\to\mathbb{N}_0$, причём
    
    $$R\langle f,g\rangle(\overrightarrow{x},y)=
     \left\{\begin{array}{ll} 
      f(\overrightarrow{x}), &y=0\\
      g(\overrightarrow{x},y-1,R\langle f,g\rangle (\overrightarrow{x},y-1)), &y > 0
    \end{array}\right.$$
\end{enumerate}

Функция $f$ --- \textbf{примитивно-рекурсивна}, если может быть
выражена как композиция примитивов $Z$, $N$, $U$, $S$ и $R$.

Функция --- \textbf{общерекурсивная}, если может быть построена при помощи
примитивов $Z$, $N$, $U$, $S$, $R$ и примитива минимизации:
$$M\langle f \rangle (x_1,x_2,\dots,x_n) = \min\{y: f(x_1,x_2,\dots,x_n,y) = 0\}$$
Если $f(x_1,x_2,\dots,x_n,y) > 0$ при любом $y$, результат не определён.

\textbf{Вычисление простых чисел}

Обозначим $B(n)=2^{2^{\,n}}$ (примитивно-рекурсивная верхняя оценка для $p_n$).
Пусть $\operatorname{rem}(x,y)=x\bmod y$ и $\operatorname{sg}(x)=\begin{cases}0,&x=0,\\1,&x>0\end{cases}$
— примитивно-рекурсивны.

\medskip
\emph{Делимость и признак простоты.}
\[
\operatorname{Divides}(d,m)\;=\;1-\operatorname{sg}\big(\operatorname{rem}(m,d)\big)
\qquad(\text{1 при }d\mid m,\ \ 0\text{ иначе}),
\]
\[
S(m,t)=\sum_{d=2}^{t}\operatorname{Divides}(d,m)
\quad\text{(сумма задаётся примитивной рекурсией)},
\]
\[
\operatorname{HasDiv}(m,t)=\operatorname{sg}\big(S(m,t)\big)
\quad(\text{1, если }\exists d\in[2,t]\ d\mid m).
\]
Тогда индикатор простоты
\[
\operatorname{IsPrime}(m)=\big(1-\operatorname{HasDiv}(m,m-1)\big)\cdot\operatorname{sg}(m-1),
\]
(равен $1$ именно для простых $m\ge2$).

\medskip
\emph{Накопитель по верхней границе.} Для фиксированного $n$ определим функцию
$F(n,t)=(\count_t,\res_t,\flag_t)$ рекурсивно по $t$ ($t=0,\dots,B(n)$):
\[
F(n,0)=(0,0,0).
\]
Пусть $F(n,t)=(c,r,f)$. Обозначим $a=\operatorname{IsPrime}(t+1)$ и $c' = c+a$.
Тогда
\[
F(n,t+1)=
\begin{cases}
(c',\,t+1,\,1), & \text{если }f=0,\ a=1,\ c'=n,\\
(c',\,r,\,f),   & \text{иначе.}
\end{cases}
\]
(В словах: при первом достижении счётчика равного $n$ запоминаем текущее простое в $r$ и ставим флаг.)

Эта рекурсия по $t$ — примитивно-рекурсивна, поэтому композиции и проекции дают функцию
\[
\operatorname{nthPrime}(n)=U_2\big(F(n,B(n))\big),
\]
где $U_2$ — проекция на компоненту $\res$.

\textbf{Частичный логарифм}

Определим характеристику неделимости:
\[
\operatorname{Div}(n,m)=
\begin{cases}
0,& m\mid n,\\
1,& m\nmid n.
\end{cases}
\]

Пусть
\[
h(n,y)=\operatorname{Div}\bigl(n,\operatorname{pow}_k(y+1)\bigr).
\]

Тогда
\[
s(n)=M\,[\,h(n,y)=1\,], \qquad
\operatorname{plog}_k(n)=s(n)-1.
\]


Запись вида $\psi(\theta_1,\dots,\theta_n)$ означает
$\psi[x_1:=\theta_1,\dots,x_n:=\theta_n]$

Литерал числа

$$\overline{a} = \left\{\begin{array}{ll} 0, &\mbox{если } a = 0\\
                (\overline{b})', &\mbox{если } a = b+1
\end{array}\right.$$

Будем говорить, что отношение $R\subseteq \mathbb{N}^n_0$ \textbf{выразимо} в ФА, 
если существует формула $\rho$, что:
\begin{enumerate}
\item если $\langle a_1,\dots,a_n \rangle \in R$, то $\vdash \rho(\overline{a_1},\dots,\overline{a_n})$
\item если $\langle a_1,\dots,a_n \rangle \notin R$, то $\vdash \neg\rho(\overline{a_1},\dots,\overline{a_n})$
\end{enumerate}

Будем говорить, что функция $f: \mathbb{N}^n_0\to\mathbb{N}_0$ \textbf{представима} в ФА, 
если существует формула $\varphi$, что:
\begin{enumerate}
\item если $f(a_1,\dots,a_n) = u$, то $\vdash \varphi(\overline{a_1},\dots,\overline{a_n},\overline{u})$
\item если $f(a_1,\dots,a_n) \ne u$, то $\vdash \neg\varphi(\overline{a_1},\dots,\overline{a_n},\overline{u})$
\item для всех $a_i \in \mathbb{N}_0$ выполнено $\vdash (\exists x.\varphi(\overline{a_1},\dots,\overline{a_n},x)) \with 
   (\forall p.\forall q.\varphi(\overline{a_1},\dots,\overline{a_n},p)\with \varphi(\overline{a_1},\dots,\overline{a_n},q)\rightarrow p=q)$
\end{enumerate}

\textbf{Функция Аккермана}
$$A(m,n) = \left\{\begin{array}{ll}
  n+1,&m = 0\\
  A(m-1,1),&m > 0, n = 0\\
  A(m-1,A(m,n-1)),&m > 0, n > 0
\end{array}\right.$$

Лемма о росте функции Аккермана

\begin{enumerate}
    \item $A(p,q) = A^{(q+1)}(p-1,1)$
    \item $A^{(x+2)}(k,x) < A(k+2,x)$
    \end{enumerate}

    \begin{proof}
        \begin{enumerate}
        \item $A(p,q) = A(p-1,A(p,q-1)) = \dots = A(p-1,A(p-1,\dots A(p,0)) = A^{(q)}(p-1,A(p,0)) = A^{(q+1)}(p-1,1)$
        
        \item
        %$\alpha_{m+2}(n) = \alpha_{m+1}^{n+1}(1) = \alpha_{m+1}(\alpha_{m+1}^n(1)) =
        %\alpha_m^{\alpha_{m+1}^n(1)+1}(1) = \alpha_m^{\alpha_{m+2}(n-1)+1}(1)
        %\ge \alpha_m^{\alpha_2(n-1)+1}(1) = \alpha_m^{2(n-1)+3+1}(1) =
        % \alpha_m^{n+n+2}(1) = \alpha_m^{n+2} (\alpha_m^{n}(1))$
        
        $A(k+2,x) = A(k+1,A(k+2,x-1)) = A^{(A(k+2,x-1)+1)}(k,1) \ge A^{(A(2,x-1)+1)}(k,1) = A^{(2(x-1)+3+1)}(k,1)
        = A^{(2x+2)}(k,1) = A^{(x+2)}(k,A^{(x)}(k,1)) \ge A^{(x+2)}(k,A^{(x)}(0,1)) = A^{(x+2)}(k,x+1) > A^{(x+2)}(k,x)$
        
        \end{enumerate}
        \end{proof}

\textbf{Функция Аккермана не примитивно-рекурсивна}        

Пусть $f(\overrightarrow{x})$ --- примитивно-рекурсивная.
Тогда найдётся $k$, что $f(\overrightarrow{x}) < A(k,\max(\overrightarrow{x}))$
\begin{proof}
    Индукция по структуре $f$.
    \begin{enumerate}
    \item $f = Z$, тогда $k = 0$, т.к. $A(0,x) = x+1 > Z(x) = 0$;
    \item $f = N$, тогда $k = 1$, т.к. $A(1,x) = x + 2 > N(x) = x+1$;
    \item $f = U^n_s$, тогда $k = 0$, т.к. $f(\overrightarrow{x}) \le \max(\overrightarrow{x}) < A(0,\max(\overrightarrow{x}))$;
    \item $f = S\langle g,h_1,\dots,h_n\rangle$, тогда $k = k_g + \max(k_{h_1},\dots,k_{h_n}) + 2$;
    %f(\overrightarrow{x}) < A(k_f,\max(\overrightarrow{x}))$, $h_t(\overrightarrow{x}) < A(k_{h_t},\max(\overrightarrow{x}))$.
    %Пусть $k = \max(k_g,k_{h_1},\dots,k_{h_n})+2$;
     %$S\langle g,h_1,\dots,h_n\rangle < A(k_g, A(\max (\overrightarrow{k_h}), \max({\overrightarrow{x}))) = A^{(\max)}$
    \item $f = R\langle g,h \rangle$, тогда $k = \max(k_g,k_h)+2$. %$R\langle g,h\rangle(x,v) = g^{[v]}(h(x)) < A(k_g,A(\dots A(k_g,A(k_h,x)))) \le A(x \ $
    % < A(k_g,A(\dots A(k_g,A(k_h,max(\overrightarrow{x},y))))) 
    %\le A^{(y}$
    \end{enumerate}
\end{proof}

\textbf{Оценка для R}

Пусть $f = R\langle g,h \rangle$. Тогда при $k = \max(k_g,k_h)+2$ выполнено $f(\overrightarrow{x},y) \le A^{(y+1)}(k-2,\max(\overrightarrow{x},y))$.

\begin{proof}

    Индукция по $y$.
    \begin{itemize}
    \item База: $y = 0$. Тогда: $f(\overrightarrow{x},0) = g(\overrightarrow{x}) \le 
    A(k_g,\max(\overrightarrow{x})) \le A^{(1)}(k-2,\max(\overrightarrow{x},0))$.
    
    \item Переход: пусть $f(\overrightarrow{x},y) \le A^{(y+1)}(k-2,\max(\overrightarrow{x},y))$.
    Тогда $f(\overrightarrow{x},y+1) = h(\overrightarrow{x},y,f(\overrightarrow{x},y))
    \le A(k_h,\max(\overrightarrow{x},y,f(\overrightarrow{x},y))) 
    \le A(k_h,\max(\overrightarrow{x},y,A^{(y+1)}(k-2,\max(\overrightarrow{x},y))) = A(k_h,A^{(y+1)}(k-2,\max(\overrightarrow{x},y)))
    \le A^{(y+2)}(k-2,\max(\overrightarrow{x},y+1))$
    
    %\le \alpha^{y+2}_{K-2}(\max(\overrightarrow{x},y)))\le$$
    \end{itemize}
    \vspace{-0.3cm}
    \end{proof}

    Заметим, что $A^{(y+1)}(k-2,\max(\overrightarrow{x},y)) \le
A^{(\max(\overrightarrow{x},y)+1)}(k-2,\max(\overrightarrow{x},y)) \le
 A^{(\max(\overrightarrow{x},y)+2)}(k-2,\max(\overrightarrow{x},y)) <
A(k,\max(\overrightarrow{x},y))$

\textbf{Тезис Чёрча для общерекурсивных функций}:

любая эффективно-вычислимая функция $\mathbb{N}^k_0\to\mathbb{N}_0$ является общерекурсивной.

\section{Представимость примитивов $N$, $Z$, $U$, $S$. Бета-функция Гёделя.
Представимость $R$ и $M$, представимость рекурсивных функций в формальной арифметике.}

\textbf{Примитивы $Z$, $N$ и $U^k_n$ представимы в Ф.А.}

\begin{proof}
    \begin{itemize}\pause
    \item $\zeta(x_1,x_2) := x_2=0$, \pause формальнее: $\zeta(x_1,x_2) := x_1=x_1 \with x_2=0$\pause
    \item $\nu(x_1,x_2) := x_2=x_1'$\pause
    \item $\upsilon(x_1,\dots,x_n,x_{n+1}) := x_k = x_{n+1}$ \pause \vspace{0.1cm}
    
    формальнее: 
         $\upsilon(x_1,\dots,x_n,x_{n+1}) := (\underset{i\ne k,n+1}{\with} x_i=x_i) \with x_k = x_{n+1}$
    \end{itemize}
\end{proof}

Пусть функции $f,g_1,\dots,g_k$ представимы в Ф.А. Тогда $S\langle f,g_1,\dots,g_k \rangle$ представима в Ф.А.

\begin{proof}
    Пусть $f$, $g_1$, ..., $g_k$ представляются формулами $\varphi$, $\gamma_1$, ..., $\gamma_k$. \pause
    
    Тогда 
    $S\langle f,g_1,\dots,g_k\rangle$ будет представлена формулой
    $$\exists g_1.\dots.\exists g_k.\varphi(g_1,\dots,g_k,x_{n+1})\with\gamma_1(x_1,\dots,x_n,g_1)\with\dots\with\gamma_k(x_1,\dots,x_n,g_k)$$
\end{proof}

\textbf{Бета-функция Геделя}

Задача: закодировать последовательность натуральных чисел произвольной длины.
$\beta$-функция Гёделя: $\mathcal{\beta}(b,c,i) := b \% (1 + (i+1) \cdot c)$\\
Здесь (\%) --- остаток от деления.
\end{dfn}\pause
\begin{thm}$\beta$-функция Гёделя представима в Ф.А. формулой
$$\hat{\beta}(b,c,i,d) := \exists q.(b = q \cdot (1 + c \cdot (i+1)) + d) \& (d < 1 + c \cdot (i+1))$$

Китайская теорема об остатках (вариант формулировки): если $u_0, \dots, u_n$ --- попарно 
взаимно просты, и $0 \le a_i < u_i$, то существует такой $b$, что $a_i = b \% u_i$.

\begin{proof}
    Положим $c = \max(a_0,\dots,a_n,n)!$ и $u_i = 1+c\cdot(i+1)$.\pause
    \begin{itemize}
    \item $\text{НОД}(u_i,u_j) = 1$, если $i \ne j$.\pause
    
    Пусть $p$ --- простое, $u_i\divisible p$ и $u_j\divisible p$ ($i < j$). \pause
    Заметим, что $u_j-u_i = c \cdot (j-i)$. Значит, $c\divisible p$ или $(j-i)\divisible p$. \pause
    Так как $j-i \le n$, то $c \divisible (j-i)$, потому если и $(j-i)\divisible p$, всё равно $c \divisible p$. \pause
    Но и $(1+c\cdot(i+1))\divisible p$, отсюда $1 \divisible p$ --- что невозможно. \pause
    \item $0 \le a_i < u_i$.\pause
    \end{itemize}
    Условия китайской теоремы об остатках выполнены и найдётся $b$, что \vspace{-0.2cm}
    $$a_i = b \% (1 + c\cdot(i+1)) = \beta(b,c,i)$$

\end{proof}

\textbf{Примитив <<примитивная рекурсия>> представим в Ф.А.}

Пусть $f: \mathbb{N}^n_0\to\mathbb{N}_0$ и $g:\mathbb{N}^{n+2}_0\to\mathbb{N}_0$ представлены формулами $\varphi$ и $\gamma$. 

Зафиксируем $x_1, \dots, x_n, y \in \mathbb{N}_0$.\pause\vspace{0.3cm}

\begin{tabular}{lll}
Шаг вычисления & Об. & Утверждение в Ф.А.\\
$R\langle f,g\rangle (x_1,\dots,x_n,0) = f(x_1,\dots,x_n)$ & $a_0$ & $\vdash \varphi(\overline{x_1},\dots,\overline{x_n},\overline{a_0})$\\ \pause
$R\langle f,g\rangle (x_1,\dots,x_n,1) = g(x_1,\dots,x_n,0,a_0)$ & $a_1$ & $\vdash \gamma(\overline{x_1},\dots,\overline{x_n},0,\overline{a_0},\overline{a_1})$\\ \pause
$\dots$\\
$R\langle f,g\rangle (x_1,\dots,x_n,y) = g(x_1,\dots,x_n,y-1,a_{y-1})$ & $a_y$ & $\vdash \gamma(\overline{x_1},\dots,\overline{x_n},\overline{y-1},\overline{a_{y-1}},\overline{a_y})$
\end{tabular}\pause\vspace{0.3cm}

По свойству $\beta$-функции, найдутся $b$ и $c$, что
$\beta (b,c,i) = a_i$ для $0 \le i \le y$.\pause

Примитив $R\langle f,g\rangle$ представим в Ф.А. формулой $\rho(x_1,\dots,x_n,y,a)$:\vspace{-0.1cm}
$$\begin{array}{l}\exists b. \exists c. (\exists a_0. \hat{\beta} (b,c,0,a_0) \& \varphi (x_1,...x_n, a_0)) \\
       \&\;\;\;\;\forall k.k < y \rightarrow \exists d . \exists e . \hat{\beta} (b,c,k,d) \& \hat{\beta} (b,c,k',e) \& \gamma (x_1,..x_n,k,d,e) \\
       \&\;\;\;\;\hat{\beta} (b,c,y,a) 
\end{array}$$


Пусть функция $f:\mathbb{N}^{n+1}_0 \to \mathbb{N}_0$ представима в Ф.А.
формулой $\varphi(x_1,\dots,x_{n},y,r)$. Тогда \textbf{примитив $M\langle f\rangle$ представим в Ф.А.} 
формулой $$\mu(x_1,\dots,x_n,y) := \varphi(x_1,\dots,x_n,y,0) \with \forall u.u < y \to \neg\varphi(x_1,\dots,x_n,u,0)$$

\textbf{Если $f$ --- рекурсивная функция, то она представимав Ф.А.}
\begin{proof}Индукция по структуре $f$.
\end{proof}

\section{Гёделева нумерация. Рекурсивность представимых в формальной арифметике функций.
Функции $W_1$ и $W_2$, их представления $\omega_1$, $\omega_2$, самоприменимость.}

Фиксируем $f$ и $x_1, x_2, \dots, x_n$. Обозначим $y = f(x_1,x_2,\dots,x_n)$. \pause
По представимости нам известна $\varphi$, что $\vdash \varphi(\overline{x_1},\overline{x_2},\dots,\overline{x_n},\overline{y})$. \pause
Давайте просто переберём все результаты и доказательства!\pause

\begin{enumerate}
\item Закодируем доказательства натуральными числами.\pause
\item Напишем рекурсивную функцию, проверяющую доказательства на корректность.\pause
\item Параллельный перебор значений и доказательств: $s = 2^y \cdot 3^p$. Переберём все $s$, по $s$ получим $y$ и $p$.
Проверим, что $p$ --- код доказательства $\vdash \varphi(\overline{x_1},\overline{x_2},\dots,\overline{x_n},\overline{y})$.
\end{enumerate}

\textbf{Геделева нумерация}

\begin{enumerate}
    \item Отдельный символ.\vspace{0.2cm}
    

\begin{tabular}{lc|lc||cll}
    Номер & Символ & Номер & Символ & Имя & $k,n$ & Гёделев номер\\
    3 & ( &               17 & $\&$ &  0 & $0,0$ & $27 + 6$\\
    5 & ) &               19 & $\forall$ & $(')$ & $0,1$ & $27 + 6 \cdot 3$\\
    7 & , &               21 & $\exists$ & $(+)$ & $0,2$ & $27 + 6 \cdot 9$\\
    9 & . &               23 & $\vdash$ & $(\cdot)$ & $1,2$ & $27 + 6 \cdot 2 \cdot 9$\\
    11 & $\neg$ &         $25 + 6\cdot k$ & $x_k$ & $(=)$ & $0,2$ & $29 + 6 \cdot 9$\\
    13 & $\rightarrow$ &  $27 + 6\cdot 2^k \cdot 3^n$ & $f_k^n$\\
    15 & $\vee$ &         $29 + 6\cdot 2^k \cdot 3^n$ & $P_k^n$ 
\end{tabular}

\item Формула. $\phi \equiv s_0s_1\dots s_{n-1}$. Гёделев номер: $\ulcorner\phi\urcorner = 2^{\ulcorner s_0\urcorner}\cdot 3^{\ulcorner s_1\urcorner} 
\cdot \dots \cdot p_{n-1}^{\ulcorner s_{n-1}\urcorner}$.\pause

\item Доказательство. $\Pi = \delta_0\delta_1\dots\delta_{k-1}$, его гёделев номер: $\ulcorner\Pi\urcorner =
2^{\ulcorner \delta_0\urcorner}\cdot 3^{\ulcorner \delta_1\urcorner} \cdot \dots \cdot p_{k-1}^{\ulcorner \delta_{k-1}\urcorner}$
\end{enumerate}

(Лемма) Следующая функция рекурсивна:
$$\text{proof}(f,x_1,x_2,\dots,x_n,y,p) = \left\{\begin{array}{ll} 
0, & \mbox{если} \vdash\phi(\overline{x_1},\overline{x_2},\dots,\overline{x_n},\overline{y}),\\
   & p \mbox{ --- гёделев номер вывода}, f = \ulcorner\phi\urcorner \\ 
1, & \mbox{иначе}
\end{array}\right.$$\vspace{-0.1cm}

\begin{proof}
    \begin{enumerate}
    \item Проверка доказательства вычислима.\pause
    \item Согласно тезису Чёрча, любая вычислимая функция
    вычислима с помощью рекурсивных функций.
    \end{enumerate}
    \end{proof}

    Следующие функции рекурсивны:
\begin{enumerate}
\item Функции $\text{plog}_k(n) = \max\{p: n \mathrel{\vdots} k^p\}$, $\text{fst}(x) = \text{plog}_2(x)$ и $\text{snd}(x) = \text{plog}_3(x)$.\pause
\item Числовые литералы: $\overline{k}: \mathbb{N}_0 \rightarrow \mathbb{N}_0$, $\overline{k}(x) = k$.\pause
\end{enumerate}

\textbf{(Теорема)} Любая представимая в Ф.А. функция рекурсивна(может быть построена через примитивы).

\begin{proof}
    Пусть заданы $x_1,x_2,\dots,x_n$. Ищем $\langle y, p\rangle$, 
    что $\text{proof}(\ulcorner\varphi\urcorner,x_1,x_2,\dots,x_n,y,p)=1$, \pause
    напомним: $y = f(x_1,x_2,\dots,x_n)$, $p = \ulcorner\Pi\urcorner$,
    $\Pi$ --- доказательство $\varphi(\overline{x_1},\overline{x_2},\dots,\overline{x_n},\overline{y})$. \pause
    $$f = S \langle \text{fst}, M\langle S \langle \text{proof}, \overline{\ulcorner\varphi\urcorner}, U^1_{n+1}, U^2_{n+1}, \dots, U^n_{n+1}, 
      S \langle \text{fst},U^{n+1}_{n+1}\rangle, S\langle \text{snd}, U^{n+1}_{n+1}\rangle \rangle \rangle \rangle$$
    \end{proof}

\textbf{(Теорема)} Формальная арифметика корректна.

\begin{proof}Свойства аксиом $A1\dots A8$ очевидны. 

    Доказательство схемы аксиом индукции:
    Фиксируем некоторую формулу $\psi$, 
и рассмотрим схему аксиом $$\psi(0) \with (\forall x.\psi(x) \rightarrow \psi(x')) \rightarrow \psi(x)$$

Разберём случаи:
\begin{enumerate}
\item $\llbracket\psi(0)\rrbracket = \text{Л}$ либо 
$\llbracket\forall x.\psi(x) \rightarrow \psi(x')\rrbracket = \text{Л}$. Тогда антецедент импликации ложен 
и формула истинна.
\item Антецедент импликации истинен. Тогда докажем индукцией по $x$, что $\llbracket \psi(x) \rrbracket = \text{И}$.
База:
$x = 0$.

        Переход: $x := s'$. Тогда $s'$ раз применяем переход $$\llbracket \theta_0(x) = \theta_1(x) \rightarrow \theta_0(x') = \theta_1(x') \rrbracket^{x := \overline{0\dots s}} = \text{И}$$
        отсюда $$\llbracket\theta_0(x') = \theta_1(x') \rrbracket^{x := s} = \llbracket\theta_0(x) = \theta_1(x) \rrbracket^{x := s+1} = \text{И}$$ 
    \end{enumerate}
\end{proof} 

\textbf{Определения}

Пусть $\xi$ --- формула с единственной свободной
переменной $x_1$. Тогда:
$\langle\ulcorner \xi \urcorner,p\rangle \in W_1$, если $\vdash \xi(\overline{\ulcorner \xi \urcorner})$ и $p$ --- номер доказательства.


Отношение $W_1$ рекурсивно, поэтому выражено в Ф.А. формулой $\omega_1$ со свободными переменными $x_1$ и $x_2$, причём:
\begin{enumerate}
\item $\vdash \omega_1(\overline{\ulcorner \varphi \urcorner},\overline{p})$, если $p$ --- гёделев номер
доказательства самоприменения $\varphi$;
\item $\vdash \neg\omega_1(\overline{\ulcorner \varphi \urcorner},\overline{p})$ иначе.
\end{enumerate}

Пусть $\langle \ulcorner\xi\urcorner,p\rangle \in W_2$, если $\vdash\neg\xi(\overline{\ulcorner\xi\urcorner})$ и $p$ --- номер доказательства.
Пусть $\omega_2$ выражает $W_2$ в формальной арифметике.

\section{Непротиворечивость (эквивалентные определения), $\omega$-не\-про\-ти\-во\-ре\-чи\-вость.
Первая теорема Гёделя о неполноте арифметики.
Формулировка первой теоремы Гёделя о неполноте арифметики в форме Россера.
Синтаксическая и семантическая неполнота арифметики.
Неполнота расширений формальной арифметики.
Ослабленные варианты: арифметика Пресбургера, система Робинсона.}

Определим формулу $\sigma(x_1) := \forall p.\neg\omega_1(x_1,p)$(x1 несамоприменима).

Если для любой формулы $\phi(x)$ из $\vdash\phi(0)$, $\vdash\phi(\overline{1})$,
$\vdash\phi(\overline{2})$, $\dots$ выполнено $\not\vdash\exists x.\neg\phi(x)$, 
то теория \textbf{омега-непротиворечива}.

\textbf{Первая теорема Гёделя о неполноте арифметики}
\begin{itemize}
    \item Если формальная арифметика непротиворечива, то $\not\vdash\sigma(\overline{\ulcorner\sigma\urcorner})$.
    \item Если формальная арифметика $\omega$-непротиворечива, то $\not\vdash\neg\sigma(\overline{\ulcorner\sigma\urcorner})$.
\end{itemize}

\begin{proof}
    \begin{itemize}
    \item Пусть $\vdash\sigma(\overline{\ulcorner\sigma\urcorner})$. Значит, $p$ --- номер доказательства. \pause Тогда
    $\langle\ulcorner\sigma\urcorner,p\rangle \in W_1$. \pause Тогда $\vdash\omega_1(\overline{\ulcorner\sigma\urcorner},\overline{p})$. \pause 
    Тогда $\vdash\exists p.\omega_1(\overline{\ulcorner\sigma\urcorner},p)$. \pause То есть
    $\vdash\neg\forall p.\neg\omega_1(\overline{\ulcorner\sigma\urcorner},p)$. \pause То есть $\vdash\neg\sigma(\overline{\ulcorner\sigma\urcorner})$. Противоречие.
    \pause
    \item Пусть $\vdash\neg\sigma(\overline{\ulcorner\sigma\urcorner})$. \pause То есть $\vdash\exists p.\omega_1(\overline{\ulcorner\sigma\urcorner},p)$.
    \begin{itemize}
    \item
    \pause Но найдётся ли натуральное число $p$, что $\vdash\omega_1(\overline{\ulcorner\sigma\urcorner},\overline{p})$?
    \pause Пусть нет. То есть $\vdash\neg\omega_1(\overline{\ulcorner\sigma\urcorner},\overline{0})$,
                              $\vdash\neg\omega_1(\overline{\ulcorner\sigma\urcorner},\overline{1})$, 
                              \dots \pause По $\omega$-непротиворечивости $\not\vdash\exists p.\neg\neg\omega_1(\overline{\ulcorner\sigma\urcorner},p)$. \pause
    \end{itemize}
           Значит, найдётся натуральное $p$, что $\vdash\omega_1(\overline{\ulcorner\sigma\urcorner},\overline{p})$. \pause 
    То есть, $\langle\ulcorner\sigma\urcorner, p\rangle\in W_1$. \pause
    То есть, $p$ --- доказательство самоприменения $\sigma$: $\vdash\sigma(\overline{\ulcorner\sigma\urcorner})$. \pause Противоречие.
    \end{itemize}
\end{proof}

\textbf{Семантически полная теория} --- теория, в которой любая общезначимая формула доказуема.\\
\textbf{Синтаксически полная теория} --- теория, в которой для каждой формулы $\alpha$ выполнено $\vdash\alpha$ или $\vdash\neg\alpha$.

\textbf{(Теорема)} Формальная арифметика с классической моделью семантически неполна.

\begin{proof}
    Рассмотрим Ф.А. с классической моделью. \pause
    Из теоремы Гёделя имеем $\not\vdash\sigma(\overline{\ulcorner\sigma\urcorner})$. \pause
    Рассмотрим $\sigma(\overline{\ulcorner\sigma\urcorner}) \equiv \forall p.\neg\omega_1(\overline{\ulcorner\sigma\urcorner},p)$:
   нет числа $p$, что $p$ --- номер доказательства $\sigma(\overline{\ulcorner\sigma\urcorner})$. \pause 
    То есть, $\llbracket \forall p.\neg\omega_1(\overline{\ulcorner\sigma\urcorner},p) \rrbracket = \text{И}$. \pause
    То есть, $\models \sigma(\overline{\ulcorner\sigma\urcorner})$.
\end{proof}

\textbf{Первая теорема Гёделя о неполноте в форме Россера}

$\theta_1\le\theta_2 \equiv (\exists p.p+\theta_1=\theta_2)\quad\quad\theta_1<\theta_2\equiv\theta_1\le\theta_2\with\neg\theta_1=\theta_2$

Рассмотрим $\rho(x_1) = \forall p.\omega_1(x_1,p)\rightarrow\exists q.q \le p \with \omega_2(x_1,q)$. \pause
Тогда $\not\vdash\rho(\overline{\ulcorner\rho\urcorner})$ и $\not\vdash\neg\rho(\overline{\ulcorner\rho\urcorner})$. \pause
$\rho(\overline{\ulcorner\rho\urcorner})$: <<Меня легче опровергнуть, чем доказать>>\\

\textbf{О сужениях Ф.А.}

Теория $\mathcal{S}$ --- расширение теории $\mathcal{T}$, если
из $\vdash_\mathcal{T} \alpha$ следует $\vdash_\mathcal{S} \alpha$

Теория $\mathcal{S}$ --- рекурсивно-аксиоматизируемая, если найдётся теория $\mathcal{S'}$ с тем же языком, что:
\begin{enumerate}
\item $\vdash_\mathcal{S} \alpha$ тогда и только тогда, когда $\vdash_\mathcal{S'} \alpha$;
\item Множество аксиом теории $\mathcal{S'}$ рекурсивно.
\end{enumerate}

\textbf{(Теорема)}Если $\mathcal{S}$ --- непротиворечивое рекурсивно-аксиоматизируемое расширение формальной арифметики, то
в ней можно доказать аналоги теорем Гёделя о неполноте арифметики.

Теория первого порядка, использующая нелогические функциональные символы $0$, $(+)$ и $(\cdot)$, нелогический
предикатный символ $(=)$ и следующие нелогические аксиомы, называется \textbf{системой Робинсона}. Она оказывается неполной.

\vspace{-0.4cm}
$$\begin{array}{ll}
a = a & a = b \rightarrow b = a \\
a = b \rightarrow b = c \rightarrow a = c & a = b \rightarrow a' = b' \\
a' = b' \rightarrow a = b & \neg 0 = a' \\
a = b \rightarrow a + c = b + c \with c + a = c + b & a = b \rightarrow a \cdot c = b \cdot c \with c \cdot a = c \cdot b \\
\neg a = 0 \rightarrow \exists b. a = b' & a + 0 = a\\
a + b' = (a + b)' & a \cdot 0 = 0 \\
a \cdot b' = a \cdot b + a 
\end{array}$$

Теория первого порядка, использующая нелогические функциональные символы $0$, $1$, $(+)$, нелогический
предикатный символ $(=)$ и следующие нелогические аксиомы, называется \textbf{арифметикой Пресбургера}. Она уже и семантически, и синтаксически полна.

$$\begin{array}{l}
\neg (0 = x + 1) \\
x + 1 = y + 1 \rightarrow x = y\\
x + 0 = x \\
x + (y + 1) = (x + y) + 1\\
(\varphi(0) \with \forall x.\varphi(x) \rightarrow \varphi(x+1)) \rightarrow \forall y.\varphi(y)
\end{array}$$

\section{Вторая теорема Гёделя о неполноте арифметики, $Consis$.
Лемма об автоссылках. Условия Гильберта-Бернайса-Лёба. Неразрешимость формальной арифметики.
Теорема Тарского о невыразимости истины.}

(Лемма) $\vdash 1=0$ тогда и только тогда, когда $\vdash\alpha$ при любом $\alpha$(очевидно из принципа взрыва).

Обозначим за $\psi(x,p)$ формулу, выражающую в формальной арифметике рекурсивное
отношение Proof: $\langle \ulcorner\xi\urcorner,p\rangle \in \text{Proof}$, если $p$ --- гёделев номер
доказательства $\xi$. \pause\\
Обозначим $\pi(x)\equiv\exists p.\psi(x,p)$

Формулой \textbf{Consis} назовём формулу
$\neg \pi(\overline{\ulcorner 1=0 \urcorner})$

Будем говорить, что формула $\psi$, выражающая отношение Proof, 
формула $\pi$ и формула Consis соответствуют
\textbf{условиям выводимости Гильберта-Бернайса-Лёба}, если следующие условия выполнены для любой формулы $\alpha$:

\begin{enumerate}
\item $\vdash \alpha$ влечет $\vdash \pi(\overline{\ulcorner\alpha\urcorner})(\text{доказуема формула что доказуема } \alpha)$
\item $\vdash \pi (\overline{\ulcorner\alpha\urcorner}) \rightarrow \pi(\overline{\ulcorner\pi(\overline{\ulcorner\alpha\urcorner})\urcorner})$
\item $\vdash \pi (\overline{\ulcorner\alpha\rightarrow \beta\urcorner}) \rightarrow \pi(\overline{\ulcorner\alpha\urcorner}) \rightarrow \pi(\overline{\ulcorner\beta\urcorner})$
\end{enumerate}

\textbf{(Лемма об автоссылках)} Для любой формулы $\phi(x_1)$ можно построить 
такую замкнутую формулу $\alpha$ (не использующую неаксиоматических предикатных
и функциональных символов), что $\vdash \phi(\overline{\ulcorner\alpha\urcorner}) \leftrightarrow \alpha$ (найдется такая виртуалка, в которой программа выполнится так же, как и нативно).

\textbf{(Первая теорема Гёделя в УГБЛ)} Существует такая замкнутая формула $\gamma$, что если Ф.А. непротиворечива, то 
$\not\vdash \gamma$, а если Ф.А. $\omega$-непротиворечива, то и $\not\vdash\neg\gamma$.

\begin{proof}Рассмотрим $\phi(x_1) \equiv \neg\pi(x_1)$. Тогда по лемме об автоссылках существует
    $\gamma$, что $\vdash \gamma \leftrightarrow \neg\pi(\overline{\ulcorner\gamma\urcorner})$.
    
    \begin{itemize}
    \item Предположим, что $\vdash \gamma$. Тогда $\vdash \gamma \rightarrow \neg\pi(\overline{\ulcorner\gamma\urcorner})$, то есть $\not\vdash\gamma$
    \item Предположим, что $\vdash \neg\gamma$. Тогда $\vdash \pi(\overline{\ulcorner\gamma\urcorner})$,
    то есть $\vdash \exists p.\psi(\overline{\ulcorner\gamma\urcorner},p)$. Тогда по $\omega$-непротиворечивости
    найдётся $p$, что $\vdash \psi(\overline{\ulcorner\gamma\urcorner},\overline{p})$, то есть $\vdash \gamma$.
    \end{itemize}
\end{proof}

\textbf{(Вторая теорема Гёделя)} Если Consis доказуем, то формальная арифметика противоречива

\begin{proof}(неформально) \pause(Вкратце - если бы $\sigma(\overline{\ulcorner\sigma\urcorner})$ был доказуем, то получалось бы противоречие, 
    тк $\sigma(\overline{\ulcorner\sigma\urcorner})$ говорит о своей недоказуемости).
    Формулировка 1 теоремы Гёделя о неполноте арифметики: 
    <<если Ф.А. непротиворечива, то недоказуемо $\sigma(\overline{\ulcorner\sigma\urcorner})$>>. \pause
    То есть, $\forall p.\neg\omega_1(\overline{\ulcorner\sigma\urcorner},p)$. \pause То есть, 
    если $\text{Consis}$, то $\sigma(\overline{\ulcorner\sigma\urcorner})$. \pause
    То есть, если $\text{Consis}$, то $\sigma(\overline{\ulcorner\sigma\urcorner})$, --- и это можно доказать,
    то есть $\vdash\text{Consis}\rightarrow\sigma(\overline{\ulcorner\sigma\urcorner})$. \pause Однако
    если формальная арифметика непротиворечива, то $\not\vdash\sigma(\overline{\ulcorner\sigma\urcorner})$.
\end{proof}

\begin{proof}(формально) \pause
    \begin{enumerate}
        \item Пусть $\gamma$ таково, что $\vdash \gamma \leftrightarrow \neg\pi(\overline{\ulcorner\gamma\urcorner})$.
        \item Покажем $\pi(\overline{\ulcorner\gamma\urcorner})\vdash \pi(\overline{\ulcorner 1=0\urcorner})$. 
        
        \begin{enumerate}
        \item По условию 2, $\vdash \pi(\overline{\ulcorner\gamma\urcorner}) \rightarrow \pi(\overline{\ulcorner\pi(\overline{\ulcorner\gamma\urcorner})\urcorner})$.
        По теореме о дедукции $\pi(\overline{\ulcorner\gamma\urcorner})\vdash \pi(\overline{\ulcorner\pi(\overline{\ulcorner\gamma\urcorner})\urcorner})$;
        
        \item Так как $\vdash \pi(\overline{\ulcorner\gamma\urcorner})\rightarrow\neg\gamma$, то 
        по условию 1 $\vdash \pi(\overline{\ulcorner\pi(\overline{\ulcorner\gamma\urcorner})\rightarrow\neg\gamma\urcorner})$;
        
        \item По условию 3, $\pi(\overline{\ulcorner\gamma\urcorner})\vdash \pi(\overline{\ulcorner\pi(\overline{\ulcorner\gamma\urcorner})\rightarrow\neg\gamma\urcorner})\rightarrow
        \pi(\overline{\ulcorner\pi(\overline{\ulcorner\gamma\urcorner})\urcorner}) \rightarrow 
        \pi(\overline{\ulcorner\neg\gamma\urcorner})$;
        
        \item Таким образом, $\pi(\overline{\ulcorner\gamma\urcorner})\vdash\pi(\overline{\ulcorner\neg\gamma\urcorner})$;
        
        \item Однако $\vdash \gamma\rightarrow\neg\gamma\rightarrow 1=0$. Условие 3 (применить два раза) даст $\pi(\overline{\ulcorner\gamma\urcorner})\vdash \pi(\overline{\ulcorner 1=0 \urcorner})$.
        \end{enumerate}
        
        \item $\vdash \neg\pi(\overline{\ulcorner 1=0 \urcorner})\rightarrow\neg\pi(\overline{\ulcorner\gamma\urcorner})$ (т. о дедукции, контрапозиция).
        \item $\vdash \neg\pi(\overline{\ulcorner 1=0 \urcorner})\rightarrow\gamma$ (определение $\gamma$)(какой бы $\gamma$ мы не взяли, мы его можем доказать, что влечет противоречивость теории).
        \end{enumerate}
\end{proof}

$\text{Д}_\mathcal{S} = \{ \ulcorner\alpha\urcorner\ |\ \vdash_\mathcal{S}\alpha \}$; 
$\text{И}_\mathcal{S} = \{ \ulcorner\alpha\urcorner\ |\ \llbracket\alpha\rrbracket_\mathcal{S} = \text{И} \}$

(Лемма) Пусть $D(\ulcorner\alpha\urcorner) = \ulcorner\alpha(\overline{\ulcorner\alpha\urcorner})\urcorner$ для
любой формулы $\alpha(x)$. Тогда $D$ представима в формальной арифметике.

\textbf{Теорема} Если расширение Ф.А. $\mathcal{S}$ непротиворечиво и $D$ представима в нём, то $\text{Д}_\mathcal{S}$ невыразимо в $\mathcal{S}$

%TODO написать доказательство 
\begin{proof}Пусть $\delta(a,p)$ представляет $D$, и пусть $\sigma(x)$ выражает множество $\text{Д}_\mathcal{S}$ (рассматриваемое как
    одноместное отношение).
    
    Пусть $\alpha(x) := \forall p.\delta(x,p)\rightarrow\neg\sigma(p)$. Верно ли, что $\ulcorner\alpha\urcorner\in\text{Д}_\mathcal{S}$?
\end{proof}

(\textbf{Неразрешимость Ф.А.}) Если формальная арифметика непротиворечива, то формальная арифметика неразрешима

\begin{proof}
    Пусть формальная арифметика разрешима. 
    Значит, есть рекурсивная функция $f(x)$: $f(x)=1$ тогда и только тогда, 
    когда $x \in \text{Д}_\text{Ф.А.}$. То есть, $\text{Д}_\text{Ф.А.}$ выразимо в формальной арифметике.
    
    По теореме о невыразимости доказуемости, 
    $\text{Д}_\text{Ф.А.}$ невыразимо в формальной арифметике. Противоречие.
\end{proof}

(\textbf{Теорема Тарского})
Не существует формулы $\varphi(x)$, что $\llbracket \varphi(\overline{x}) \rrbracket = \text{И}$ (в стандартной интерпретации) тогда и только
тогда, когда $x \in \text{И}_\text{ФА}$.

\begin{proof}
    Пусть теория $\mathcal{S}$ --- формальная арифметика + аксиомы: все истинные в стандартной интерпретации формулы.
    Очевидно, что $\text{Д}_\mathcal{S} = \text{И}_\mathcal{S} = \text{И}_\text{ФА}$. 
    То есть $\text{И}_\text{ФА}$ невыразимо в $\mathcal{S}$.
    
    Пусть $\varphi$ таково, что $\llbracket\varphi(\overline{x})\rrbracket = \text{И}$ при $x \in \text{И}$.
    Тогда $\vdash\varphi(\overline{x})$, если $x \in \text{И}$ и $\vdash\neg\varphi(\overline{x})$, если $x \notin\text{И}$.
    
    Тогда $\text{И}$ выразимо в $\mathcal{S}$. Противоречие.
\end{proof}

Однако, если взять $D = \mathbb{R}$, истина становится выразима (алгоритм Тарского).

\section{ Теория множеств. Определения равенства. Парадокс брадобрея. Аксиоматика Цермело-Френкеля.
Конструктивные аксиомы (пустого, пары, объединения, множества подмножеств, выделения).
Частичный, линейный, полный порядок. Ординальные числа, аксиома бесконечности.
Конечные ординалы, предельные ординалы, доказательство существования ординала $\omega$,
операции над ординалами (варианты определения), факты об операциях над ординалами
(выполнены ли ассоциативность и коммутативность операций). Связь ординалов и упорядочений.}

\textbf{Парадокс Брадобрея} --- Пусть в некой деревне живёт брадобрей, который бреет всех жителей деревни, которые не бреются сами, и только их.
Бреет ли брадобрей сам себя? Аксиоматика теории множеств пытается запретить все опасные ситуации.

\textbf{Теория множеств} --- теория первого порядка,
с дополнительным нелогическим двухместным функциональным символом $\in$, и следующими 
дополнительными нелогическими аксиомами и схемами аксиом.

(Равенство <<по Лейбницу>>): объекты равны, если неразличимы. Если нечто ходит как утка, выглядит как 
утка и крякает как утка, то это утка.

(Принцип объёмности): объекты равны, если состоят из одинаковых частей.

 \textbf{Равенство: }\\
 $A \subseteq B \equiv \forall x.x \in A \rightarrow x \in B$ \\\pause
 $A = B \equiv A \subseteq B \with B \subseteq A$

\textbf{Аксиома пустого: }Существует пустое множество $\varnothing$. $$\exists s.\forall t.\neg t \in s$$ 

\textbf{Аксиома пары: }  Существует $\{a,b\}$.
Каковы бы ни были два множества $a$ и $b$, существует множество, состоящее 
в точности из них. 

$$\forall a.\forall b.\exists s.a \in s \with b \in s \with \forall c.c \in s \rightarrow c = a \vee c = b$$

\textbf{Аксимома объединения: } существует $\cup x$. 
Для любого непустого множества $x$ найдется такое множество, состоящее в точности
из тех элементов, из которых состоят элементы $x$. 
$$\forall x.(\exists y.y \in x) \rightarrow \exists p.\forall y.y \in p \leftrightarrow \exists s.y \in s \with s \in x$$

\textbf{Аксиома степени: }существует $\mathcal{P}(x)$.
Каково бы ни было множество $x$, существует множество, содержащее в точности
все возможные подмножества множества $x$.

$$\forall x.\exists p.\forall y.y \in p \leftrightarrow y \subseteq x$$

\textbf{Схема аксиом выделения: }существует $\{ t \in x\ |\ \varphi(t)\}$.
Для любого множества $x$ и любой формулы от одного аргумента $\varphi(y)$
($b$ не входит свободно в $\varphi$), найдется $b$, в которое
входят те и только те элементы из множества $x$, что $\varphi(y)$ истинно.

$$\forall x.\exists b.\forall y.y \in b \leftrightarrow (y \in x \with \varphi(y))$$

Упорядоченная пара множеств a и b --- $\{\{a\},\{a,b\}\}$, или $\langle{}a,b\rangle$

\textbf{Отношения нестрого порядка}
\begin{enumerate}
    \item Частичный: рефлексивность ($a \preceq a$), антисимметричность ($a \preceq b \rightarrow b \preceq a\rightarrow a=b$),
    транзитивность ($a \preceq b \rightarrow b \preceq c \rightarrow a \preceq c$).\pause
    \item Линейный: частичный + $\forall a.\forall b.a \preceq b \vee b \preceq a$.\pause
    \item Полный: линейный + в любом непустом подмножестве есть наименьший элемент.\pause
\end{enumerate}

\textbf{Полный строгий порядок}
$A$ вполне упорядочено отношением $(\prec)$, если:
\begin{enumerate}
\item при всех $a,b \in A$ выполнено либо $a \prec b$, либо $b \prec a$, либо $a = b$;
\item в любом $S \subseteq A$ и $S \ne \varnothing$ найдётся $n \in S$, что $\forall x.x \in S \rightarrow n = x \vee n \prec x$.
\end{enumerate}

Инкремент: $x' \equiv x \cup \{x\}$

\textbf{Аксиома бесконечности: }Существует $N: \varnothing \in N \with \forall x.x \in N\rightarrow x' \in N$\\

Транзитивное множество $X$: $\forall x.\forall y.x \in y \with y \in X \rightarrow x \in X$.

\textbf{Ординал} (порядковое число) --- вполне упорядоченное отношением $(\in)$ транзитивное множество.

\textbf{Предельный ординал}: такой $x$, что $x \ne \varnothing$ и нет $y: y' = x$

\textbf{Ординал $x$ конечный}, если он сам не предельный и нет предельного, меньшего его.

\textbf{$\omega$ --- наименьший предельный ординал.}\\

\textbf{(Теорема)} $\omega$ существует.
\begin{proof}Пусть $\omega = \{ x \in N\ |\ x\text{ конечен}\}$. Тогда:
    \begin{itemize}
    \item меньше $\omega$ предельных нет: если $\theta$ таков, что $\theta \in \omega$, тогда $\theta$ конечен.\pause
    \item $\omega$ предельный: Пусть $\theta$ таков, что $\theta' = \omega$. Тогда $\theta$ конечен и $\theta'$ тоже конечен.
    \end{itemize}
\end{proof}

Порядковый тип множества --- некоторое свойство, общее для всех множеств, 
изоморфных относительно биективных отображений, сохраняющих порядок.

Порядковый тип вполне упорядоченного множества $\langle S, (\preceq)\rangle$ --- ординал $A$, для которого есть биективное отображение $f: S \rightarrow A$, сохраняющее порядок:
$a \preceq b$ тогда и только тогда, когда $f(a) \le f(b)$

\textbf{Операции над ординалами}\\

$a + b$ --- порядковый тип $a \uplus b$ (отмеченного объединения), причём $x_a < y_b$ при любых
$x \in a$ и $y \in b$

$a \cdot b$ --- порядковый тип $a \times b$, произведение упорядочено лексикографически: $\langle x_1, y_1 \rangle < \langle x_2, y_2 \rangle$, 
если $y_1 < y_2$ или $y_1 = y_2$ и $x_1 < x_2$.

$\text{upb } x$ --- верхняя грань множества ординалов, $\text{upb }x = \bigcup_{a \in x} a$.

\textbf{Теорема}
$$a + b \equiv \left\{ \begin{array}{rl} 
   a, & b \equiv \varnothing\\
   (a + c)', & b \equiv c'\\
   \text{upb } \{ a+c \mid c \prec b \}, &\mbox{$b$ --- предельный ординал }\end{array}\right.$$

\textbf{Теорема}
$$a \cdot b \equiv \left\{ \begin{array}{rl} 
   0, & b \equiv \varnothing\\
   (a \cdot c) + a, & b \equiv c'\\
   \text{upb } \{ a \cdot c \mid c \prec b \}, &\mbox{$b$ --- предельный ординал }\end{array}\right.$$

\textbf{Определение}
$$a ^ b \equiv \left\{ \begin{array}{rl} 
   1, & b \equiv \varnothing\\
   (a ^ c) \cdot a, & b \equiv c'\\
   \text{upb } \{ a^c \mid c \prec b \}, &\mbox{$b$ --- предельный ординал }\end{array}\right.$$

\textbf{Ассоциативность есть, коммутативность отсутствует.}

$1 + \omega = \omega$, $\omega + 1 \ne \omega$.

\section{Аксиомы фундирования и подстановки. Кардинальные числа, мощность множеств, операции над
кардинальными числами (сложение, умножение, возведение в степень). Теорема Кантора-Бернштейна,
теорема Кантора. }

\subsection{Аксиома фундирования}
	\begin{dfn}Аксиома фундирования. 
		В каждом непустом множестве найдется элемент, не пересекающийся с исходным множеством.
		$$\forall x .x = \varnothing \vee \exists y .y \in x \with \forall z.z \in x \rightarrow z \notin y$$
	\end{dfn}
	
	Иными словами, в каждом множестве есть элемент, минимальный по отношению $(\in)$.
	
	Идея Рассела: каждому множеству припишем \emph{тип} (тип пустого 0, тип множеств 1,
	тип множеств множеств 2 и т.п.). Тогда конструкция невозможна: $\{ x\ |\ x \in x\}$.
	Аксиома фундирования позволяет определить функцию ранга:
	$$rk(x) = \text{upb }\{rk(y)\ |\ y\in x\}$$.

\subsection{Схема аксиом подстановки}
	\begin{dfn}Схема аксиом подстановки. 
		Пусть задана некоторая функция f, представимая в исчислении предикатов:
		то есть задана некоторая формула $\phi$, такая, что $f(x) = y$
		тогда и только тогда, когда $\phi(x,y) \with \exists ! z. \phi(x,z)$.
		Тогда для любого множества S существует множество f(S) --- образ
		множества S при отображении f.
		$$\forall s .(\forall x .\forall y_1 .\forall y_2 .x \in s \with \phi (x,y_1) \with \phi
		(x,y_2) \rightarrow y_1=y_2) \rightarrow 
		(\exists t .\forall y .y \in t
		\leftrightarrow \exists x . x \in s \with \phi (x,y)) $$
	\end{dfn}

\subsection{Кардинальные числа, мощность множества}
	\begin{dfn}Кардинальное число --- наименьший ординал, не равномощный никакому меньшему:
		$$\forall x.x \in c \rightarrow |x| < |c|$$\end{dfn}
	\begin{thm}Конечные ординалы --- кардинальные числа.\end{thm}
	
	\begin{dfn}Множество $A$ \emph{равномощно} $B$ $(|A|=|B|)$, если существует биекция
		$f: A \rightarrow B$.
		
		Множество $A$ имеет мощность, не превышающую мощности $B$ $(|A|\le|B|)$, если существует инъекция $f: A \rightarrow B$.
	\end{dfn}
	
	\begin{dfn}Мощность множества $(|S|)$ --- равномощное ему кардинальное число.\end{dfn}

\subsection{Теорема Кантора-Бернштейна}
	\begin{thm}Если $|A| \le |B|$ и $|B| \le |A|$, то $|A| = |B|$.\end{thm}
	Заметим, $f: A \rightarrow B$, $g: B \rightarrow A$ --- инъекции, но не обязательно $g(f(x)) = x$.
	\begin{proof}
		%Пусть $f: A \rightarrow B$ и $g: B \rightarrow A$. Построим биекцию в явном виде. %\begin{enumerate}
		
		Избавимся от множества $B$: пусть $A_0 = A$; $A_1 = g(B)$; $A_{k+2} = g(f(A_k))$.
		
		\vspace{-0.2cm}
		\begin{center}\tikz{
				\node[inner sep=0, outer sep=0] (A0) at (0,0) {};
				\node[inner sep=0, outer sep=0] (A1) at (2,0) {};
				\node[inner sep=0, outer sep=0] (A2) at (3,0) {};
				\node[inner sep=0, outer sep=0] (A3) at (3.5,0) {};
				\node[inner sep=0, outer sep=0] (A4) at (3.75,0) {};
				\node[inner sep=0, outer sep=0] (AN) at (4,0) {};
				\node[inner sep=0, outer sep=0] (AE) at (6,0) {};
				
				\node (B0) at (0,-1.5) {};
				\node (B1) at (2,-1.5) {};
				\node (B2) at (3,-1.5) {};
				\node (B3) at (3.5,-1.5) {};
				\node (B4) at (3.75,-1.5) {};
				\node (BN) at (4,-1.5) {};
				\node (BE) at (6,-1.5) {};
				
				\fill[gray!80] ($(A0)+(0,0.1)$) rectangle node[pos=0.1,above]{\color{gray} $A_0$} ($(AE)+(0,0.15)$);
				\fill[gray!30] ($(A1)+(0,0.2)$) rectangle node[pos=0.1,above]{\color{gray} $A_1$} ($(AE)+(0,0.25)$);
				\fill[gray!80] ($(A2)+(0,0.3)$) rectangle node[pos=0.1,above]{\color{gray} $A_2$} ($(AE)+(0,0.35)$);
				\fill[gray!30] ($(A3)+(0,0.4)$) rectangle node[pos=0.1,above]{\color{gray} $A_3$} ($(AE)+(0,0.45)$);
				\fill[gray!80] ($(AN)+(0,0.5)$) rectangle node[midway,above]{\color{gray} $\cap A_k$} ($(AE)+(0,0.55)$);
				%\draw (A3) -- node[midway,above]{$\dots$} (AN) (AE);
				%\draw (B0) to (BE);
				
				%\draw (AN) to (BN);
				
				\fill[gray!80] ($(B0)-(0,0.1)$) rectangle node[pos=0.1,below]{\color{gray} $B_0$} ($(BE)-(0,0.15)$);
				\fill[gray!80] ($(B1)-(0,0.1)$) rectangle node[pos=0.1,below]{\color{gray} $B_1$} ($(BE)-(0,0.15)$);
				\fill[gray!80] ($(B2)-(0,0.1)$) rectangle node[pos=0.1,below]{\color{gray} $B_2$} ($(BE)-(0,0.15)$);
				\fill[gray!80] ($(B3)-(0,0.1)$) rectangle node[pos=0.1,below]{\color{gray} $B_3$} ($(BE)-(0,0.15)$);
				\fill[gray!80] ($(BN)-(0,0.1)$) rectangle node[midway,below]{\color{gray} $\cap B_k$} ($(BE)-(0,0.15)$);
				
				\fill[gray!30] (4,0) -- (6,0) -- (6,-1.5) -- (4,-1.5);
				
				\fill[gray!30] (3,-1.5) -- (3.5,-1.5) -- (3.75,0) -- (3.5,0);
				\fill[gray!30] (0,-1.5) -- (2,-1.5) -- (3,0) -- (2,0);
				\fill[gray!30] (3.75,-1.5) -- (3.875,-1.5) -- (3.875+0.0625,0) -- (3.875,0);
				\fill[gray!80] (0,0) -- (2,-1.5) -- (3,-1.5) -- (2,0);
				\fill[gray!80] (3,0) -- (3.5,-1.5) -- (3.75,-1.5) -- (3.5,0);
				\fill[gray!80] (3.75,0) -- (3.875,-1.5) -- (3.875+0.0625,-1.5) -- (3.875,0);
				%\draw[dashed,->] (B3) -- (A4);
				
		}\end{center}
		
		\vspace{-0.4cm}
		
		Тогда, если существует $h: A_0 \rightarrow A_1$ --- биекция, то тогда $g^{-1}\circ h: A \rightarrow B$ --- 
		требуемая биекция.
		
		%\item Построим биекцию $h: A_0 \rightarrow A_1$\end{enumerate}
\end{proof}

\subsection{Построение биекции $h: A_0 \rightarrow A_1$}
Пусть $C_k = A_k \setminus A_{k+1}$. Тогда $g(f(C_k)) = g(f(A_k))\setminus g(f(A_{k+1})) = A_{k+2}\setminus A_{k+3} = C_{k+2}$.

\begin{center}\tikz{
		\node[inner sep=0, outer sep=0] (A0) at (0,0) {};
		\node[inner sep=0, outer sep=0] (A1) at (2,0) {};
		\node[inner sep=0, outer sep=0] (A2) at (3,0) {};
		\node[inner sep=0, outer sep=0] (A3) at (3.5,0) {};
		\node[inner sep=0, outer sep=0] (A4) at (3.75,0) {};
		\node[inner sep=0, outer sep=0] (AN) at (4,0) {};
		\node[inner sep=0, outer sep=0] (AE) at (6,0) {};
		
		\node (B0) at (0,-1.5) {};
		\node (B1) at (2,-1.5) {};
		\node (B2) at (3,-1.5) {};
		\node (B3) at (3.5,-1.5) {};
		\node (B4) at (3.75,-1.5) {};
		\node (BN) at (4,-1.5) {};
		\node (BE) at (6,-1.5) {};
		
		\fill[gray!80] ($(A0)+(0,0.1)$) rectangle node[pos=0.1,above]{\color{gray} $C_0$} ($(AE)+(0,0.15)$);
		\fill[gray!30] ($(A1)+(0,0.1)$) rectangle node[pos=0.1,above]{\color{gray} $C_1$} ($(AE)+(0,0.15)$);
		\fill[gray!80] ($(A2)+(0,0.1)$) rectangle node[pos=0.1,above]{\color{gray} $C_2$} ($(AE)+(0,0.15)$);
		\fill[gray!30] ($(A3)+(0,0.1)$) rectangle node[pos=0.1,above]{\color{gray} $C_3$} ($(AE)+(0,0.15)$);
		\fill[gray!80] ($(AN)+(0,0.1)$) rectangle node[midway,above]{\color{gray} $\cap A_k$} ($(AE)+(0,0.15)$);
		\fill[white] ($(A4)+(0,0.1)$) rectangle ($(AN)+(0,0.15)$);
		%\draw (A3) -- node[midway,above]{$\dots$} (AN) (AE);
		%\draw (B0) to (BE);
		
		%\draw (AN) to (BN);
		
		%\fill[gray!80] ($(B0)-(0,0.1)$) rectangle node[pos=0.1,below]{\color{gray} $B_0$} ($(BE)-(0,0.15)$);
		%\fill[gray!80] ($(B1)-(0,0.1)$) rectangle node[pos=0.1,below]{\color{gray} $B_1$} ($(BE)-(0,0.15)$);
		%\fill[gray!80] ($(B2)-(0,0.1)$) rectangle node[pos=0.1,below]{\color{gray} $B_2$} ($(BE)-(0,0.15)$);
		%\fill[gray!80] ($(B3)-(0,0.1)$) rectangle node[pos=0.1,below]{\color{gray} $B_3$} ($(BE)-(0,0.15)$);
		%\fill[gray!80] ($(BN)-(0,0.1)$) rectangle node[midway,below]{\color{gray} $\cap B_k$} ($(BE)-(0,0.15)$);
		
		\fill[gray!30] (4,0) -- (6,0) -- (6,-1.5) -- (4,-1.5);
		
		\fill[gray!30] (3,-1.5) -- (3.5,-1.5) -- (3.75,0) -- (3.5,0);
		\fill[gray!30] (0,-1.5) -- (2,-1.5) -- (3,0) -- (2,0);
		\fill[gray!30] (3.75,-1.5) -- (3.875,-1.5) -- (3.875+0.0625,0) -- (3.875,0);
		\fill[gray!80] (0,0) -- (2,-1.5) -- (3,-1.5) -- (2,0);
		\fill[gray!80] (3,0) -- (3.5,-1.5) -- (3.75,-1.5) -- (3.5,0);
		\fill[gray!80] (3.75,0) -- (3.875,-1.5) -- (3.875+0.0625,-1.5) -- (3.875,0);
		%\draw[dashed,->] (B3) -- (A4);
		
}\end{center}

Тогда определим $h(x)$ следующим образом:

\tikz{
	\node (F) at (-3,-1) {$h(x) = \left\{\begin{array}{ll} x, & x \in C_{2k+1} \vee x \in \cap A_k\\
			g(f(x)), & x \in C_{2k}\end{array}\right.$};
	
	
	\node[inner sep=0, outer sep=0] (A0) at (0,0) {};
	\node[inner sep=0, outer sep=0] (A1) at (2,0) {};
	\node[inner sep=0, outer sep=0] (A2) at (3,0) {};
	\node[inner sep=0, outer sep=0] (A3) at (3.5,0) {};
	\node[inner sep=0, outer sep=0] (A4) at (3.75,0) {};
	\node[inner sep=0, outer sep=0] (AN) at (4,0) {};
	\node[inner sep=0, outer sep=0] (AE) at (6,0) {};
	
	\node (B0) at (0,-1.5) {};
	\node (B1) at (2,-1.5) {};
	\node (B2) at (3,-1.5) {};
	\node (B3) at (3.5,-1.5) {};
	\node (B4) at (3.75,-1.5) {};
	\node (BN) at (4,-1.5) {};
	\node (BE) at (6,-1.5) {};
	
	\fill[gray!80] ($(A0)+(0,0.1)$) rectangle node[pos=0.1,above]{\color{gray} $C_0$} ($(AE)+(0,0.15)$);
	\fill[gray!30] ($(A1)+(0,0.1)$) rectangle node[pos=0.1,above]{\color{gray} $C_1$} ($(AE)+(0,0.15)$);
	\fill[gray!80] ($(A2)+(0,0.1)$) rectangle node[pos=0.08,above]{\color{gray} $C_2$} ($(AE)+(0,0.15)$);
	\fill[gray!30] ($(A3)+(0,0.1)$) rectangle node[pos=0.08,above]{\color{gray} $C_3$} ($(AE)+(0,0.15)$);
	\fill[gray!80] ($(AN)+(0,0.1)$) rectangle node[midway,above]{\color{gray} $\cap A_k$} ($(AE)+(0,0.15)$);
	\fill[white] ($(A4)+(0,0.1)$) rectangle ($(AN)+(0,0.15)$);
	%\draw (A3) -- node[midway,above]{$\dots$} (AN) (AE);
	%\draw (B0) to (BE);
	
	%\draw (AN) to (BN);
	
	%\fill[gray!80] ($(B0)-(0,0.1)$) rectangle node[pos=0.1,below]{\color{gray} $C_0$} ($(BE)-(0,0.15)$);
	\fill[gray!30] ($(B1)-(0,0.1)$) rectangle node[pos=0.1,below]{\color{gray} $C_1$} ($(BE)-(0,0.15)$);
	\fill[gray!80] ($(B2)-(0,0.1)$) rectangle node[pos=0.08,below]{\color{gray} $C_2$} ($(BE)-(0,0.15)$);
	\fill[gray!30] ($(B3)-(0,0.1)$) rectangle node[pos=0.08,below]{\color{gray} $C_3$} ($(BE)-(0,0.15)$);
	\fill[gray!80] ($(BN)-(0,0.1)$) rectangle node[midway,below]{\color{gray} $\cap A_k$} ($(BE)-(0,0.15)$);
	\fill[white] ($(B4)-(0,0.1)$) rectangle ($(BN)-(0,0.15)$);
	
	\fill[gray!30] (4,0) -- (6,0) -- (6,-1.5) -- (4,-1.5);
	\fill[gray!30] (2,-1.5) -- (3,-1.5) -- (3,0) -- (2,0);
	%\fill[gray!30] (0,-1.5) -- (2,-1.5) -- (3.5,0) -- (3,0);
	\fill[gray!30] (3.5,-1.5) -- (3.75,-1.5) -- (3.75,0) -- (3.5,0);
	\fill[gray!80] (0,0) -- (3,-1.5) -- (3.5,-1.5) -- (2,0);
	\fill[gray!80] (3,0) -- (3.75,-1.5) -- (3.875,-1.5) -- (3.5,0);
	%\fill[gray!80] (3.75,0) -- (3.875,-1.5) -- (3.875+0.0625,-1.5) -- (3.875,0);
	%\draw[dashed,->] (B3) -- (A4);
	
}

\subsection{Теорема Кантора}
	\begin{thm}$|\mathcal{P}(S)| > |S|$\end{thm}
	\begin{proof}Пусть $S = \{a,b,c,\dots\}$
		
		\begin{center}\begin{tabular}{c|cccl}
				$n$ & $a \in f(n)$ & $b \in f(n)$ & $c \in f(n)$ & $\dots$ \\\hline
				$a$ & \color{red}И & Л & И \\
				$b$ &       Л & \color{red}Л& И \\
				$c$ &   И & И & \color{red}И\\\hline
				& Л & И & Л & $y \notin f(y)$
		\end{tabular}\end{center}
		
		Пусть $f: S \rightarrow \mathcal{P}(S)$ --- биекция. Тогда 
		$\sigma = \{ y\in S\ |\ y\notin f(y)\}$. Пусть $f(x) = \sigma$.
		Но $x \in f(x)$ тогда и только тогда, когда $x \notin \sigma$, то есть $f(x) \ne \sigma$.
	\end{proof}


    \section{Мощность модели. Элементарные подмодели. Теорема Лёвенгейма-Сколема, парадокс Сколема.}


    \begin{dfn} Пусть задана модель $\langle D, F_n, P_n \rangle$ для некоторой теории первого порядка.
    Её мощностью будем считать мощность $D$.
    \end{dfn}


    \begin{dfn}$\mathcal{M}' = \langle D', F'_n, P'_n \rangle$ --- элементарная подмодель $\mathcal{M} = \langle D, F_n, P_n \rangle$,
    если: \begin{enumerate}
              \item $D' \subseteq D$,  $F'_n$, $P'_n$ --- сужение $F_n$, $P_n$ (замкнутое на $D'$). \item $\mathcal{M}\models \varphi(x_1,\dots,x_n)$ тогда и только тогда, когда $\mathcal{M}'\models \varphi(x_1,\dots,x_n)$
              при $x_i \in D'$. \end{enumerate}
    \end{dfn}

    \begin{exm}Когда сужение $M$ не является элементарной подмоделью?
        $\forall x.\exists y.x \ne y$. Истинно в $\mathbb{N}$.  Но пусть $D' = \{ 0 \}$.
    \end{exm}


    \begin{thm}Пусть $T$ --- множество всех формул теории первого порядка.
    Пусть теория имеет некоторую модель $\mathcal{M}$.
    Тогда найдётся элементарная подмодель $\mathcal{M'}$, причём $|\mathcal{M'}| \leq \max(\aleph_0, |T|)$.
    \end{thm}
    \begin{proof} (Схема доказательства)
\begin{enumerate}
    \item Построим $D_0$ --- множество всех значений, которые упомянуты в языке теории. \item Будем последовательно пополнять $D_i$: $D_0 \subseteq D_1 \subseteq D_2 \dots$, следя за мощностью.
    $D' = \cup D_i$.
    \item Покажем, что $\langle D', F_n, P_n\rangle$ --- требуемая подмодель.
\end{enumerate}
    \end{proof}


    Пусть $\{f^0_k\}$ --- все 0-местные функциональные символы теории. \begin{enumerate}
                                                                           \item $D_0 = \{ \llbracket f^0_k \rrbracket \}$, если есть хотя бы один $f^0_k$. \item Если таких $f^0_k$ нет, возьмём какое-нибудь одно значение из $D$. \end{enumerate}
    Очевидно, $|D_0| \le |T|$.



    Фиксируем некоторый $D_k$. Напомним, $T$ --- множество всех формул теории. Рассмотрим $\varphi \in T$.\begin{enumerate}
                                                                                                              \item $\varphi$ не имеет свободных переменных --- пропустим. \item $\varphi$ имеет хотя бы одну свободную переменную $y$. \begin{enumerate}
                                                                                                                                                                                                                                            \item $\varphi (y, x_1, \dots, x_n)$ при $y,x_i \in D_k$ бывает истинным и ложным --- ничего не меняем \item $\varphi (y, x_1, \dots, x_n)$ при $y \in D$ и $x_i \in D_k$ либо всегда истинен, либо всегда ложен --- ничего не меняем \item $\varphi (y, x_1, \dots, x_n)$ при $y,x_i \in D_k$ тождественно истинен или ложен, но при
                                                                                                                                                                                                                                            $y' \in D \setminus D_k$ отличается --- добавим $y'$ к $D_{k+1}$. Вместе добавим всевозможные $\llbracket\theta(y')\rrbracket$.
                                                                                                              \end{enumerate}
    \end{enumerate}


    \begin{enumerate}
        \item Всего добавили не больше $|T| \cdot |T|$ (для каждой формулы $\varphi$, возможно, будет добавлен $y$ ---
        и всевозможные выражения $\theta(y)$, допустимые в языке), и $|D_0| \le |T| \le |T|\cdot|T|$,
        отсюда $|D_k| \le |T| \cdot |T|$.
        \item $|D'| = |\bigcup D_i| \le |T| \cdot |T| \cdot \aleph_0$.
        \item Тогда $|T| \cdot |T| \cdot \aleph_0 = \max(|T|,\aleph_0)$. Разберём случаи:

        \begin{enumerate}
            \item Если $|T| < \aleph_0$, тогда $(|T| \cdot |T|) \cdot \aleph_0 = \aleph_0$
            \item Если $|T| \ge \aleph_0$, тогда $(|T| \cdot |T|) \cdot \aleph_0 = |T| \cdot \aleph_0 = |T|$.

        \end{enumerate}
        \item Итого, $|D'| \le \max(|T|,\aleph_0)$.
    \end{enumerate}



    Индукцией по структуре формул $\tau \in T$ покажем,
    что все формулы можно вычислить, и что $\llbracket \varphi \rrbracket_\mathcal{M'} = \llbracket \varphi \rrbracket_\mathcal{M}$.
    \begin{enumerate}
        \item База, 0 связок. $\tau \equiv P(f_1(x_1,\dots,x_n),\dots,f_n(x_1,\dots,x_n))$.  Если $x_i \in D'$, то значит,
        добавлены на некоторых шагах (максимальный пусть $t$). Поэтому в $D_{t+1}$ можно вычислить формулу, и её значение сохранилось. \item Переход. Пусть формулы из $k$ связок сохраняют значения. Рассмотрим $\tau$ с $k+1$ связкой. \begin{enumerate}
                                                                                                                                                                                                                                             \item $\tau \equiv \rho \star \sigma$ --- очевидно. \item $\tau\equiv\forall y.\varphi(y,x_1,\dots,x_n)$. Каждый $x_i$ добавлен на каком-то шаге --- максимум $t$. Если $\varphi(y,x_1,\dots,x_n)$ бывает истинен и ложен при $y_t, y_f \in D$, то $y_t, y_f \in D_{t+1}$ (по построению). Поэтому, если $\mathcal{M}\not\models\forall y.\varphi(y,x_1,\dots,x_n)$, то и
                                                                                                                                                                                                                                             $\mathcal{M'}\not\models\forall y.\varphi(y,x_1,\dots,x_n)$. Если же $\varphi(y,x_1,\dots,x_n)$ не меняется от $y$, то тем более
                                                                                                                                                                                                                                             $\llbracket \varphi \rrbracket_\mathcal{M'} = \llbracket \varphi \rrbracket_\mathcal{M}$. \item $\tau\equiv\exists y.\varphi(y,x_1,\dots,x_n)$ --- аналогично.
        \end{enumerate}
    \end{enumerate}



    \begin{enumerate}
        \item Как известно, $|\mathbb{R}| = |\mathcal{P}(\mathbb{N})| > |\mathbb{N}| = \aleph_0$.  Однако, ZFC --- теория со счётным
        количеством формул. Значит, существует счётная модель ZFC, то есть $|\mathbb{R}| = \aleph_0$.  В чём ошибка? \item У равенств разный смысл, первое --- в предметном языке, второе --- в метаязыке.
    \end{enumerate}

    \section{Аксиома выбора, альтернативные формулировки (лемма Цорна, теорема Цермело, существование
    частичной обратной), доказательство переходов (кроме доказательства леммы Цорна).}

    \begin{dfn}Дизъюнктное (разделённое) множество --- множество, элементы которого
        не пересекаются. 
        $$Dj(x) \equiv \forall y.\forall z.(y \in x \with z \in x \with \neg y=z) \rightarrow 
        \neg \exists t.t \in y \with t \in z$$
        \end{dfn}

    \begin{axm}[выбора]
Из любого семейства дизъюнктных непустых множеств $\mathcal{A}$ можно выбрать непустую трансверсаль ---
множество $S$, что $|S \cap A| = 1$ для каждого $A\in\mathcal{A}$. Иначе, $S \in \times \mathcal{A}$.
    \end{axm}

    \begin{thm}[функциональный вариант аксиомы выбора]
Пусть $\mathcal{A}$ --- семейство непустых множеств. Тогда существует
$f : \mathcal{A} \rightarrow \cup \mathcal{A}$, причём $\forall a.a \in \mathcal{A} \rightarrow f(a) \in a$
    \end{thm}

    \begin{proof}
        Пусть $X(A) = \{ \langle A, a \rangle \ |\ a \in A \}$,
        по семейству $\mathcal{A}$ рассмотрим $\{X(A)\ |\ A\in\mathcal{A}\}$
        \begin{itemize}
            \item непустых: если $A\in\mathcal{A}$, $A \ne \varnothing$, то $X(A) \ne \varnothing$;
            \item дизъюнктное: если $A_0,A_1\in\mathcal{A}$, $A_0 \ne A_1$, то $X(A_0) \cap X(A_1) = \varnothing$
        \end{itemize}
        тогда по аксиоме выбора $\exists f.f \in \times \mathcal{A}$.
    \end{proof}
    Обратное утверждение также легко показать.

    \subsection{Аксиома выбора: альтернативные формулировки}

    \begin{thm}[Лемма Цорна]
Если задано $\langle M, (\preceq) \rangle$ и для всякого линейно упорядоченного $S \subseteq M$ выполнено
$\text{upb}_M S \ne \varnothing$, то в $M$ существует максимальный элемент.
    \end{thm}
    \begin{thm}[Теорема Цермело]
На любом множестве можно задать полный порядок.
    \end{thm}
    \begin{thm}
        У любой сюръективной функции существует частичная обратная.
    \end{thm}

    \begin{thm}
        Аксиома выбора $\Rightarrow$ лемма Цорна: без доказательства
    \end{thm}

    \subsection{Начальный отрезок}

    \begin{dfn}
        Назовём (для данного раздела) упорядоченным множеством пару $\langle S, (\prec_S)\rangle$.
        Отношение порядка $(\prec_S)$ может быть как строгим, так и нестрогим.
        Будем говорить, что $\langle S, (\prec_S)\rangle$ --- начальный отрезок $\langle T, (\prec_T) \rangle$,
        если:\begin{itemize}
                 \item $S \subseteq T$;
                 \item если $a,b \in S$, то $a \prec_S b$ тогда и только тогда, когда $a \prec_T b$;
                 \item если $a \in S$, $b \in T\setminus S$, то $a \prec_T b$.
        \end{itemize}
        Будем обозначать это как $\langle S, (\prec_S)\rangle\sqsubseteq\langle T, (\prec_T)\rangle$ или как $S \sqsubseteq T$, если порядок на множествах понятен из контекста.
    \end{dfn}

    \begin{thm}
        Отношение <<быть начальным отрезком>> является отношением нестрогого порядка.
    \end{thm}

    \subsection{Верхняя грань семейства упорядоченных множеств}
    \begin{thm}[о верхней грани]
Если семейство упорядоченных множеств $X$ линейно упорядочено отношением <<быть начальным отрезком>>, то у него есть верхняя грань.
    \end{thm}

    \begin{proof}
        Пусть $M = \cup \{ T | \langle T, (\prec) \rangle \in X \}$ и
        $(\prec)_M = \cup \{ (\prec) | \langle T, (\prec) \rangle \in X \}$.

        Покажем, что если $\langle A, (\prec_A)\rangle \in X$, то $A \sqsubseteq M$. Рассмотрим определение:
        \begin{itemize}
            \item $A \subseteq M$ --- выполнено по построению $M$;
            \item если $a,b \in A$, то $a \prec_A b$ влечёт $a \prec_M b$ (по построению $M$). Если же $a \prec_M b$, но $a \not\prec_A b$,
            то существует $A'$, что $a,b \in A'$ и $a \prec_{A'} b$. Тогда $A\not\sqsubseteq A'$ и $A'\not\sqsubseteq A$, что невозможно
            по линейности порядка;
            \item если $a \in A$, $b \in M\setminus A$, то найдётся $B$, что $b\in B$, отчего $a \prec_B b$ (так как $A \sqsubseteq B$)
            и $a \prec_M b$ (по построению $M$).
        \end{itemize}
        Тогда $\langle M, (\prec_M)\rangle$ --- требуемая верхняя грань.
    \end{proof}

    \subsection{Лемма Цорна $\Rightarrow$ теорема Цермело}
    Пусть выполнена лемма Цорна и дано некоторое $X$. Покажем, что на нём можно ввести полный порядок.
    \begin{itemize}
        \item Пусть $S = \{ \langle P, (\prec)\rangle \ |\ P \subseteq X, (\prec)\text{ --- полный порядок} \}$.
            {\color{gray}Например, для $X = \{0,1\}$ множество
        $S = \{
        \langle\varnothing,\varnothing\rangle,
        \langle \{0\},\varnothing\rangle,
        \langle\{1\},\varnothing\rangle,
        \langle X, 0 \prec 1\rangle,
        \langle X, 1 \prec 0\rangle
        \}$}

        \item Введём порядок на $S$ как $(\sqsubseteq)$. Заметим, что это --- частичный, но не линейный порядок.
            {\color{gray}Например, $\langle X, 0 \prec 1\rangle$ несравним с $\langle X, 1 \prec 0\rangle$.}

        \item По теореме о верхней грани любое линейно упорядоченное подмножество
        $\langle T, (\sqsubseteq) \rangle$ (где $T \subseteq S$) имеет
        верхнюю грань.

            {\color{gray}Например,
        для $\{\langle\varnothing,\varnothing\rangle,
        \langle \{0\},\varnothing\rangle,
        \langle X, 0 \prec 1\rangle\}$ это $\langle X, 0 \prec 1\rangle$.}

        \item По лемме Цорна тогда есть $\langle R, (\sqsubseteq_R)\rangle = \max S$. Заметим, что $R = X$, потому что иначе пусть
        $a \in X\setminus R$. Тогда положив $M = \langle R\cup\{a\}, (\sqsubseteq_R)\cup\{x\prec a\ |\ x \in R\} \rangle$
        получим, что $M$ тоже вполне упорядоченное (и потому $M \in S$), значит, $R$ не максимальное.
    \end{itemize}

    \subsection{Теорема Цермело $\Rightarrow$ существование обратной $\Rightarrow$ аксиома выбора}
    \begin{thm}Теорема Цермело $\Rightarrow$ у сюръективных функций существует частичная обратная.\end{thm}
    \begin{proof}
        Рассмотрим сюръективную $f: A \rightarrow B$. Рассмотрим семейство $R_b = \{ a \in A\ |\ f(a) = b \}$.
        Построим полный порядок на каждом из $R_b$. Тогда $f^{-1}(b) = \min R_b$.
    \end{proof}
    \begin{thm}Существует частичная обратная у сюръективных функций $\Rightarrow$ существует трансверсаль у семейства непустых дизъюнктных множеств.\end{thm}
    \begin{proof}
        Пусть дано семейство непустых дизъюнктных множеств $\mathcal{A}$.
        Рассмотрим $f: \cup \mathcal{A} \rightarrow \mathcal{A}$, что
        $f(a) = \cup\{ A \in \mathcal{A}\ |\ a \in A \}$. Поскольку элементы $\mathcal{A}$ дизъюнктны,
        $f(a) \in \mathcal{A}$ при всех $a$. Тогда существует $f^{-1}: \mathcal{A} \rightarrow \cup\mathcal{A}$. Тогда
        $\{ f^{-1}(A)\ |\ A\in\mathcal{A} \} \in \times \mathcal{A}$.
    \end{proof}


    \section{Применение аксиомы выбора: эквивалентность определений пределов (по Коши и по Гейне).
    Теорема Диаконеску. Ослабленные варианты (счётный выбор и зависимый выбор), универсум фон Неймана.
    Аксиома конструктивности.}

    \begin{dfn}Пределом функции $f$ в точке $x_0$ по \emph{Коши} называется такой $y$, что
        $$\forall \varepsilon\in\mathbb{R}^+.\exists \delta\in\mathbb{R}^+.\forall x.|x-x_0| < \delta \rightarrow |f(x) - y| < \varepsilon$$
    \end{dfn}

    \vspace{-0.5cm}
    \begin{dfn}Пределом функции $f$ в точке $x_0$ по \emph{Гейне} называется такой $y$, что
    для любой $x_n \rightarrow x_0$ выполнено $f(x_n) \rightarrow y$.
    \end{dfn}


    \begin{thm}
        Если $\lim\limits_{x \rightarrow x_0}f(x) = y$ по Гейне, то
        $\forall \varepsilon>0.\exists \delta>0.\forall x.|x-x_0|<\delta \rightarrow |f(x)-y| < \varepsilon$.
    \end{thm}

    \begin{proof}
        Пусть не так:
        $\exists \varepsilon>0.\forall \delta>0.\exists x_\delta.|x_\delta-x_0|<\delta \with |f(x_\delta)-y| \ge \varepsilon$.
        Фиксируем $\varepsilon$ и возьмём $\delta_n = \frac{1}{n}$ и $p_n = x_{\delta_n}$.
        $p_n \rightarrow x_0$, так как $|x_\frac{1}{n} - x_0| < \frac{1}{n}$,
        по определению предела по Гейне $f(p_n) \rightarrow y$,
        но по предположению $\forall n\in\mathbb{N}.|f(p_n) - y| \ge \varepsilon$.
    \end{proof}
    \begin{snote}
        Для применения предела по Гейне нужна $p_n$: то есть $p: \mathbb{N}\rightarrow\mathbb{R}$.
%где $\Gamma(p) \subseteq \mathbb{N}\times\mathbb{R}$
%$\langle x_\frac{1}{1}: |x_\frac{1}{1}-x_0|<1 \with |f(x_\frac{1}{1})-y| \ge \varepsilon$; $x_\frac{1}{2}: |x_\frac{1}{2}-x_0|<\frac{1}{2} \with |f(x_\frac{1}{2})-y| \ge \varepsilon; ...\rangle$
%
%\vspace{0.3cm}
        ... %$\exists \varepsilon.\forall \delta.\exists x_\delta.|x_\delta-x_0|<\delta \with |f(x_\delta)-y| \ge \varepsilon$.\\
        Фиксируем $\varepsilon$ и рассмотрим $X_\delta = \{ x \ |\ |x-x_0| <\delta \with |f(x)-y| \ge \varepsilon\}$.
        Отрицание предела по Коши означает, что $X_\delta \ne \varnothing$ при любом $\delta > 0$.
%Возьмём $\delta_n = \frac{1}{n}$ и $x_{\frac{1}{n}} \in X_\frac{1}{n}$.

        ... То есть, по семейству $Q:=\{ X_1, X_\frac{1}{2}, X_\frac{1}{4}, \dots \}$
        по аксиоме выбора построим $q: Q \rightarrow \cup Q$, что $q(X_\frac{1}{n}) \in X_\frac{1}{n}$.
        Далее, взяв композицию $p_n := q(X_{\delta_n})$, получаем $p_n \rightarrow x_0$, что $\forall n\in\mathbb{N}.|f(p_n) - y| \ge \varepsilon$.
    \end{snote}



    \begin{thm}Пусть $\lim\limits_{x \rightarrow x_0} f(x) = y$ и дана $x_n \rightarrow x_0$.
    Тогда $f(x_n) \rightarrow y$.\end{thm}

    \begin{proof}
%Пусть $\lim_{x \rightarrow x_0} f(x) = y$ и дана $x_n \rightarrow x_0$.
%То есть, $\forall \varepsilon>0.\exists \delta>0.\forall x.|x-x_0| < \delta \rightarrow |f(x) - y| < \varepsilon$
        Фиксируем $\varepsilon > 0$.
        \begin{itemize}
%\item Заметим, что $\forall $ и $|x_n - x_0| < \delta\rightarrow|f(x_n) - y| < \varepsilon$.
            \item $\exists \delta > 0.\exists N\in\mathbb{N}.(\forall x.|x - x_0| < \delta \rightarrow |f(x) - y| < \varepsilon) \with
            (\forall n\in\mathbb{N}.n > N \rightarrow |x_n - x_0|<\delta)$
            \item $(\forall x.|x - x_0| < \delta \rightarrow |f(x) - y| < \varepsilon) \rightarrow (|x_n - x_0| < \delta \rightarrow |f(x_n) - y| < \varepsilon)$ \ (сх. 11).
            \item $\exists \delta > 0.\exists N\in\mathbb{N}.\forall n\in\mathbb{N}.n > N\rightarrow |f(x_n) - y| < \varepsilon$.
            \item Поскольку $\delta$ не используется в формуле, $\exists \delta > 0$ можно устранить.
            \item Отсюда $\exists N\in\mathbb{N}.\forall n\in\mathbb{N}.n > N\rightarrow |f(x_n) - y| < \varepsilon$
        \end{itemize}
    \end{proof}
    Почему здесь не потребовалась аксиома выбора? Потому что нам нужен единственный $\delta$, а для него ---
    единственный $N$


    \begin{exm}
        Пусть $A_0 = \{0,1,3,5\}$ и $A_1 = \{3,5,1,0,0,5,3\}$.
        Верно ли, что $A_0 = A_1$?
        Да, так как $\forall x.x \in \{0,1,3,5\} \leftrightarrow x \in \{3,5,1,0,0,5,3\}$.\end{exm}
    \begin{thm}[конгруэнтность]
Если $f: A \rightarrow B$, также $a,b\in A$ и $a=b$, то $f(a) = f(b)$.
    \end{thm}

    \begin{proof}
        Пусть $F \subseteq A\times B$ --- график функции $f$.

%Легко показать, что если $a=b$ и $y_1 = y_2$, то $\langle a, y_1\rangle = \langle b,y_2\rangle$.\\
%Значит (по аксиоме равенства), $\langle a,x\rangle \in F$ тогда и только тогда,
%когда $\langle b,x\rangle \in F$.
        По определению функции, $\forall x.\forall y_1.\forall y_2.\langle x,y_1\rangle \in F \with \langle x,y_2 \rangle \in F \rightarrow y_1 = y_2$.\\
        Также, если $f(a) = y_1$, $f(b) = y_2$, то $\langle a,y_1 \rangle \in F$ и $\langle b,y_2 \rangle \in F$.\\
        Тогда: $\langle a,y_1\rangle = \langle b,y_1\rangle = \langle b,y_2 \rangle = \langle a,y_2\rangle$,
        то есть $f(a) = y_2 = f(b)$.

%Пусть $\langle a,x \rangle \in F$ (поскольку $f$ --- функция, такое $x$ должно существовать).
%Тогда из $a=b$ следует $\langle b,x \rangle = \langle a,x \rangle$ (по свойствам упорядоченной пары), значит, $f(b) = x$.
    \end{proof}
% следует $f(A_0) = f(A_1)$
%по определению функционального бинарного отношения:
%$$\forall x.\exists y.F(x, y) \with \forall y_0.\forall y_1.F(x,y_0) \with F(x,y_1) \rightarrow y_0=y_1$$.
%\end{exm}


    \begin{thm}Если рассмотреть ИИП с ZFC, то для любого $P$ выполнено $\vdash P \vee \neg P$.\end{thm}
    \begin{proof}Рассмотрим $\mathcal{B} = \{0,1\}$, $A_0 = \{ x \in \mathcal{B} | x = 0 \vee P \}$ и
        $A_1 = \{ x \in \mathcal{B} | x = 1 \vee P\}$.
        $\{A_0,A_1\}$ --- семейство непустых множеств, и по акс. выбора существует
        $f: \{A_0,A_1\} \rightarrow \cup A_i$, что $f(A_i) \in A_i$. (Если $P$, то $A_0 = A_1$ и $\{A_0,A_1\} = \{\mathcal{B}\}$).

        \vspace{0.3cm}
        \begin{tabular}{ll}
            $\vdash f(A_0) \in A_0 \with f(A_1) \in A_1$ & а.выбора: $f(A_i) \in A_i$\\
            $\vdash {\color{olive}f(A_0) \in \mathcal{B}} \with (f(A_0) = 0 \vee P) \with {\color{olive}f(A_1) \in \mathcal{B}} \with (f(A_1) = 1 \vee P)$ & а.выделения\\
%$\vdash(f(A_0) = 0 \vee P) \with (f(A_1) = 1 \vee P)$ & Удал. $(\with)$\\
            $\vdash (f(A_0) = 0 \with f(A_1) = 1) \vee P$ & Удал. $(\with)$ + дистр.\\
            $\vdash P\vee{\color{blue}f(A_0) \ne f(A_1)}$ & $0 \ne 1$ и транз.\\$\vdash P \rightarrow A_0 = A_1$ & Определение $A_i$\\
            $\vdash A_0 = A_1 \rightarrow f(A_0) = f(A_1)$ & Конгруэнтность\\
            $\vdash \color{blue} f(A_0) \ne f(A_1) \rightarrow \neg P$ & Контрапозиция\\
            $\vdash P \vee \neg P$ & Подставили
        \end{tabular}
    \end{proof}




    \begin{thm}[конечного выбора]
Если $X_1\ne\varnothing, \dots, X_n\ne\varnothing$, $X_i\cap X_j = \varnothing$ при $i \ne j$, то $\times \{X_1, \dots, X_n\} \ne \varnothing$.
    \end{thm}

    \begin{proof}
        \begin{itemize}\item База: $n=1$. Тогда $\exists x_1.x_1 \in X_1$, поэтому $\exists x_1.\{x_1\} \in \times \{X_1\}$.

            \item Переход: %если $\exists v.v \in \times \{X_{1,n}\}$ и $\exists x_{n+1}.x_{n+1} \in X_{n+1}$, то
            $\exists v.v \in \times \{X_{1,n}\}\rightarrow\exists x_{n+1}.x_{n+1} \in X_{n+1}\rightarrow
            v \cup \{x_{n+1}\} \in \times (X_{1,n}\cup\{X_{n+1}\})$
        \end{itemize}\vspace{-0.3cm}\end{proof}

%Построим явно: $(\exists x_1.x_1 \in X_1) \rightarrow \exists f.\exists x_1.f = \{\langle X_1, x_1 \rangle\}\with x_1 \in X_1$

%Построим явно: $$\exists x_1.\dots\exists x_n.x_1 \in X_1 \with \dots \with x_n \in X_1 \rightarrow \varphi(\langle X_1, x_1\rangle, \dots, \langle X_n, x_n\rangle)$$
%И потом:
%$$X_1 \ne \varnothing \with \dots \with X_n \ne \varnothing \rightarrow \exists f.\varphi(f)$$

%Докажем явным выписыванием:
%$x_1 \in X_1 \with \dots \with x_n \in X_n \rightarrow \varphi(\{\langle X_1, x_1\rangle, \dots, \langle X_n, x_n\rangle\})$\\
%$\exists x_1 \in X_1 \with \dots \with (\exists x_n \in X_n)\rightarrow \exists f.\varphi(f)$
%$$(x_1 \in X_1) \with \dots \with (x_n \in X_n) \rightarrow (f = \{\langle X_1, x_1 \rangle, \dots, \langle X_n, x_n \rangle\} \rightarrow f(X_1) = x_1 \with \dots \with f(X_n) = x_n)$$
%$$(f = \{\langle X_1, x_1 \rangle, \dots, \langle X_n, x_n \rangle\} \rightarrow f(X_1) = x_1 \with \dots \with f(X_n) = x_n)$$
%$$(\exists x_1. x_1 \in X_1)\with\dots\with(\exists x_n.x_n \in X_n)\rightarrow\exists f.f(X_1) \in X_1 \with \dots \with f(X_n) \in X_n$$

    \begin{axm}[счётного выбора]
Для счётного семейства непустых множеств существует функция, каждому из которых сопоставляющая один из своих элементов
    \end{axm}

    \begin{axm}[зависимого выбора]
если $\forall x \in E.\exists y \in E. x R y$, то существует последовательность $x_n: \forall n.x_n R x_{n+1}$
    \end{axm}



    Заметим, что семейство $\{A_0, A_1\}$ из теоремы Диаконеску в ИИП не является конечным (равно как и бесконечным).
    \begin{dfn}Конечное множество --- равномощное некоторому конечному кардинальному числу.\end{dfn}

    \begin{itemize}
        \item Какова мощность семейства?
        \item 1, если $P$, и 2, если $\neg P$.
        \item Но поскольку $P \vee \neg P$ не выполнено в ИИП, мы не можем
        доказать, что мощность семейства 1 или 2.
        \item Поэтому мы не можем воспользоваться теоремой конечного выбора.
    \end{itemize}


    \begin{dfn}Наследственным свойством множества назовём такое свойство, которым обладает как само
        множество, так и все его подмножества.
    \end{dfn}

    \begin{dfn}Фундированным множеством назовём такое, которое не пересекается хотя бы с одним своим элементом.\end{dfn}

    \begin{dfn}Аксиома фундирования.
        В каждом непустом множестве найдется элемент, не пересекающийся с исходным множеством.
        $$\forall x .x = \varnothing \vee \exists y .y \in x \with \forall z.z \in x \rightarrow z \notin y$$
    \end{dfn}

    Иными словами, в каждом множестве есть элемент, минимальный по отношению $(\in)$.


    \begin{dfn}\emph{Универсум фон Неймана} $V$ --- все наследственные фундированные множества.\end{dfn}

    При наличии аксиомы фундирования можно показать, что $V = \cup_a V_a$, где:
    $$V_a = \left\{\begin{array}{ll}
                       \varnothing, & a=0\\
                       \mathcal{P}(V_b), & a = b'\\
                       \bigcup_{b < a}(V_b), & a \text{ --- предельный}
    \end{array}\right.$$

    \begin{dfn}
        \emph{Конструктивный универсум} $L = \cup_a L_a$, где:
        $$L_a = \left\{\begin{array}{ll}
                           \varnothing, & a=0\\
                           \{ \{ x\in L_b\ |\ \varphi(x,t_1,\dots,t_k) \}\ |\ \varphi\text{ --- формула}, t_i \in L_b\}, & a = b'\\
                           \bigcup_{b < a}(L_b), & a \text{ --- пред.}
        \end{array}\right.$$
    \end{dfn}



    \begin{dfn}
        Аксиома конструктивности: $V=L$, то есть допустимы только те фундированные множества, которые задаются формулами.
    \end{dfn}

    \begin{thm}Аксиома выбора и континуум-гипотеза следуют из аксиомы конструктивности\end{thm}

    Для некоторых теорий аксиома слишком сильна.



    \section{Индукция и полная индукция. Наследственные множества. Трансфинитная индукция
        (аналоги полного и обычного варианта математической индукции). Доказательство $a \cdot a = a$ при $a \ge \aleph_0$.}

    \subsection{Индукция и полная индукция}

    \begin{dfn}[принцип математической индукции]
Какое бы ни было $\varphi(x)$, если $\varphi(0)$ и при всех $x$ выполнено $\varphi(x)\rightarrow \varphi(x')$, то
при всех $x$ выполнено и само $\varphi(x)$.
    \end{dfn}

    \begin{dfn}[принцип полной математической индукции]
Какое бы ни было $\psi(x)$, если $\psi(0)$ и при всех $x$ выполнено $(\forall t.t \leq x \rightarrow \psi(t))\rightarrow \psi(x')$, то
при всех $x$ выполнено и само $\psi(x)$.
    \end{dfn}

    \begin{thm}Принципы математической индукции эквивалентны\end{thm}
    \begin{proof}
        $(\Rightarrow)$ взяв $\varphi := \psi$, имеем выполненность $\varphi(x)\rightarrow\varphi(x')$, значит, $\forall x.\psi(x)$. \\
        $(\Leftarrow)$ возьмём $\psi(x) := \forall t.t\le x\rightarrow\varphi(t)$.
    \end{proof}

    \subsection{Наследственные множества}

    \begin{dfn} Назовём вполне упорядоченное отношением $(\in)$ множество $S$ наследственным подмножеством $A$, если
        $\forall x.x \in A \rightarrow (\forall t.t \in x \rightarrow t \in S) \rightarrow x \in S$.
    \end{dfn}
    \begin{thm}Единственным наследственным подмножеством вполне упорядоченного множества является оно само.\end{thm}
    \begin{proof}Пусть $B \subseteq A$ --- наследственное и $B \ne A$.
    Тогда существует $a = \min (A \setminus B)$. Тогда $(\forall t.t \in a \rightarrow t \in B) \rightarrow a \in B$ по наследственности $B$,
    и выполнено $\forall t.t \in a \rightarrow t \in B$ (по минимальности $a$). Значит, $a \in B$.
    \end{proof}

    \subsection{Трансфинитная индукция}

    \begin{thm}[ограниченная трансфинитная индукция] Если для $\varphi(x)$ (некоторого утверждения
    теории множеств) и некоторого ординала $\varepsilon$ (ограничения) выполнено
        $\forall x.x \in \varepsilon \rightarrow (\forall t.t \in x \rightarrow \varphi(t)) \rightarrow \varphi(x)$,
        то $\forall x.x \in \varepsilon \rightarrow \varphi(x)$.
    \end{thm}
    \begin{proof}Рассмотрим $S = \{ x\in \varepsilon\ |\ \varphi(x) \}$. Тогда $x \in S$ равносильно
        $x\in\varepsilon\with\varphi(x)$.
        Тогда перепишем: $\forall e.e \in \varepsilon \rightarrow (\forall x.x \in e \rightarrow x \in S) \rightarrow e \in S$.
        Отсюда по теореме о наследственных множествах $S = \varepsilon$.\end{proof}

    \begin{thm}[неограниченная трансфинитная индукция] Если для $\varphi(x)$ (некоторого утверждения
    теории множеств) выполнено
        $\forall x.\text{ординал}(x) \rightarrow (\forall t.t \in x \rightarrow \varphi(t)) \rightarrow \varphi(x)$,
        то $\forall x.\text{ординал}(x) \rightarrow \varphi(x)$.
    \end{thm}

    \subsection{Альтернативная формулировка трансфинитной индукции}

    \begin{thm}Для ординала $\varepsilon$ подмножество $S \in \varepsilon$ --- наследственное, если и только если одновременно:
        \begin{enumerate}
            \item Если $x \in \varepsilon$ и $x = \varnothing$, то $x \in S$;
            \item Если $x \in \varepsilon$ и существует $y$: $y' = x$, то $y \in S \rightarrow x \in S$;
            \item Если $x \in \varepsilon$ и $x$ --- предельный, то $(\forall t.t \in x \rightarrow t \in S) \rightarrow (x \in S)$.
        \end{enumerate}
    \end{thm}

    \begin{proof}
        $(\Rightarrow)$ очевидно. Докажем $(\Leftarrow)$: пусть $S$ не наследственное:
        $E := \{e \in \varepsilon \ |\  (\forall t.t \in e \rightarrow t \in S) \with e \notin S \}$
        и $E \ne \varnothing$. Тогда пусть $e = \min E$.

        \begin{enumerate}
            \item $e = \varnothing$ или предельный. Тогда $(\forall t.t \in e \rightarrow t \in S) \rightarrow (e \in S)$.
            \item $e = y'$. Тогда $y \in \varepsilon$ ($\varepsilon$ --- ординал) и
            $(\forall t.t \in y \rightarrow t \in S) \rightarrow (y \in S)$ (так как $e$ минимальный, для которого $S$ не наследственное).
            По условию, $(y \in S) \rightarrow (e \in S)$, отсюда $(\forall t.t \in e \rightarrow t \in S) \rightarrow (e \in S)$.

            \begin{center}\tikz{\draw[thick,-stealth] (0,0) -- (7,0);
            \filldraw (2,0) circle (1pt);
            \filldraw (1,0) circle (1pt);
            \filldraw (3,0) node[above] {$t \in y$} circle (2pt);
            \filldraw[red] (4,0) node[above] {$y$} circle (2pt) ;
            \filldraw (5,0) node[above] {$e\vphantom{y}$} circle (2pt) ; }\end{center}
        \end{enumerate}
    \end{proof}

    \subsection{Доказательство $a \cdot a = a$ при $a \ge \aleph_0$}

    \begin{thm}Если $\alpha$ --- кардинальное число и $\alpha \ge \aleph_0$,
    то $\alpha\cdot\alpha = \alpha$.\end{thm}
    \begin{proof}Формализуем свойство <<быть кардинальным числом>>:
        $|x|=x$ и утверждение теоремы: $\varphi(x) := |x|=x\rightarrow x<\omega\vee|x\times x|=x$.

        Транфинитная индукция: при $\forall y.y < x \rightarrow \varphi(y)$ покажем $\varphi(x)$,
        разберём варианты:

        \begin{enumerate}
            \item $|x|\ne x$ или $|x| < \omega$, тогда $\varphi(x)$ истинно (из лжи следует любое утверждение).
            \item $|x|=x$ и $|x| = \omega$, тогда надо показать $\omega < \omega \vee |\omega\times\omega|=\omega$
            (утверждение можно показать без индукции, рассмотрим отдельно).
            \item $|x|=x$ и $|x| > \omega$, тогда надо показать $x < \omega \vee |x\times x|=x$ (рассмотрим отдельно).
        \end{enumerate}
    \end{proof}

    \subsubsection{Счётный случай: $|\omega \times \omega| = \omega$}
    Тогда $\omega \times \omega$ упорядочим так: $\langle p,q \rangle \prec \langle s,t \rangle$,
    если \begin{enumerate}
             \item $\max(p,q) < \max(s,t)$
             \item $\max(p,q) = \max(s,t)$ и $q < t$
             \item $\max(p,q) = \max(s,t)$, $q = t$ и $p < s$
    \end{enumerate}
    Очевидно, можно построить биекцию между так упорядоченными значениями и $\omega$.

    \begin{center}\begin{tikzpicture}

                      \filldraw[gray!20] (0,0) -- (4.5, 0) -- (4.5, 3) -- (0, 3) -- cycle;
                      \filldraw[gray!50] (0,0) -- (3, 0) -- (3, 2) -- (0, 2) -- cycle;
                      \filldraw[gray] (0,0) -- (1.5, 0) -- (1.5, 1) -- (0, 1) -- cycle;

                      \foreach \x in {0, 1, 2, 3} {
                          \foreach \y in {0, 1, 2, 3} {
                              \node at (1.5*\x + 1, \y + 0.6) {
                                  \pgfmathparse{(max(\x,\y))*(max(\x,\y)) + \y + ((\y+1)==(max(\x,\y)+1))*\x}%
                                  \pgfmathprintnumber{\pgfmathresult}%
                              };
                              \node at (1.5*\x + 0.4, \y + 0.2) {\footnotesize $\langle \x,\y \rangle$};
                              \draw (1.5*\x, \y) -- (1.5*\x +1.5, \y) -- (1.5*\x +1.5, \y +1) -- (1.5*\x, \y +1) -- cycle;
                          }
                      }

        \end{tikzpicture}\end{center}

    \subsubsection{Общий случай: $\alpha$ --- кардинал, $\alpha > \omega$ и $|\alpha \times \alpha| = \alpha$}
    Аналогично счётному случаю, $\alpha \times \alpha$ упорядочим так: $\langle p,q \rangle \prec \langle s,t \rangle$,
    если \begin{enumerate}
             \item $p \cup q < s \cup t$
             \item $p \cup q = s \cup t$ и $q < t$
             \item $p \cup q = s \cup t$, $q = t$ и $p < s$
    \end{enumerate}
    \begin{itemize}
        \item Легко заметить, что это --- линейный порядок (показав, что $p \not\prec q$ и $q \not\prec p$ влечёт $p = q$)
        \item ... и полный порядок. Найти наименьший в $S \ne \varnothing$ возможно, рассмотрев $m_1 := \min \{ p \cup q\ |\ \langle p,q \rangle \in S\}$ и
        $M_1 := \{ \langle p,q\rangle\ |\ \langle p,q \rangle \in S, p \cup q = m_1\}$,
        затем $m_2 := \min \{q\ |\ \langle p,q \rangle \in M_1 \}$,
        $M_2 := \{\langle p,q\rangle\ |\ \langle p,q \rangle \in M_1, q = m_1\}$.
        Тогда требуемым наименьшим в $S$ будет $\min \{ p\ |\ \langle p,q \rangle \in M_2\}$
        \item Тогда $\langle \alpha\times\alpha, (\prec)\rangle$ соответствует какой-то ординал $\tau$
        и сохраняющая порядок биекция $t: \tau\rightarrow\alpha\times\alpha$.
        \item Заметим, что $x < \omega$ тогда и только тогда, когда $\cup(\cup t(x)) < \omega$
        (очевидно из того, что $|\{z\ |\ \text{ординал}(z), z < x\}|=|\{p\ |\ p \prec t(x)\}|$).
        \item Покажем, что $|\tau| = \alpha$.
    \end{itemize}

    \subsection{Докажем $\tau = \alpha$}

    Очевидно, что $\tau \ge \alpha$ (так как $|\tau| = |\alpha\times\alpha| \ge \alpha$). Но пусть $\tau > \alpha$.
    \begin{itemize}
        \item Тогда $t(\alpha) = \langle\zeta,\eta\rangle$ определено (у $\alpha$ есть образ).
        \item Пусть $\sigma := \zeta \cup \eta$. Очевидно, $\langle \zeta, \eta \rangle \preceq \langle \sigma,\sigma \rangle$
        и $\sigma \in \alpha$.
        \item Каков образ $t$ на этом начальном отрезке?
        $\{t(x)\ |\ x < \alpha\} \subseteq \{\langle p,q\rangle\ |\ p,q \le \sigma\}$.
        Поэтому $\alpha \le |(\sigma+1)\times(\sigma+1)|$.
        \item С другой стороны, $\sigma < \alpha$. Поскольку $\alpha$ --- кардинал (т.е., в частности, предельный ординал),
        то $\sigma+1 < \alpha$ и $|\sigma+1| < \alpha$.
        \item По предположению индукции, $|\sigma+1|<\omega \vee |\sigma+1| = |\sigma+1|\cdot|\sigma+1|$,
        по свойствам $(\prec)$ имеем $\sigma\ge\omega$.
        \item Отсюда $\alpha \le |(\sigma+1)\times(\sigma+1)| = |\sigma+1| < \alpha$, что невозможно.
    \end{itemize}



    \section{Система $S_\infty$, степень и порядок доказательства.
    Правило сечения, теорема об устранении сечений. Доказательство непротиворечивости формальной арифметики.}


    \begin{enumerate}
        \item Язык: связки $\neg$, $\vee$, $\forall$, $=$; нелогические символы: $(+)$,$(\cdot)$,$(')$,$0$; переменные: $x$.
        \item Аксиомы: все истинные формулы вида $\theta_1=\theta_2$ $(0 = 0, 5 = 5)$; все истинные отрицания формул вида $\neg\theta_1=\theta_2 $$(\neg 1 = 0, \neg 2 \cdot 2 = 5)$
        ($\theta_i$ --- термы без переменных).
        \item Структурные (слабые) правила:
        $$\infer{\zeta\vee\beta\vee\alpha\vee\delta}{\zeta\vee\alpha\vee\beta\vee\delta} \quad\quad
        \infer{\alpha\vee\delta}{\alpha\vee\alpha\vee\delta}$$

        сильные правила
        $$\infer{\alpha\vee\beta}{\beta}\quad
        \infer{\neg(\alpha\vee\beta)\vee\delta}{\neg\alpha\vee\delta\quad\neg\beta\vee\delta}\quad
        \infer{\neg\neg\alpha\vee\delta}{\alpha\vee\delta}\quad
        \infer{(\neg\forall x.\alpha)\vee\delta}{\neg\alpha[x := \theta]\vee\delta}\quad$$

        Формулы в правилах, обозначенные буквами $\zeta$ и $\delta$, называются боковыми и могут отсутствовать.
        \item и ещё два правила
    \end{enumerate}



    Бесконечная индукция:
    $$\infer{(\forall x.\alpha)\vee\delta}{\alpha[x:=\overline{0}]\vee\delta
    \quad\alpha[x:=\overline{1}]\vee\delta
    \quad\alpha[x:=\overline{2}]\vee\delta\quad\dots}$$

    Сечение:
    $$\infer{\zeta\vee\delta}{\zeta\vee\alpha\quad\quad\neg\alpha\vee\delta} $$ (аналог M.P. если нет $\zeta$, то в точности он)
    Здесь $\alpha$ --- секущая формула, число связок в $\neg\alpha$ --- степень сечения.\\
    В отличие от других правил, в правиле сечения хотя бы одна из боковых формул $\zeta$ или $\delta$ должна присутствовать.


    \begin{enumerate}
        \item Доказательства образуют деревья.
        \item Каждой формуле в дереве сопоставим порядковое число (ординал).
        \item Порядковое число заключения любого неструктурного правила строго больше порядкового числа его посылок
        (больше или равно в случае структурного правила).

%$$%\infer{(\forall a.a = a)_\omega}{
%  % \infer{(0 = 0)_1}{}\quad
%  % \infer{(0'= 0')_2}{\infer{\dots\vphantom{0}}{\infer{(0= 0)_1}{}}}\quad
%  % \infer{0''= 0''}{\infer{\dots\vphantom{0}}{\infer{0'= 0'}{\infer{\dots\vphantom{0}}{\infer{0= 0}{}}}}}\quad\dots
%\infer{(\forall a.a = a)_\omega}{
%   \infer{(0 = 0)_1}{}\quad
%   \infer{(0'= 0')_2}{}\quad
%   \infer{(0''= 0'')_3}{}\quad\dots
%}\quad\quad
%\infer{(\forall a.a = a)_1}{
%   \infer{(0 = 0)_0}{}\quad
%   \infer{(0'= 0')_0}{}\quad
%   \infer{(0''= 0'')_0}{}\quad\dots
%}$$

        $$\infer{(\neg\neg\forall x.\neg x'=0)_{\omega+1}}{\infer{(\forall x.\neg x' = 0)_\omega}{(\neg 1=0)_1\quad (\neg 2=0)_2 \quad (\neg 3=0)_4 \quad (\neg 4 = 0)_8 \dots}}$$

        \item Существует конечная максимальная степень сечения в дереве (назовём её степенью вывода).
    \end{enumerate}



    \begin{thm}Если $\vdash_\text{фа}\alpha$, то $\vdash_\infty|\alpha|_\infty$ \end{thm}
    \begin{exm}Обратное неверно: $$\infer{\forall x.\neg\omega_1(x,\overline{\ulcorner\sigma\urcorner})}
    {\neg\omega_1(\overline{0},\overline{\ulcorner\sigma\urcorner})\quad\quad
    \neg\omega_1(\overline{1},\overline{\ulcorner\sigma\urcorner})\quad\quad
    \neg\omega_1(\overline{2},\overline{\ulcorner\sigma\urcorner})\quad\quad\dots}$$
    \end{exm}
    \begin{thm}Если Ф.А. противоречива, то противоречива и $S_\infty$\end{thm}



    \begin{thm}\vspace{-0.3cm}
        $$\infer{\neg\alpha\vee\delta\quad\neg\beta\vee\delta\vphantom{overline{1}]}}{\neg(\alpha\vee\beta)\vee\delta}\quad
        \infer{\alpha\vee\delta\vphantom{overline{1}]}}{\neg\neg\alpha\vee\delta}\quad
        \infer{\alpha[x:=\overline{0}]\vee\delta
        \quad\alpha[x:=\overline{1}]\vee\delta
        \quad\alpha[x:=\overline{2}]\vee\delta\quad\dots}{(\forall x.\alpha)\vee\delta}
        $$\vspace{-0.5cm}
%Если формула $\alpha$ доказана и имеет вид, похожий на заключение правил де Моргана,
%отрицания и бесконечной индукции --- то посылки соответствующих правил могут быть получены из самой
%формулы $\alpha$ доказательством, причём доказательством с не большей степенью и не большим порядком.
    \end{thm}
    \begin{proof}
        \begin{tabular}{ll}\begin{minipage}{0.5\linewidth}
                               Например, формула вида $\neg\neg \alpha\vee\delta$.
                               \vspace{0.2cm}Проследим историю $\neg\neg\alpha$; она могла быть получена:
                               \begin{enumerate}
                                   \item ослаблением --- заменим $\neg\neg\alpha$ на $\alpha$ в этом узле и последующих.
                                   \item отрицанием --- выбросим правило, заменим $\neg\neg\alpha$ на $\alpha$ в последующих.
                               \end{enumerate}
%Изменённый вывод --- доказательство требуемого.

            \end{minipage} &
            \begin{minipage}{0.5\linewidth}
                \tikz{
                    \node at (-1.5,3) (J1) { $\delta(0)$ };
                    \node at (1.5,3) (J2) { $\alpha\vee\delta(2)$ };
                    \node at (-1.5,1.5) (I1) { ${\color{red}\neg\neg}\alpha\vee\delta(0)$ };
                    \node at (1.5,1.5) (I2) { $\color{red}\neg\neg\alpha\vee\delta(2)$ };
                    \node at (1.5,0) (C) { ${\color{red}\neg\neg}\alpha\vee\forall x.\delta(x)$ };
                    \node at (3.5,1.5) (D) { $\dots$ };
                    \draw[->] (J1) -- (I1); \draw[->] (I1) -- (C);
                    \draw[red,->] (J2) -- (I2); \draw[red,->] (I2) -- (C);
                    \draw[->] (D) -- (C);
                    \draw[blue,->,bend right=20] (J2) .. controls (0,1.5) .. (C);
                }\end{minipage}
        \end{tabular}
    \end{proof}

    \begin{thm}Если $\alpha$ имеет вывод степени $m>0$ порядка $t$, то
    можно найти вывод степени строго меньшей $m$ с порядком $2^t$.
    \end{thm}

    \begin{proof}Трансфинитная индукция. Пусть для всех деревьев порядка $t_1 < t$
        условие выполнено. Покажем, что оно выполнено для порядка $t$.
        Рассмотрим заключительное правило. Это может быть...

        \begin{enumerate}
            \item Не сечение.
            \item Сечение, секущая формула --- элементарная.
            \item Сечение, секущая формула --- $\neg\alpha$.
            \item Сечение, секущая формула --- $\alpha\vee\beta$.
            \item Сечение, секущая формула --- $\forall x.\alpha$.
        \end{enumerate}
    \end{proof}



    $$\infer{(\alpha)_{t}}{(\pi_0)_{t_0}\quad(\pi_1)_{t_1}\quad(\pi_2)_{t_2}\quad\dots}$$
    Заменим доказательства посылок $(\pi_i)_{t_i}$ на $(\pi'_i)_{2^{t_i}}$ по индукционному предположению. (Что если ti $=$ t?)

    \begin{enumerate}
        \item Поскольку степени посылок $m'_i < m_i$, то $\max m'_i < \max m_i$.
        \item Поскольку $t_i \le t$, то $2^{t_i} \le 2^t$.
    \end{enumerate}


    \vspace{-0.1cm}$$\infer{\zeta\vee\delta}{\zeta\vee\forall x.\alpha\quad\quad(\neg\forall x.\alpha)\vee\delta}$$\vspace{-0.1cm}
    Причём степень и порядок выводов компонент, соответственно, $(m_1,t_1)$ и $(m_2,t_2)$.\vspace{-0.1cm}
    \begin{enumerate}
        \item По индукции, вывод $\zeta\vee\forall x.\alpha$ можно упростить до $(m_1',2^{t_1})$.
        \item По обратимости, можно построить вывод $\zeta\vee\alpha[x := \theta]$ за $(m_1',2^{t_1})$ [предыдущая теорема предполагает, что степень и порядок при перестроении не увеличиваются].
        \item В формуле $(\neg \forall x. \alpha)\vee\delta$ формула $\neg\forall x.\alpha$ получена
        либо ослаблением, либо квантификацией из $\neg\alpha[x := \theta_k]\vee\delta_k$.
        \begin{enumerate}
            \item Каждое правило квантификации заменим на:
            $$\infer{\zeta\vee\delta_k}{\zeta\vee\alpha[x := \theta_k]\quad\quad(\neg\alpha[x := \theta_k])\vee\delta_k}$$
            \item Остальные вхождения $\neg\forall x.\alpha$ заменим на $\zeta$ (в правилах ослабления).
        \end{enumerate}
        \item В получившемся дереве меньше степень --- так как в $\neg\alpha[x := \theta]$ меньше связок, чем в $\neg\forall x.\alpha$.
%\item Нумерацию можно также перестроить.
    \end{enumerate}



    \begin{center}\tikz{
        \node (RRW) at (6,5.5) {$\delta_l$};
        \node (RR) at (6,4) {${\color{red}(\neg\forall x.\alpha)}\vee\delta_l$};
        \node (RRnew) at (6.5,3) {$\color{blue}\zeta\vee\delta_l$};
        \node (LL) at (-2,5) {$\dots$};

        \node (R) at (3.2,3) {${\color{red}(\neg\forall x.\alpha)}\vee\delta_k$};
        \node (Rnew) at (1,3) {$\color{blue}\zeta\vee\delta_k$};
        \node (RQ) at (3,5) {$\neg\alpha[x := \theta]\vee\delta_k$};
        \node (L1) at (0,4) {$\color{blue}\zeta\vee\alpha[x := \theta]$};
        \node (L) at (-2,2) {$\color{red}\zeta\vee\forall x.\alpha$};

        \node (CR) at (4.5,2) {${\color{red}(\neg\forall x.\alpha)}\vee\delta$};
        \node (CRnew) at (5,1) {$\color{blue}\zeta\vee\delta$};
        \node (C) at (0,0) {$\color{red}\zeta\vee\delta$};
        \draw[red,->] (L) -- (C);
        \draw[red,->] (LL) -- (L);
        \draw[red,->] (CR) -- (C);
        \draw[blue,->,bend right=30] (LL) to (L1);
        \draw[dashed,->] (R) -- (CR);
        \draw[dashed,->] (RR) -- (CR);
        \draw[double,->,blue] (RR) -- (RRnew);
        \draw[double,->,blue] (CR) -- (CRnew);
        \draw[double,->,blue] (R) -- (Rnew);
        \draw[->] (RRW) -- (RR);
        \draw[->] (RQ) -- (R);
        \draw[blue,->,bend right=20] (RQ) to (Rnew);
        \draw[blue,->] (L1) -- (Rnew);
    }\end{center}



    \begin{dfn}Итерационная экспонента
        $$(a\uparrow)^m(t
        ) =
        \left\{
            \begin{array}{ll}     t,&m=0\\
                a^{(a\uparrow)^{m-1}(t)},&m > 0
            \end{array}
        \right.
        $$
    \end{dfn}
    \begin{thm}Если $\vdash_\infty\sigma$ степени $m$ порядка $t$, то найдётся доказательство без сечений
    порядка $(2\uparrow)^m(t)$
    \end{thm}
    \begin{proof}
        В силу конечности $m$ воспользуемся индукцией по $m$ и теоремой об уменьшении степени.
    \end{proof}


    \begin{dfn}$\varepsilon_0$ --- неподвижная точка $\varepsilon_0 = \omega^{\varepsilon_0}$\end{dfn}

    Иначе говоря, $\varepsilon_0 = \{ \omega, \omega^\omega, \omega^{\omega^\omega}, (\omega \uparrow)^3(\omega), (\omega\uparrow)^4(\omega), \dots \}$.

    Очевидно, что теорема об устранении сечений может быть доказана трансфинитной индукцией до ординала $\varepsilon_0$
    (максимальный порядок дерева вывода, при правильной нумерации вершин).



    \begin{lmm}Если $\vdash_\infty\alpha$ и $\vdash_\infty\neg\alpha$, тогда $\vdash_\infty\neg 0=0$.
    \end{lmm}

    \begin{thm}$\not\vdash_\infty\neg 0=0$\end{thm}
    \begin{proof} Пусть $\vdash_\infty\neg 0=0$, устраним сечения и рассмотрим заключительное правило.
        \begin{enumerate}
            \item Правило де Моргана?  Нет отрицаний дизъюнкции ($\neg(\alpha\vee\beta)\vee\delta$). \item Отрицание?  Нет двойного отрицания ($\neg\neg\alpha\vee\delta$). \item Бесконечная индукция или квантификация?  Нет квантора. \item Ослабление?  Нет дизъюнкции ($\alpha \vee \beta$), хотя $\beta$ обязана присутствовать. \item Сечение?  Исключено по условию.
        \end{enumerate}

        То есть, неизбежно, $\neg 0=0$ --- аксиома, что также неверно.
    \end{proof}

    \section{Сколемизация. Эрбранов универсум, основные термы, эрбранова интерпретация,
        система дизъюнктов, основные примеры, система основных примеров, теорема Гёделя о компактности,
        теорема Эрбрана. Правило резолюции (для исчисления высказываний и для исчисления предикатов),
        задачи унификации, уравнения в алгебраических термах, наибольший общий унификатор.
        Общая формулировка метода резолюции. SMT-решатели.}

    \subsection{Сколемизация}

    \begin{enumerate}
        \item Для любой $\alpha$ найдётся $\beta$ с поверхностными кванторами, что $\vdash\alpha \leftrightarrow \beta$.

        \item Исходная задача: проверка $\vdash\alpha$. Это эквивалентно $\vdash\beta$. Эквивалентно $\models\beta$.
        То есть, при любом $D$:
        \begin{itemize}
            \item при любом $x_1$ найдётся такой $x_2$, что ...
            \item при любом $x_3$ найдётся $x_4$, что ... (и т.д.) ...
            \item что найдётся $x_n$, что $\varphi$ истинен.
        \end{itemize}

        \item Заменим $x_{2k}$ \emph{функциями Сколема} $e_{2k}(x_1,x_3,\dots,x_{2k-1})$.
        Получим:
        $$\eta := \forall x_1.\forall x_3\dots\forall x_{n-1}.\varphi[x_2:=e_2(x_1), x_4:=e_4(x_1,x_3), \dots, x_n := e_n(x_1,x_3,\dots,x_{n-1})]$$

        \item Сколемизация сохраняет выполнимость:
        $(\Rightarrow)$ если $\beta$ истинна, то рассмотрим все $x_1,x_2,\dots$, что $\varphi$ истинна ---
        и положим $e_k(x_1,x_3,\dots,x_{k-1}) := x_k$;
        $(\Leftarrow)$ если $\eta$ истинна, то $\beta$ истинна в той же оценке.
    \end{enumerate}

    \subsection{Сколемизация, избавляемся от чередований кванторов}

    \begin{enumerate}
        \item Было: $\beta := \forall x_1.\exists x_2.\forall x_3.\exists x_4\dots \forall x_{n-1}.\exists x_n.\varphi$
        ($n$ чередований кванторов).

        \item Сколемизация не сохраняет общезначимость: $\vdash\forall x.\exists y.y > x$, но
        $\forall x.e(y) > x$ ложно при $e(x) := x$.

        \item Поэтому мы проверяем иное свойство: при любом $D$ найдутся $e_i$, что
        $$\eta := \forall x_1.\forall x_3\dots\forall x_{n-1}.\varphi[x_2:=e_2(x_1), \dots, x_n := e_n(x_1,x_3,\dots,x_{n-1})]$$

        \item Как бы убрать и это чередование? С помощью отрицания --- рассмотрим $\alpha' := \neg\alpha$ и
        сколемизированное представление для неё:
        $$\eta' := \forall x_1.\forall x_3\dots\forall x_{m-1}.\varphi'[x_2:=e'_2(x_1), \dots, x_m := e'_n(x_1,x_3,\dots,x_{m-1})]$$

        Тогда $\models\alpha$ соответствует невыполнимости $\alpha'$, то есть невыполнимости $\eta'$.

        \item То есть, $\models \alpha$ тогда и только тогда, когда
        при любом $D$ --- при любых $e'$ --- найдутся $x_1, x_3, \dots x_{m-1}$, что ложно

        $$\varphi'[x_2:=e'_2(x_1), x_4:=e'_4(x_1,x_3), \dots, x_m := e'_n(x_1,x_3,\dots,x_{m-1})]$$
    \end{enumerate}

    \subsection{Эрбранов универсум, основные термы}

    \begin{dfn}Пусть $\varphi$ --- формула и $\mathcal{F}_k$ --- все $k$-местные функциональные символы из $\varphi$.
    Тогда:\\
        $H^0_\varphi := \mathcal{F}_0$ (либо $\{a\}$, если $\mathcal{F}_0 = \varnothing$);
        $H^{k+1}_\varphi := H^k_\varphi \cup \{``f(``++x_1++``,``\dots``,``++x_n++``)``\ |\ x_i \in H^k_\varphi, f \in \mathcal{F}_n\}$\\

        Тогда $H_\varphi = \bigcup_n H^n_\varphi$ --- эрбранов универсум, его элементы --- основные термы.
    \end{dfn}

    \begin{exm}[$\varphi := P(a)\vee Q(f(b))$]
$$\begin{array}{ll}
      H^0_\varphi &= \{a,b\}\\
      H^1_\varphi &= \{a,b,f(a),f(b)\}\\
      H^2_\varphi &= \{a,b,f(a),f(b),f(f(a)),f(f(b))\}\\&\dots\\
      H_\varphi &= \{f^{(n)}(x)\ |\ n \in \mathbb{N}_0, x \in \{a,b\}\}\end{array}$$
    \end{exm}

    \begin{exm}
        \begin{tabular}{ll}
            $\varphi := P(0)\vee (P(x)\rightarrow P(x'))$ & $H_\varphi = \{0, 0', 0'', 0''', \dots\}$ \\
            $\varphi := P(x')$ & $H_\varphi =\{a,a',a'',a''',\dots\}$
        \end{tabular}
    \end{exm}

    \subsection{Эрбранова интерпретация}

    \begin{dfn}Для бескванторной $\varphi$ рассмотрим $H_\varphi$,
    зададим оценку функциональных символов $f$ из $\varphi$:
        $$\mathcal{F}_f(\llbracket\overline{\theta}\rrbracket) := ``f(`` ++ \llbracket \overline{\theta} \rrbracket ++ ``)``$$

        Оценку для $P$ ($k$-местного предикатного символа из $\varphi$)
        зададим набором истинных значений $S_P \subseteq (H_\varphi)^k$:

        $$P(\theta_1,\dots,\theta_{a(i)}) \text{ истинно тогда и только тогда, когда }
        \langle\llbracket\theta_1\rrbracket,\dots,\llbracket\theta_k\rrbracket\rangle \in S_P$$

        Также пусть $E: \mathcal{V}\rightarrow H_\varphi$, тогда
        $\langle H_\varphi, \mathcal{F}, \mathcal{P}, E\rangle$
        задаёт \emph{эрбранову интерпретацию}.
    \end{dfn}

    \begin{exm}
        Пусть $\varphi := P(0)\vee (P(x)\rightarrow P(x'))$ и $S_P := \{0', 0'', 0'''''\}$, тогда
        $$\begin{array}{l}\llbracket \varphi \rrbracket^{ x:=0 } = \llbracket P(0)\vee (P(0)\rightarrow P(0')) \rrbracket = \text{И}\\
            \llbracket \varphi \rrbracket^{ x:=0'' } = \llbracket P(0'')\vee (P(0'')\rightarrow P(0'''))\rrbracket = \text{Л}\end{array}$$
    \end{exm}

    \subsection{Выполнимость не теряется. Заменяем $D$ на $H$}

    \begin{thm}Формула выполнима тогда и только тогда, когда она выполнима на Эрбрановом универсуме.\end{thm}
    \begin{proof}
        $(\Rightarrow)$ Пусть $M \models\forall \overline{x}.\varphi$. Тогда построим отображение $\text{eval}: H \rightarrow M$
        (смысл названия вдохновлён языками программирования: $\text{eval}(``f(f(b))``)$ перейдёт в $f(f(b))$, где $f$ и $b$ --- из $M$).

        Предикатам дадим согласованную оценку:
        $P_H(t_1,\dots,t_n) = P_M(eval(t_1),\dots,eval(t_n))$. Очевидно, любая формула сохранит своё значение, кванторы всеобщности
        по меньшему множеству также останутся истинными ($\text{eval}(H) \subset D$).

        $(\Leftarrow)$ Очевидно.
    \end{proof}

    \subsection{Преобразуем формулу в КНФ}

    \begin{dfn}КНФ ($c$ конъюнктов, в каждом $d(c)$ дизъюнктов, каждый --- предикатный символ с возможным отрицанием):
        $$\zeta := \forall x_1.\forall x_3\dots\forall x_{n-1}.
        \bigwedge_{k=\overline{1,c}}\left(\bigvee_{i = \overline{1,d(c)}}\delta^k_i\right)$$
        при этом
        $$\delta^k_i := P^k_i(\theta^k_{i,1},\dots,\theta^k_{i,a(k,i)})\text{ или }\delta^k_i := \neg P^k_i(\theta^k_{i,1},\dots,\theta^k_{i,a(k,i)})$$
    \end{dfn}

    \begin{thm} Для любой $\varphi$ найдётся эквивалентная ей формула в КНФ.\end{thm}

    \subsection{Система дизъюнктов}

    \begin{dfn}Система дизъюнктов $S = \{\delta_1,\dots,\delta_n\}$. Каждый $\delta_i$ --- дизъюнкт вида $P_1 \vee \neg P_2 \vee \dots$
    \end{dfn}

    \begin{dfn}Система дизъюнктов $S = \{\delta_1,\dots,\delta_n\}$ противоречива,
    если для каждой оценки $M = \langle D,P,F,E \rangle$ найдётся $\delta_t$ и такой набор $\overline{d} \in D$,
    что $\llbracket\delta_t\rrbracket^{\overline{x} := \overline{d}} = \text{Л}$.
    \end{dfn}

    \begin{thm}Система дизъюнктов противоречива, если она невыполнима в эрбрановых интерпретациях.\end{thm}

    \subsection{Основные примеры.}

    Рассмотрим сколемизированную формулу $\beta$ в КНФ. Заметим, что если $\beta = \forall x_1.\dots.\forall x_k.\delta_1\with\delta_2\with\dots\with\delta_n$,
    то $$\vdash \beta \leftrightarrow (\forall x_1.\dots.\forall x_k.\delta_1)\with\dots\with(\forall x_1.\dots.\forall x_k.\delta_n)$$

    \begin{dfn}
        Дизъюнкт с подставленными значениями из эрбранового универсума $H_\beta$ (как строками) вместо переменных называется основным примером формулы $\beta$.
    \end{dfn}

    \begin{exm}Пусть $\beta := \forall x.P(0) \with (P(x)\vee P(x'))$, тогда $P(0''')\vee P(0'''')$ --- основной пример, а $P(0''''')$ --- нет.
    \end{exm}

    \subsection{Система основных примеров}

    \begin{dfn}Система основных примеров --- все основные примеры, опровергаемые хоть при какой-то эрбрановой интерпретации $\mathcal{M}$:

        $$\mathcal{E}_S = \{ \delta_t[\overline{x} := \overline{d}]\ |\ \text{существует }\mathcal{M}\text{, что } \llbracket \delta_t \rrbracket^{\overline{x} := \overline{d}}_\mathcal{M}=\text{Л};\quad d_i \in H_\beta\}$$
    \end{dfn}

    \begin{dfn}Система основных примеров $E$ противоречива в эрбрановых интерпретациях, если
    для любой эрбрановой интерпретации $\mathcal{M}$ найдётся такой $\varepsilon\in E$,
    что $\llbracket \varepsilon \rrbracket_\mathcal{M} = \text{Л}$.
    \end{dfn}

    \begin{thm}Система дизъюнктов $S$ противоречива тогда и только тогда, когда система её
    основных примеров $\mathcal{E}_S$ противоречива в эрбрановых интерпретациях.\end{thm}

    \subsection{Теорема Эрбрана}

    \begin{thm}[Гёделя о компактности]Если $\Gamma$ --- некоторое семейство бескванторных формул, то $\Gamma$ имеет модель
    тогда и только тогда, когда любое его конечное подмножество имеет модель.\end{thm}

    \begin{thm}[Эрбрана]Система дизъюнктов $S$ противоречива тогда и только тогда, когда у
        $\mathcal{E}_S$ существует конечное противоречивое в эрбрановой интерпретации подмножество.\end{thm}
    \begin{proof}$(\Leftarrow)$
        Пусть $\{\varepsilon_1,\dots,\varepsilon_t\} \subseteq \mathcal{E}_S$ противоречиво, $\varepsilon_i = \delta_{m_i}[\overline{x} := \overline{d_i}]$,
        где $\overline{d_i}$ --- набор значений из $H$.
        То есть, для любой эрбрановой интерпретации $M$ существует $\varepsilon_p$, что $\llbracket\varepsilon_p\rrbracket_M = \text{Л}$.
        Отсюда, по теореме о выполнимости $S$ тоже противоречива.

        $(\Rightarrow)$ Если $S$ противоречива, то $\mathcal{E}_S$ противоречива.
        Тогда у неё нет модели. Тогда у неё найдётся конечное противоречивое подмножество (компактность).
    \end{proof}

    Возможно убедиться в невыполнимости за конечное время.

    \subsection{Правило резолюции (исчисление высказываний)}

    Пусть даны два дизъюнкта, $\alpha_1 \vee \beta$ и $\alpha_2 \vee \neg \beta$.
    Тогда следующее правило вывода называется правилом резолюции:

    $$\infer{\alpha_1\vee \alpha_2}{\alpha_1\vee \beta\quad\quad \alpha_2\vee\neg \beta}$$

    \begin{thm}Система дизъюнктов противоречива, если в процессе всевозможного применения
        правила резолюции будет построено явное противоречие,
        т.е. найдено два противоречивых дизъюнкта: $\beta$ и $\neg\beta$.
    \end{thm}

    \subsection{Правило резолюции для исчисления предикатов}

    \begin{dfn}
        Пусть $\sigma_1$ и $\sigma_2$ --- подстановки, заменяющие переменные в формуле на свежие.
        Тогда правило резолюции выглядит так:

        $$\infer[{\pi = \mathcal{U}\big[\sigma_1(\beta_1),\sigma_2(\beta_2)\big]}]
        {\pi(\sigma_1(\alpha_1)\vee \sigma_2(\alpha_2))}
        {\alpha_1\vee \beta_1\quad\quad\alpha_2\vee\neg \beta_2}$$
    \end{dfn}

    $\sigma_1$ и $\sigma_2$ разделяют переменные у дизъюнктов, чтобы $\pi$ не осуществила лишние
    замены, ведь $\vdash(\forall x.P(x) \with Q(x)) \leftrightarrow (\forall x.P(x))\with(\forall x.Q(x))$, но
    $\not\vdash (\forall x.P(x) \vee Q(x)) \rightarrow (\forall x.P(x))\vee(\forall x.Q(x))$.

    \begin{exm}
        $$\infer[\text{ подстановки: } \sigma_1(x) = x', \sigma_2(x) = x'', \pi(x')=a]{Q(a)\vee T(x'')}{Q(x)\vee P(x) \quad \neg P(a)\vee T(x)}$$
    \end{exm}

    \subsection{Уравнения в алгебраических термах}

    \begin{dfn}Алгебраический терм $$\theta := x\:|\:(f(\theta_1,\ldots,\theta_n))$$
        где $x-$переменная, $f(\theta_1,\ldots,\theta_n)-$применение функции. Напомним, что константы --- нульместные
        функциональные символы, собственно переменные будем обозначать последними буквами латинского алфавита. \end{dfn}

    \begin{dfn}Система уравнений в алгебраических термах
        $\begin{cases}
             \theta_1=\sigma_1\\
             \vdots\\
             \theta_n=\sigma_n\\
        \end{cases}$
        \par где $\theta_i \text{ и } \sigma_i-\text{термы}$\par
    \end{dfn}

    \subsection{Уравнение в алгебраических термах, определение}

    \begin{dfn}$\{x_i\}=X-$множество переменных, $\{\theta_i\}=T-$множество термов.\end{dfn}

    \begin{dfn}Подстановка$-$отображение вида: $\pi_0:X\to T$, тождественное почти везде (за исключением конечного числа переменных).
        \par $\pi_0(x)$ может быть либо $\pi_0(x)=\theta_i\text{, либо }\pi_0(x)=x$.\end{dfn}

    Доопределим $\pi:T\to T$, где \begin{enumerate}
                                      \item $\pi(x)=\pi_0(x)$
                                      \item $\pi(f(\theta_1, \ldots, \theta_k))=f(\pi(\theta_1), \ldots, \pi(\theta_k))$
    \end{enumerate}

    \begin{dfn}Решить уравнение в алгебраических термах$-$найти такую наиболее общую подстановку $\pi$,
    что $\pi(\theta_1)=\pi(\theta_2)$.
    Наиболее общая подстановка --- такая, для которой другие подстановки являются её частными случаями.\end{dfn}

    \subsection{Задачи унификации, наибольший общий унификатор}

    \begin{dfn}
        Пусть даны формулы $\alpha$ и $\beta$. Тогда решением задачи унификации
        будет такая наиболее общая подстановка $\pi = \mathcal{U}\big[\alpha,\beta\big]$, что $\pi(\alpha) = \pi(\beta)$.

        Также, $\pi$ назовём наиболее общим унификатором.
    \end{dfn}

    \begin{exm}
        \begin{itemize}
            \item Формулы $P(a,g(b))$ и $P(c,d)$ не имеют унификатора (мы считаем, что $a,b,c,d$ --- нульместные функции, а
            $g$ --- одноместная функция).

            \item Проверим формулу на соответствие 11 схеме аксиом $(\forall x.\varphi)\rightarrow\varphi[x := \theta]$: $$(\forall x.P(x))\rightarrow P(f(t,g(t),y))$$
            Для этого решим задачу унификации: $\pi = \mathcal{U}\big[P(x),P(f(t,g(t),y))\big]$, тогда $\pi(x) = f(t,g(t),y)$.
        \end{itemize}
    \end{exm}

    \subsection{Общая формулировка метода резолюции}

    Ищем $\vdash\alpha$.

    \begin{enumerate}
        \item будем искать опровержение $\neg\alpha$.
        \item перестроим $\neg\alpha$ в КНФ.
        \item будем применять правило резолюции, пока получаем новые дизъюнкты и пока
        не найдём явное противоречие (дизъюнкты вида $\beta$ и $\neg\beta$).
    \end{enumerate}

    Если противоречие нашлось, значит, $\vdash\neg\neg\alpha$. Если нет --- значит, $\vdash\neg\alpha$.
    Процесс может не закончиться.

    \subsection{SMT-решатели}

    Обычно требуется не логическое исчисление само по себе, а теория первого порядка.
    То есть, <<Satisfability Modulo Theory>>, <<выполнимость в теории>> --- вместо SAT, выполнимости.
    \begin{itemize}
        \item Иногда можно вложить теорию в логическое исчисление,
        даже в исчисление высказываний: $\overline{S_2S_1S_0} = \overline{A_1A_0}+\overline{B_1B_0}$
        $$\begin{array}{ll}
              S_0 = A_0 \oplus B_0 & C_0 = A_0 \with B_0\\
              S_1 = A_1 \oplus B_1 \oplus C_0 & C_1 = (A_1 \with B_1) \vee (A_1 \with C_0) \vee (B_1 \with C_0) \\
              S_2 = C_1\end{array}$$

        \item А можно что-то добавить прямо на уровень унификации / резолюции:
        Например, можем зафиксировать арифметические функции --- и производить вычисления
        в правиле резолюции вместе с унификацией.

        Тогда противоречие в $\{x = 1+3+1,\neg x = 5\}$ можно найти за один шаг.
    \end{itemize}

    \section{Лямбда-исчисление. Пред- и лямбда-термы. Альфа-эквивалентность, бета-редукция, бета-эквивалентность.
    Теорема Чёрча-Россера (формулировка). Нормальная форма и её единственность (с доказательством).
    Представление истины и лжи, чёрчевские нумералы, арифметические функции (сложение, умножение, вычитание).
    Комбинатор неподвижной точки. Импликационный фрагмент ИИВ. Замкнутость импликационного фрагмента ИИВ
        (формулировка). Типизация лямбда-исчисления по Чёрчу и по Карри. Гильбертов вывод и комбинаторы.}

    \subsection{Пред- и лямбда-термы}

    \begin{dfn} Предтермы
        $$\Lambda ::= (\lambda x.\Lambda) | (\Lambda\ \Lambda) | x$$
    \end{dfn}

    Мета-язык:
    \begin{itemize}
        \item Мета-переменные:\begin{itemize}
                                  \item $A\dots Z$ --- мета-переменные для термов.
                                  \item $x,y,z$ --- мета-переменные для переменных.
        \end{itemize}

        \item Правила расстановки скобок аналогичны правилам для кванторов:
        \begin{itemize}
            \item Лямбда-выражение ест всё до конца строки
            \item Аппликация левоассоциативна
        \end{itemize}
    \end{itemize}

    \begin{exm}
        \begin{itemize}
            \item $a\ b\ c\ (\lambda d.e\ f\ \lambda g.h)\ i \equiv \Big({\color{red}\Big(}((a\ b)\ c)\ {\color{blue}\Big(}\lambda d.((e\ f)\ (\lambda g.h)){\color{blue}\Big)}{\color{red}\Big)}\ i\Big)$
            \item $0 := \lambda f.\lambda x.x;\quad(+1) := \lambda n.\lambda f.\lambda x.n\ f\ (f\ x);\quad(+2) := \lambda x.(+1)\ ((+1)\ x)$
        \end{itemize}
    \end{exm}

    \subsection{Альфа-эквивалентность}

    $$FV(A) = \left\{\begin{array}{ll} \{x\}, & A \equiv x\\
        FV(P)\cup FV(Q), & A \equiv P\ Q\\
        FV(P)\setminus\{x\}, & A \equiv \lambda x.P\end{array}\right.$$

    Примеры:
    \begin{itemize}
        \item $M := \lambda b.\lambda c.a\ c\ (b\ c)$; $FV(M) = \{a\}$
        \item $N := x\ (\lambda x.(x\ (\lambda y.x)))$; $FV(N) = \{x\}$
    \end{itemize}

    \begin{dfn}$A=_\alpha B$, если и только если выполнено одно из трёх:
        \begin{enumerate}
            \item $A \equiv x$, $B \equiv y$, $x \equiv y$;
            \item $A \equiv P_a Q_a$, $B \equiv P_b Q_b$ и $P_a =_\alpha P_b$, $Q_a =_\alpha Q_b$;
            \item $A \equiv (\lambda x.P)$, $B \equiv (\lambda y.Q)$, $P[x := t] =_\alpha Q[y := t]$, где $t$ не входит в $A$ и $B$.
        \end{enumerate}\end{dfn}

    \begin{dfn}$L = \Lambda/_{=_\alpha}$\end{dfn}

    \subsection{Альфа-эквивалентность, пример}

    \begin{lmm}
        $$\lambda a.\lambda b.a\ b =_\alpha \lambda b.\lambda a.b\ a$$
    \end{lmm}

    \begin{proof}
        \begin{center}\begin{tabular}{rcll}
                          $t$ & $=_\alpha$ &$t$& Правило 1\\
                          $s$ & $=_\alpha$ &$s$& Правило 1\\
                          $t\ s$ & $=_\alpha$ &$t\ s$& Правило 2\\
                          $\lambda b.(t\ b)$ & $=_\alpha$ &$\lambda a.(t\ a)$ & Правило 3\\
                          $\lambda a.\lambda b.(a\ b)$ & $=_\alpha$ &$\lambda b.\lambda a.(b\ a)$ & Правило 3\\
            \end{tabular}\end{center}
    \end{proof}

    \subsection{Бета-редукция}

    Интуиция: вызов функции.
    \begin{center}\begin{tabular}{l|l}
                      $\lambda$-выражение & Python \\\hline
                      $\lambda f.\lambda x.f\ x$ & \texttt{def one(f,x): return f(x)}\\
                      $(\lambda x.x\ x)\ (\lambda x.x\ x)$ & \texttt{(lambda x: x x) (lambda x: x x)}\\
                      &   \texttt{def omega(x): return x(x); omega(omega)}
        \end{tabular}\end{center}

    \begin{dfn} Терм вида $(\lambda x.P)\ Q$ --- бета-редекс.\end{dfn}
    \begin{dfn} $A \rightarrow_\beta B$, если:
        \begin{enumerate}
            \item $A \equiv (\lambda x.P)\ Q$, $B \equiv P\ [x := Q]$, при условии свободы для подстановки;
            \item $A \equiv (P\ Q)$, $B \equiv (P'\ Q')$, при этом $P \rightarrow_\beta P'$ и $Q = Q'$, либо $P = P'$ и $Q \rightarrow_\beta Q'$;
            \item $A \equiv (\lambda x.P)$, $B \equiv (\lambda x.P')$, и $P \rightarrow_\beta P'$.
        \end{enumerate}
    \end{dfn}
    \begin{dfn}$(\twoheadrightarrow_\beta)$ --- транзитивное и рефлексивное замыкание $(\rightarrow_\beta)$.\end{dfn}

    \begin{dfn}Бета-эквивалентность$(=_\beta)$ --- транзитивное, рефлексивное и симметричное замыкание $(\rightarrow_\beta)$.\end{dfn}

    \subsection{Бета-редукция, пример}

    \begin{exm}
        $(\lambda x.x\ x)\ (\lambda n.n) \rightarrow_\beta (\lambda n.n)\ (\lambda n.n) \rightarrow_\beta \lambda n.n$
    \end{exm}
    \begin{exm}
        $(\lambda x.x\ x)\ (\lambda x.x\ x) \rightarrow_\beta (\lambda x.x\ x)\ (\lambda x.x\ x)$
    \end{exm}

    \subsection{Теорема Чёрча-Россера}

    \begin{thm}[Чёрча-Россера] Для любых термов $N$, $P$, $Q$, если $N \twoheadrightarrow_\beta P$, $N \twoheadrightarrow_\beta Q$,
    и $P \ne Q$, то найдётся $T$: $P \twoheadrightarrow_\beta T$ и $Q \twoheadrightarrow_\beta T$.\end{thm}

    \subsection{Нормальная форма}

    \begin{dfn}Лямбда-терм $N$ находится в нормальной форме, если нет $Q$: $N \rightarrow_\beta Q$.\end{dfn}

    \begin{exm}В нормальной форме:\\
        $\lambda f.\lambda x.x\ (f\ (f\ \lambda g.x))$\end{exm}

    \begin{exm}Не в нормальной форме (редексы подчёркнуты):\\
        $\lambda f.\lambda x.\underline{(\lambda g.x)\ (f\ (f\ x))}$\\
        $(\underline{(\lambda x.x)\ (\lambda x.x)})\ (\underline{(\lambda x.x)\ (\lambda x.x)})$
    \end{exm}

    \begin{thm}Если у терма $N$ существует нормальная форма, то она единственна\end{thm}
    \begin{proof}Пусть не так и $N \twoheadrightarrow_\beta P$ вместе с $N \twoheadrightarrow_\beta Q$, $P \ne Q$.
    Тогда по теореме Чёрча-Россера существует $T$: $P \twoheadrightarrow_\beta T$ и $Q \twoheadrightarrow_\beta T$,
    причём $T \ne P$ или $T \ne Q$ в силу транзитивности $(\twoheadrightarrow_\beta)$.\end{proof}

    \subsection{Представление истины и лжи}

    $T := \lambda x.\lambda y.x$

    $F := \lambda x.\lambda y.y$

    Тогда: $Or := \lambda a.\lambda b.a\ T\ b$:

    $Or\ F\ T = \underline{((\lambda a.\lambda b.a\ T\ b)\ F)}\ T \rightarrow_\beta (\lambda b.F\ T\ b)\ T
    \rightarrow_\beta F\ T\ T =$

    $=(\lambda x.\lambda y.y)\ T\ T\rightarrow_\beta (\lambda y.y)\ T \rightarrow_\beta T$

    \subsection{Чёрчевские нумералы, арифметические функции}

    $$f^{(n)}(x) = \left\{\begin{array}{ll}x, & n = 0\\f(f^{(n-1)}(x)), & n > 0\end{array}\right.$$

    \begin{dfn}
        Чёрчевский нумерал $\overline{n} = \lambda f.\lambda x.f^{(n)}(x)$
    \end{dfn}

    \begin{exm}
        $\overline{3} = \lambda f.\lambda x.f(f(f(x)))$

        Инкремент: $Inc = \lambda n.\lambda f.\lambda x.n\ f\ (f\ x)$
    \end{exm}

    $$\begin{array}{l}(\lambda n.\lambda f.\lambda x.n\ f\ (f\ x))\ \overline{0} = (\lambda n.\lambda f.\lambda x.n\ f\ (f\ x))\ (\lambda f'.\lambda x'.x') \rightarrow_\beta \\
        \dots\lambda f.\lambda x.(\lambda f'.\lambda x'.x')\ f\ (f\ x) \rightarrow_\beta \\
        \dots\lambda f.\lambda x.(\lambda x'.x')\ (f\ x) \rightarrow_\beta \\
        \dots\lambda f.\lambda x.f\ x = \overline{1}\end{array}$$

    Декремент: $Dec = \lambda n.\lambda f.\lambda x.n\ (\lambda g.\lambda h.h\ (g\ f))\ (\lambda u.x)\ (\lambda u.u)$

    \subsection{Комбинатор неподвижной точки}

    \begin{exm}$\Omega = (\lambda x.x\ x)\ (\lambda x.x\ x)$ не имеет нормальной формы:
        $\Omega \rightarrow_\beta \Omega$\end{exm}

    \begin{thm}Для любого терма $N$ найдётся такой терм $R$, что $R =_\beta N\ R$.\end{thm}
    \begin{proof}Пусть $Y = \lambda f.(\lambda x.f\ (x\ x))\ (\lambda x.f\ (x\ x))$.
    Тогда $R := Y\ N$:

        $$Y\ N =_\beta (\lambda x.N\ ({\color{red}x}\ {\color{blue}x}))\ (\lambda x.N\ (x\ x)) =_\beta N\ (({\color{red}\lambda x.N\ (x\ x)})\ ({\color{blue}\lambda x.N\ (x\ x)}))$$
    \end{proof}

    \subsection{Импликационный фрагмент ИИВ}

    \begin{dfn}Импликационный фрагмент интуиционистской логики:

        $$\infer{\Gamma,\varphi \vdash_\rightarrow \varphi}{} \quad\quad
        \infer{\Gamma\vdash_\rightarrow\varphi\rightarrow\psi}{\Gamma,\varphi\vdash_\rightarrow\psi} \quad\quad
        \infer{\Gamma\vdash_\rightarrow\psi}{\Gamma\vdash_\rightarrow\varphi\quad\quad\Gamma\vdash_\rightarrow\varphi\rightarrow\psi}$$
    \end{dfn}

    \begin{thm}(Замкнутость импликационного фрагмента) Если $\Gamma\vdash\alpha$, то $\Gamma\vdash_\rightarrow\alpha$.\end{thm}
    \begin{proof}
        Определим модель Крипке: \begin{itemize}
                                     \item миры --- замкнутые множества формул: $\alpha\in\Gamma$ т.и.т.т. $\Gamma\vdash_\rightarrow\alpha$,
                                     \item порядок --- $(\subseteq)$,
                                     \item $\Gamma\Vdash X$ т.и.т.т. $X\in\Gamma$.
        \end{itemize}

        Из корректности моделей Крипке следует, что что если $\Gamma\vdash\alpha$, то $\Gamma\Vdash \alpha$.
        Требуемое следует из того, что $\Gamma\Vdash \alpha$ влечёт $\Gamma\vdash_\rightarrow\alpha$.
    \end{proof}

    \subsection{Замкнутость импликационного фрагмента ИИВ}

    Индукция по структуре $\alpha$.
    \begin{itemize}
        \item $\alpha \equiv X$. Утверждение следует из определения;
        \item $\alpha \equiv \varphi\rightarrow\psi$.
        \begin{itemize}
            \item Пусть $\Gamma\Vdash \varphi\rightarrow\psi$. То есть, $\Gamma \subseteq \Delta$ и $\Delta \Vdash \varphi$
            влечёт $\Delta \Vdash \psi$.
            Возьмём $\Delta$ как замыкание $\Gamma\cup\{\varphi\}$. Значит, $\Delta\vdash_\rightarrow\varphi$
            и, по индукционному предположению, $\Delta\Vdash\varphi$.
            Тогда $\Delta\Vdash\psi$. По индукционному предположению, $\Delta\vdash_\rightarrow\psi$.
            То есть, $\Gamma,\varphi\vdash_\rightarrow\psi$, откуда

            $$\infer{\Gamma\vdash\varphi\rightarrow\psi}{\Gamma,\varphi\vdash\psi}$$
            \item Пусть $\Gamma\vdash_\rightarrow\varphi\rightarrow\psi$. Проверим $\Gamma\Vdash\varphi\rightarrow\psi$.
            Пусть $\Gamma \subseteq \Delta$ и пусть $\Delta\Vdash\varphi$.

            По индукционному предположению, $\varphi\in\Delta$.
            То есть, $\Delta\vdash_\rightarrow\varphi$ и $\Delta\vdash_\rightarrow\varphi\rightarrow\psi$.
            Тогда $$\infer{\Delta\vdash_\rightarrow\psi}{\Delta\vdash_\rightarrow\varphi\quad\Delta\vdash_\rightarrow\varphi\rightarrow\psi}$$

            По индукционному предположению, $\Delta\Vdash\psi$, отчего $\Gamma\Vdash\varphi\rightarrow\psi$.
        \end{itemize}
    \end{itemize}

    \subsection{Типизация лямбда-исчисления по Чёрчу и по Карри}

    \begin{dfn}
        Просто-типизированное лямбда-исчисление (по Карри). Типы: $\tau ::= \alpha | (\tau\rightarrow\tau)$. Язык: $\Gamma\vdash A:\varphi$
        $$\infer[x \notin \Gamma]{\Gamma,x:\varphi \vdash x:\varphi}{} \quad\quad
        \infer[x \notin \Gamma]{\Gamma\vdash \lambda x.A: \varphi\rightarrow\psi}{\Gamma,x:\varphi\vdash A:\psi} \quad\quad
        \infer{\Gamma\vdash B A:\psi}{\Gamma\vdash A:\varphi\quad\quad\Gamma\vdash B:\varphi\rightarrow\psi}$$
    \end{dfn}

    \begin{dfn}
        Просто-типизированное лямбда-исчисление по Чёрчу.
        $$\infer[x \notin \Gamma]{\Gamma,x:\varphi \vdash x:\varphi}{} \quad\quad
        \infer[x \notin \Gamma]{\Gamma\vdash \lambda x^{\color{blue}\varphi}.A: \varphi\rightarrow\psi}{\Gamma,x:\varphi\vdash A:\psi} \quad\quad
        \infer{\Gamma\vdash B A:\psi}{\Gamma\vdash A:\varphi\quad\quad\Gamma\vdash B:\varphi\rightarrow\psi}$$
    \end{dfn}

    \begin{exm}
        \begin{tabular}{l|l}
            По Карри & По Чёрчу\\\hline
            $\lambda f.\lambda x.f\ (f\ x) : (\alpha\rightarrow\alpha)\rightarrow(\alpha\rightarrow\alpha)$ & $\lambda f^{\alpha\rightarrow\alpha}.\lambda x^\alpha.f\ (f\ x) : (\alpha\rightarrow\alpha)\rightarrow(\alpha\rightarrow\alpha)$\\
            $\lambda f.\lambda x.f\ (f\ x) : (\beta\rightarrow\beta)\rightarrow(\beta\rightarrow\beta)$ & $\lambda f^{\beta\rightarrow\beta}.\lambda x^\beta.f\ (f\ x) : (\beta\rightarrow\beta)\rightarrow(\beta\rightarrow\beta)$
        \end{tabular}
    \end{exm}

    \subsection{Пример: тип чёрчевских нумералов}

    Пусть $\Gamma = f:\alpha\rightarrow\alpha, x: \alpha$

    $$\infer[\lambda]{\vdash \lambda f.\lambda x.f\ (f\ x) : (\alpha\rightarrow\alpha)\rightarrow(\alpha\rightarrow\alpha)}{
        \infer[\lambda]{f: \alpha\rightarrow\alpha \vdash \lambda x.f\ (f\ x) : (\alpha\rightarrow\alpha)}{
            \infer[App]{{\color{blue}\{ f:\alpha\rightarrow\alpha, x: \alpha\}\ } \vdash f\ (f\ x): \alpha}{
                \infer[App]{\Gamma \vdash f\ x: \alpha}{
                    \infer[Ax]{\Gamma \vdash x: \alpha}{}\quad\quad\infer[Ax]{\Gamma \vdash f: \alpha\rightarrow\alpha}{}
                }\quad\quad
                \infer[Ax]{\Gamma \vdash f: \alpha\rightarrow\alpha}{}
            }
        }
    }$$

    \subsection{Изоморфизм Карри-Ховарда: отрицание}

    \begin{dfn}Ложь ($\bot$) --- необитаемый тип;
        $\texttt{failwith/raise/throw} : \alpha\rightarrow\bot$; $\neg\varphi\equiv\varphi\rightarrow\bot$
    \end{dfn}

    Например, контрапозиция:
    $(\alpha\rightarrow\beta)\rightarrow(\neg\beta\rightarrow\neg\alpha)$

    $$\infer[\lambda]{\lambda f^{\alpha\rightarrow\beta}.\lambda n^{\beta\rightarrow\bot}.\lambda a^\alpha.n\ (f\ a): (\alpha\rightarrow\beta)\rightarrow(\neg\beta\rightarrow\neg\alpha)}
    {\infer[\lambda]{f:\alpha\rightarrow\beta\vdash\lambda n^{\beta\rightarrow\bot}.\lambda a^\alpha.n\ (f\ a): \neg\beta\rightarrow\neg\alpha}
    {\infer[\lambda]{f:\alpha\rightarrow\beta,n:\beta\rightarrow\bot\vdash\lambda a^\alpha.n\ (f\ a): \neg\alpha}{
        \infer[App]{f:\alpha\rightarrow\beta,n:\beta\rightarrow\bot, a:\alpha \vdash n\ (f\ a): \bot}{
            \infer[App]{\Phi \vdash f\ a: \beta}{\infer[Ax]{\Phi \vdash a: \alpha}{}\quad\quad\infer[Ax]{\Phi \vdash f:\alpha\rightarrow\beta}{}}
            \quad\quad \infer[Ax]{\Phi \vdash n: \beta\rightarrow\bot}{}
        }
    }}}$$

    \subsection{Гильбертов вывод и комбинаторы}

    \begin{dfn}Комбинатор --- лямбда-терм без свободных переменных
    \end{dfn}
    \begin{dfn}[исходная идея Моисея Шейнфинкеля, 1924]$S := \lambda x.\lambda y.\lambda z.x\ z\ (y\ z)$, $K := \lambda x.\lambda y.x$, $I := \lambda x.x$\\
    (verSchmelzung, Konstanz --- исходно <<C>> у Шейнфинкеля, Identit\"at)
    \end{dfn}

    \begin{thm}Пусть $N$ --- некоторый замкнутый лямбда-терм. Тогда найдётся выражение $M$, состоящее из комбинаторов $S$,$K$,
    что $N =_\beta M$\end{thm}

    \begin{exm}
        $$I =_\beta S\ K\ K$$
    \end{exm}\vspace{-0.3cm}
    \begin{tabular}{ll}
        $K := \lambda x^\alpha.\lambda y^\beta.x$ & $\alpha\rightarrow\beta\rightarrow\alpha$\\
        $S := \lambda x^{\alpha\rightarrow\beta\rightarrow\gamma}.\lambda y^{\alpha\rightarrow\beta}.\lambda z^\alpha.x\ z\ (y\ z)$ & $(\alpha\rightarrow\beta\rightarrow\gamma)\rightarrow(\alpha\rightarrow\beta)\rightarrow\alpha\rightarrow\gamma$\\
    \end{tabular}
    Изоморфизм Карри-Ховарда: вывод в гильбертовом стиле --- комбинаторное представление.



    \section{Модальная логика, системы K, K4, T, S4, S5. Линейная темпоральная логика. Построение формулы для выбранного
    	инварианта (например, условия на семафоры критической секции для двух потоков).
    	Проверка на моделях, постановка задачи. Система переходов. Автоматы Бюхи.
    	Построение автомата Бюхи по данной формуле ЛТЛ. Схема алгоритма, проверяющего с помощью моделей
    	соответствие алгоритма утверждению в ЛТЛ.}
    
    \subsection{Модальные логики}
    
    \begin{itemize}
    	\item Модальность (лат. modus — способ, вид) — способ, вид бытия или события; 
    	категории модальности: возможность, действительность, необходимость.
    	
    	\item Расширяем язык: как бы выразить <<модальности>>?
    	\emph{Всегда} зимой идёт снег. Дождь \emph{может} идти при солнечном свете.
    	
    	\item Модифицируем язык, модифицируем аксиоматику, модифицируем теорию моделей.
    	\item Язык предполагает включение новых связок, самые типичные:
    	\begin{tabular}{ll}
    		$\Box$ & необходимость (necessity)\\
    		$\Diamond$ & возможность (possibility)
    	\end{tabular}
    	
    	\item Интуитивный смысл связок примерно понятен, конкретный смысл формализуется
    	в конкретной теории.
    	
    \end{itemize}
    
    \subsection{Некоторые модальные исчисления (обзор)}
    
    Терминология введена Кларенсом Льюисом и Купером Лангфордом в 1932 году.
    \begin{itemize}
    	\item Минимальная модальная логика (K) строится поверх ИВ:
    	$$\text{Аксиомы ИВ}\quad\quad\Box(\varphi\rightarrow\psi)\rightarrow(\Box\varphi\rightarrow\Box\psi)
    	\quad\quad\infer{\psi}{\varphi\quad\varphi\rightarrow\psi}\quad\quad\infer{\Box\varphi}{\varphi}$$
    	
    	\item K4: K и дополнительная аксиома транзитивности  $\Box\varphi \rightarrow \Box\Box\varphi$;
    	\item T: K и дополнительная аксиома рефлексивности $\Box\varphi \rightarrow \varphi$;
    	\item S4: K4 + T;
    	\item S5: T и аксиома $\Diamond\varphi\rightarrow\Box\Diamond\varphi$
    	(<<если что-то возможно, то оно обязательно реализуется в каком-то мире>>).
    	Например, утверждение <<если Бог возможен, то он необходимо существует>>
    	можно формализовать и доказать в S5.
    \end{itemize}
    
    \subsection{Линейная темпоральная логика}
    
    \begin{itemize}
    	\item Темпоральная логика: множественные миры (в стиле моделей Крипке) понимаются как 
    	расположенные в соответствие с течением времени(Специальная теория моделей, чтобы удобно задавать оценку, которая будет соответствовать реальному процессу во времени).
    	\item Линейная темпоральная логика: миры выстроены в линейном порядке.
    	\item Используем следующие связки:
    	\begin{enumerate}
    		\item $\mathcal{G}(\alpha)$ или $\Box\alpha$: утверждение $\alpha$ выполнено в любой момент (начиная с текущего).
    		\item $\mathcal{P}(\alpha)$ или $\bigcirc\alpha$: утверждение $\alpha$ выполнено в следующий момент.
    		\item $\mathcal{E}(\alpha)$ или $\Diamond\alpha$: утверждение $\alpha$ неизбежно выполнено в будущем, в какой-то момент (начиная с текущего).
    		\item $\mathcal{U}(\alpha,\beta)$ или $\alpha\mathcal{U}\beta$: утверждение $\alpha$ истинно, пока $\beta$ не станет истинным,
    		после чего $\alpha$ может быть любым.
    	\end{enumerate}
    \end{itemize}
    
    \subsection{Представление моделей ЛТЛ как множества слов}
    
    $$W(\varphi) = \{\sigma\in (\mathcal{P}(a))^\omega\ |\ \sigma\models\varphi \}$$
    На строке $\sigma = S_0S_1S_2\dots$ (каждый $S_i \subseteq \mathcal{P}(a)$) истинность задаётся так:
    $$\begin{array}{ll}
    	\sigma\models\top & \text{всегда}\\
    	\sigma\models a & a \in S_0\\
    	\sigma\models \varphi_1 \with \varphi_2 & \sigma\models\varphi_1 \text{ и } \sigma\models\varphi_2\\
    	\sigma\models\neg\varphi & \sigma\not\models\varphi\\
    	\sigma\models\bigcirc\varphi & \sigma[1\dots]\models\varphi\\
    	\sigma\models\varphi_1\mathcal{U}\varphi_2 & \text{существует }j.\sigma[j\dots]\models\varphi_2 \text{ и при всех }i, 0 \le i < j. \sigma[i\dots]\models\varphi_1
    \end{array}$$
    
    \subsection{Выразимость связок, другие формулы}
    
    Будем рассматривать следующую грамматику для формул: 
    $$\varphi ::= \top\ |\ a\ |\ \varphi\with\varphi\ |\ \neg\varphi\ |\ \bigcirc\varphi\ |\ \varphi\mathcal{U}\varphi$$
    поскольку остальные связки выражаются через эти. 
    
    В самом деле, имеем следующие тождества:
    \begin{itemize}
    	\item Связки выражаются друг через друга:
    	\begin{itemize}
    		\item $\Box\alpha = \neg\Diamond\neg\alpha$
    		\item $\Diamond\alpha = \top \mathcal{U} \alpha$
    	\end{itemize}
    	\item Правила двойственности:
    	\begin{itemize}
    		\item $\neg\bigcirc\varphi=\bigcirc\neg\varphi$
    		\item $\neg\Diamond\varphi = \Box\neg\varphi$
    		\item $\neg\Box\varphi = \Diamond\neg\varphi$
    	\end{itemize}
    	\item Правила расширения:
    	\begin{itemize}
    		\item $\varphi\mathcal{U}\psi = \psi \vee (\varphi \with \bigcirc (\varphi \mathcal{U} \psi))$
    		\item $\Diamond \varphi = \varphi \vee \bigcirc \Diamond \varphi$
    		\item $\Box \varphi = \varphi \with \bigcirc \Box \varphi$
    	\end{itemize}
    \end{itemize}
    
    \subsection{Постановка задачи}
    
    \begin{dfn}Системой переходов назовём граф состояний, в котором каждое состояние отражает
    	содержимое памяти компьютера, а переходы соответствуют инструкциям (операциям), выполняемым
    	компьютером\end{dfn}
    
    Хотим научиться проверять, выполнено ли $\varphi$ при всех возможных вариантах выполнения программы,
    то есть при всех возможных путях в системе переходов $TS$:
    
    $$TS \models \varphi$$
    
    Очевидно, $TS$ задаёт некоторое множество (бесконечных) строк в алфавите $2^{FV(\varphi)}$.
    Находится ли в этом множестве строка, удовлетворяющая $\varphi$?
    
    \subsection{Недетерменированные (обобщённые) автоматы Бюхи}
    
    \begin{dfn}
    	НАБ (НОАБ) определяется внешним алфавитом $A$, множествами состояний $Q$, функцией переходов 
    	$\delta: A\times Q \rightarrow \mathcal{P}(Q)$ и семейством допускающих множеств состояний
    	$\mathcal{F} \subseteq \mathcal{P}(Q)$.
    	
    	Бесконечная строка $\alpha = a_0a_1a_2\dots$ допускается недетерменированным (обобщённым) автоматом Бюхи, 
    	если найдётся такая последовательность состояний $q_0q_1q_2\dots$, что
    	$q_{n+1}\in\delta(a_n,q_n)$ и в процессе применения автомата к ней каждое из множеств допускающих 
    	состояний будет 
    	посещено бесконечное количество раз.
    	
    	$$\forall F\in\mathcal{F}.\forall n\in\mathbb{N}.\exists m > n.q_m\in F$$
    	
    	В случае $|\mathcal{F}| \le 1$ такой автомат --- НАБ, иначе --- НОАБ.
    \end{dfn}
    
    \subsection{Пример обобщённого недетерменированного автомата Бюхи}
    
    \begin{exm}
    	Рассмотрим автомат с двумя состояниями (0,1), начальным состоянием 0, и $\mathcal{F} = \{\{0\},\{1\}\}$.
    	
    	\begin{center}
    		\begin{tikzpicture}[->,shorten >=1pt,auto,node distance=2.8cm,
    			semithick]
    			\tikzstyle{vertex}=[draw,text=black]
    			
    			\node[vertex,double,circle] (S) {$0$};
    			\node[vertex,double,circle] (E) [right of=S] {$1$};
    			\node (EE) [right of=E] {};
    			
    			\draw [<-] (S) to node[auto] {} ++ (-1,0);
    			\path %(S) edge   node  {$\varepsilon$}(A)
    			(E) edge  [bend right=40] node [swap] {$b$}(S)
    			(S) edge  [bend right=40] node [swap] {$a$}(E);
    			\draw (E) to [out=375,in=345,looseness=8] node [above] {$b$} (E);
    		\end{tikzpicture}
    	\end{center}
    	
    	Строка $(ab)^\omega$ будет принята, строка $a(b^\omega)$ --- нет.
    \end{exm}
    
    \subsection{Строим автомат Бюхи для формулы: состояния}
    
    \begin{itemize}
    	\item Раскроем сокращения записи (выразим $\Box$, $\vee$, $\rightarrow$ и $\Diamond$ через другие связки).
    	\item Рассмотрим $\mathcal{B}(\varphi)$ --- семейство всех подформул $\varphi$ (и их отрицаний,
    	с учётом $\varphi = \neg\neg\varphi$), 
    	образующих непротиворечивое максимальное множество. Скажем, для $a\vee\neg b$ это будет
    	$\{\{\neg a,\neg b,a\vee\neg b\},\{a,\neg b,a \vee\neg b\},
    	\{\neg a,b,\neg(a\vee\neg b)\},\{a,b,a\vee\neg b\}\}$
    	\item Поскольку множество содержит модальные операторы, непротиворечивость также должна
    	соответствовать условиям реализуемости: при рассмотрении подформулы $\varphi\mathcal{U}\psi$ должно быть
    	$$\psi\in B \Rightarrow \varphi\mathcal{U}\psi \in B\quad\quad
    	\varphi\mathcal{U}\psi \in B \text{ и }\psi \notin B \Rightarrow \varphi \in B$$
    	\item Состояния автомата --- $B_n \in \mathcal{B}(\varphi)$.
    	\item Стрелки подписаны состоянием переменных $A$, и их может быть несколько одинаково
    	подписанных, поскольку в силу модального характера значение формулы не исчерпывается 
    	значением переменных.
    \end{itemize}
    
    \subsection{Строим автомат Бюхи для формулы: переходы}
    
    Рассмотрим состояние $B$ и набор переменных $A \in \mathcal{P}(FV(\varphi))$.
    \begin{itemize}
    	\item Рассмотрим $B': A = B\cup FV(\varphi)$ --- состояния, в которых
    	пропозициональные переменные соответствуют ожидаемому набору переменных.
    	\item Тогда $B' \in \delta(A,B)$, если и только если для каждого из $\psi\in B$
    	выполнено одно из следующих условий:
    	\begin{enumerate}
    		\item 
    		если $\psi$ --- не модальный оператор (здесь неявная рекурсия по структуре $\psi$), то $\psi\in B'$;
    		\item если $\psi\equiv\bigcirc\varphi$, то $\varphi\in B'$;
    		\item если $\psi \equiv \varphi_1\mathcal{U}\varphi_2$, то выполнен закон расширения для $\mathcal{U}$:
    		\begin{itemize}
    			\item либо $\varphi_2 \in B$ ($\mathcal{U}$ активирован в текущем состоянии).
    			\item либо $\varphi_1 \in B$ и $\varphi_1 \mathcal{U} \varphi_2 \in B'$
    			($\mathcal{U}$ будет активирован позже).
    		\end{itemize}
    	\end{enumerate}
    \end{itemize}
    
    \subsection{Строим автомат Бюхи для формулы: допускающие состояния}
    
    Автомат для формулы $\varphi$:
    
    $$\begin{array}{ll}Q &:= \mathcal{B}(\varphi) \\ Q_0 &:= \{ B\ |\ \varphi\in B, B\in Q\} \\
    	F_{\psi_1\mathcal{U}\psi_2} &:= \{B\ |\ \psi_1\mathcal{U}\psi_2 \not\in B \text{ или } \psi_2\in B\}, \\
    	\mathcal{F}&:=\{F_{\psi_1\mathcal{U}\psi_2}\}\end{array}$$
    
    Идея в том, что автомат окажется в допускающем состоянии относительно $\psi_1\mathcal{U}\psi_2$:
    \begin{itemize}
    	\item либо, когда формула $\psi_1\mathcal{U}\psi_2$ не нужна для результата;
    	\item либо в тот момент, когда соответствующая 
    	формула $\psi_1\mathcal{U}\psi_2$ <<активируется>> --- оператор меняет фокус восприятия
    	с ранее истинного $\psi_1$ на истинный $\psi_2$.
    \end{itemize}
    
    \subsection{Разрешимость задачи $TS \models \varphi$ в ЛТЛ}
    
    \begin{proof}[Идея алгоритма]
    	\begin{enumerate}
    		\item Построим обобщённый недетерминированный автомат Бюхи для формулы $\neg\varphi$
    		(принимает последовательность исполнения тогда и только тогда, когда она 
    		опровергает $\varphi$);
    		\item построим недетерминированный автомат по системе переходов $TS$;
    		\item построим их пересечение --- и преобразуем автомат в недетерменированный автомат Бюхи $T$
    		(с множеством допускающих состояний $F$).
    		\item проверим $\mathcal{L}_T = \varnothing$ (значит, $\varphi$ доказано), 
    		либо найдём контрпример --- последовательность исполнения, имеющая цикл, затрагивающий
    		состояние из $F$ (обход графа состояний; алгоритм заканчивается в силу конечности
    		множества состояний);
    		\item данная последовательность будет контрпримером к задаче $TS \models \varphi$.
    	\end{enumerate}
    \end{proof}

\end{document}
